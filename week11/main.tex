\documentclass{beamer}

% \usepackage[utf8]{inputenc}
% \usepackage[T1]{fontenc}
\usepackage{lmodern}   % modern Latin Modern fonts
\usepackage{textcomp}  % provides \textquoteright
\usepackage{lmodern} % Latin Modern fonts with T1 shapes


\usepackage{graphicx}
\usepackage{ragged2e} % for generating dummy text
\usepackage[backend=biber,style=authoryear]{biblatex}
% \addbibresource{references.bib}

\usetheme{Madrid}
\usecolortheme{default}
\usefonttheme{professionalfonts} % keeps proper math fonts

\usepackage{amsmath,amssymb,amsfonts} % math symbols (\mathcal, \mathbb, etc.)
\usepackage{mathrsfs}                 % optional: \mathscr for fancy script

% \setbeamercovered{invisible} 
\setbeamercovered{transparent}


\title{MF921 Topics in Dynamic Asset Pricing}
\subtitle{Week 11}
\author{Yuanhui Zhao}
\date{Boston University}

\begin{document}
\frame{\titlepage}
% \begin{frame}
% \frametitle{Outline}
% \tableofcontents
% \end{frame}

\begin{frame}{Chapter 14}

    {
    \begin{center}
        An introduction to Finite Difference methods for PDEs in Finance
    \end{center}
     \begin{center}
        Agnès Tourin
    \end{center}
    }
    
\end{frame}

\begin{frame}{Introduction}

    {\footnotesize \footnotesize
    \par This note focuses on the practical design of finite difference methods for solving
     Hamilton–Jacobi–Bellman (HJB) equations that arise in quantitative finance. These equations typically model 
     optimal decision-making over time, such as in portfolio selection or option pricing. Because these problems involve both time evolution and uncertainty (stochastic behavior), 
    the resulting partial differential equations (PDEs) are usually of parabolic type.
    \vspace{1em}
    \par \pause A key framework used in the note is the Barles–Souganidis convergence theory. 
    It shows that if a numerical scheme is consistent (approximates the correct PDE), monotone 
    (preserves order in the solution), and stable (bounded behavior over time), 
    then the numerical solution will converge to the correct viscosity solution of the HJB equation.
    \vspace{1em}
    \par  \pause Finite difference methods are most useful in problems with low spatial dimension (1 to 3 variables). 
    Their simplicity and ease of implementation make them attractive for many practical finance problems. 
    In higher dimensions, they can be combined with Monte Carlo methods to retain computational tractability.
    }
    
\end{frame}

\begin{frame}{Barles-Souganidis framework}

    {\footnotesize \footnotesize
     Consider the parabolic PDE
\begin{equation}
u_t + F(t, x, u, Du, D^2u) = 0 \text{ in } (0, T] \times \mathbb{R}^N
\end{equation}
\begin{equation}
u(0, x) = u_0(x) \text{ in } \mathbb{R}^N
\end{equation}

where \( F \) is Elliptic
\[
F(t, x, u, p, A) \leq F(t, x, u, p, B), \text{ if } A \geq B.
\]

 \pause For a sake of simplicity, we assume that \( u_0 \) is bounded in \( \mathbb{R}^N \).
 Furthermore, we assume that (1), (2) satisfy a strong comparison principle.
 \vspace{1em}
 \par Strong Comparison Principle
  \vspace{1em}
 \par  \pause For (1), (2), If $u$ is a viscosity subsolution (upper semi-continuous, bounded)
 and $v$ is a viscosity supersolution (lower semi-continuous, bounded).
 Then:
\[
u(t,x) \leq v(t,x) \quad \text{for all } (t,x) \in [0,T] \times \mathbb{R}^N
\]

This guarantees uniqueness: if two solutions start from the same initial condition, they cannot cross.
    }
    
\end{frame}

\begin{frame}{Barles-Souganidis framework}

    {\footnotesize \footnotesize
      The PDE operator $F$ often comes from stochastic control problems, like optimal trading or investment.

Here, $F$ is defined as:
\[
F(t,x,r,p,X) = \sup_{\alpha \in A} \left\{ -\mathrm{tr}[a^\alpha(t,x)X] 
- b^\alpha(t,x)p - c^\alpha(t,x)r - f^\alpha(t,x)\right\},
\]
where $a^\alpha = \frac{1}{2}\sigma^\alpha\sigma^{\alpha T}$.
\vspace{1em}
\par  \pause Typically, the set of control $\mathcal{A}$ is compact or finite, 
all the coefficients in the equations are bounded and Lipschitz continuous in $x$,
 Hölder with coefficient $\frac{1}{2}$ in $t$ and all the bounds are independent of $\alpha$. 
 Then the unique viscosity solution $u$ of (1) is a bounded and Lipschitz continuous function and 
 is the solution of the underlying stochastic control problem. The ideas, concepts and techniques actually 
 apply to a broader range of optimal control problems. In particular, you can adapt the techniques 
to handle different situations, even possibly treat some delicate singular control problems.
    }
    
\end{frame}

\begin{frame}{Barles-Souganidis framework}

    {\footnotesize \footnotesize
    \par The aim is to build an approximation scheme which preserves the Ellipticity. 
    This discrete Ellipticity property is called monotonicity. The monotonicity,
     together with the consistency of the scheme and some regularity 
    and boundedness conditions ensure its convergence to the unique viscosity solution of the PDE (1),(2).
    Without monotonicity, even a consistent scheme can converge to the wrong solution
    \vspace{1em}
    \par  \pause The theory of viscosity solutions relies on comparison principles: the idea that a subsolution is always less than a supersolution.
    A non-monotone scheme can break this order at the discrete level. A computed subsolution becomes larger than the supersolution
    and the whole logic of convergence collapses.
    \vspace{1em}
    \par  \pause A consistent scheme only tells you that you are approximating some PDE —
     but unless it is monotone, you might be approximating the wrong solution to the right PDE.

    }
    
\end{frame}


\begin{frame}{Barles-Souganidis framework}

    {\footnotesize \footnotesize
    \par A numerical scheme is an equation of the following form
    \[
    S(h,t,x,u_h(t,x),[u_h]_{t,x}) = 0 \quad \text{for } (t,x) \in G_h \setminus \{t=0\} \tag{3}
    \]

    \[
    u_h(0,x) = u_0(x) \quad \text{in } G_h \cap \{t=0\} \tag{4}
    \]
     \pause 
    \begin{itemize}
\item $h = (\Delta t, \Delta x)$: This is the discretization size. Time and space steps.
\item $G_h = \Delta t \cdot \{0, 1, \dots, n_T\} \times \Delta x \cdot \mathbb{Z}^N$:
  This defines a discrete space-time grid: time from 0 to $T$ in steps of $\Delta t$, and space as an $N$-dimensional grid in steps of $\Delta x$.
\item $u_h(t,x)$: This is the numerical approximation of the solution $u(t,x)$ on the grid.
\item $[u_h]_{t,x}$: This stands for values of $u_h$ at nearby grid points
 used to compute derivatives numerically (like neighbors in finite difference formulas).
\end{itemize}
\vspace{1em}
\par \pause  Note: The numerical solution $u_h$ is computed only at grid (mesh) points.
But we can treat it like a continuous function by interpolating between those values.
This idea connects the discrete world (numerical) with the continuous world (PDE).
    }
    
\end{frame}
\begin{frame}{Barles-Souganidis framework}

    {\footnotesize \footnotesize
    The theory requires the following assumptions
    \vspace{1em}
    \par \textbf{Monotonicity}
    \vspace{1em}
    \[
S(h,t,x,r,u) \geq S(h,t,x,r,v) \text{ if } u \leq v. 
\]

    \par The monotonicity assumption can be weakened somewhat. We only need it to hold approximately, with a margin of error that vanishes to 0 as $h$ goes to 0.
   \vspace{1em}
    \par  \pause \textbf{Consistency}
    \vspace{1em}
    \par For every smooth function $\Phi(t,x)$ with bounded derivatives,
    \begin{align*}
        \lim_{h \to 0, (n \Delta t, i \Delta x) \to (t,x), c \to 0} &S(h,n \Delta t,i \Delta x,\Phi(t,x) 
        + c,[\Phi + c]_{t,x})\\
        & = \Phi_t + F(t,x,\Phi(t,x),D\Phi(t,x),D^2\Phi(t,x)).
    \end{align*}
    \par  \pause \textbf{Stability}
    \vspace{1em}
    \par For every $h > 0$, the scheme has a solution $u_h$ which is uniformly bounded independently of $h$.

    }
    
\end{frame}
\begin{frame}{Barles-Souganidis framework}

    {\footnotesize \footnotesize
    \par Theorem (Barles-Souganidis[5])
    \vspace{1em}
    \par Under the above assumptions, if the scheme (3),(4) satisfy the consistency,
    monotonicity and stability properties, its solution $u_h$ converges locally
    uniformly to the unique viscosity solution of (1),(2).
    \vspace{1em}
    \par Next, we will go through some example tests to illustrate this theory.
    }
    
\end{frame}

\begin{frame}{The heat equation}

    {\footnotesize \footnotesize
    \par First, recall the classic explicit and implicit schemes for the linear heat equation in dimension 1 and verify that these schemes satisfy the required properties.

    \[
    u_t - u_{xx} = 0 \text{ in } (0,T] \times \mathbb{R}.  \tag{5}
    \]
    

    \[
    u(0,x) = u_0(x)  \tag{6}
    \]


    \par \pause  We assume here that \( u_0 \) is continuous and bounded in \(\mathbb{R}\). Of course, 
    the linear heat equation does not require the machinery of viscosity solutions but falls 
    into the scope of this theory and provides the opportunity to understand the connection between 
    the theory for linear parabolic equations and the theory of viscosity solutions. More precisely, 
    our goal here is to verify that the standard finite difference 
    approximations for the heat equation are convergent in the Barles-Souganidis sense.
    }
    
\end{frame}

\begin{frame}{The standard explicit scheme}

    {\footnotesize \footnotesize
     First of all, we define the mesh \(\{(n \Delta t, i \Delta x)| n \in \{0, \dots, N\}, i \in \mathbb{Z}\}\) where \((\Delta t, \Delta x)\) are the discretization steps and \(N\) is such that \(N \Delta t = T\).
\[
\frac{u_{i}^{n+1} - u_{i}^{n}}{\Delta t} = \frac{u_{i+1}^{n} + u_{i-1}^{n} - 2u_{i}^{n}}{\Delta x^{2}}.
\]
 \pause Since this scheme is explicit, it is very easy to compute at each time step \(n + 1\) the value of the approximation \(\{u_{i}^{n+1}| i \in \mathbb{Z}\}\) from the value of the approximation at the time step \(n\), namely \(\{u_{i}^{n}| i \in \mathbb{Z}\}\)
\[
u_{i}^{n+1} = u_{i}^{n} + \Delta t \left\{\frac{u_{i+1}^{n} + u_{i-1}^{n} - 2u_{i}^{n}}{\Delta x^{2}}\right\}.
\]
 \pause Furthermore, we initialize this algorithm by imposing the initial condition
\[
u_{i}^{0} = u_{0}(i \Delta x).
\]
 \pause Here, we may define formally the scheme \(S\) by setting:
\[
S(\Delta t, \Delta x, (n + 1) \Delta t, i \Delta x, u_{i}^{n+1}, [u_{i-1}^{n}, u_{i}^{n}, u_{i+1}^{n}] ) = \frac{u_{i}^{n+1} - u_{i}^{n}}{\Delta t} - \frac{u_{i+1}^{n} + u_{i-1}^{n} - 2u_{i}^{n}}{\Delta x^{2}}.
\]
\par Note that, in the last argument, only 3 grid points are used, instead of the whole grid.
    }
    
\end{frame}


\begin{frame}{The standard explicit scheme}

    {\footnotesize \footnotesize
    \par  This section is about verifying 
     that the explicit scheme is consistent, meaning it approximates the true PDE correctly as the grid is refined.
     We're checking that:
\[
\frac{u_{i}^{n+1} - u_{i}^{n}}{\Delta t} \approx u_{t}(t_{n}, x_{i}), \quad \frac{u_{i+1}^{n} + u_{i-1}^{n} - 2u_{i}^{n}}{\Delta x^{2}} \approx u_{xx}(t_{n}, x_{i})
\]
\par If both approximations match the true derivatives up to small error, then the scheme is consistent with:
\[
u_{t} = u_{xx}
\]
\par Spatial Derivative Approximation (Center Difference)
\par  \pause We expand $u_{i+1}^{n}$ and $u_{i-1}^{n}$ around $x_{i}$ using Taylor expansions:
  \[
u_{i+1}^n = u_i^n + u_x(n\Delta t,i\Delta x)\Delta x + \frac{1}{2}u_{xx}(n\Delta t,i\Delta x)\Delta x^2 + u_{xxx}(n\Delta t,i\Delta x)\frac{1}{6}\Delta x^3 + 
\]
\[
\frac{1}{24}u_{xxxx}(n\Delta t,i\Delta x)\Delta x^4 + \Delta x^4 \epsilon(\Delta x),
\]
and
\[
u_{i-1}^n = u_i^n - u_x(n\Delta t,i\Delta x)\Delta x + \frac{1}{2}u_{xx}(n\Delta t,i\Delta x)\Delta x^2 - \frac{1}{6}u_{xxx}(n\Delta t,i\Delta x)\Delta x^3 +
\]
\[
\frac{1}{24}u_{xxxx}(n\Delta t,i\Delta x)\Delta x^4 + \Delta x^4 \epsilon(\Delta x).
\]

    }
    
\end{frame}

\begin{frame}{The standard explicit scheme}

    {\footnotesize \footnotesize
     \par Where \(\lim_{\Delta x \to 0} \epsilon(\Delta x) = 0\). Then, 
     adding up the two expansions, subtracting \(2u_i^n\) from the 
     left and right hand sides and dividing by \(\Delta x^2\), one obtains

\[
\frac{u_{i+1}^n + u_{i-1}^n - 2u_i^n}{\Delta x^2} = u_{xx}(n\Delta t,i\Delta x) 
+ \frac{1}{12}u_{xxxx}(n\Delta t,i\Delta x)\Delta x^2 + \Delta x^2 \epsilon(\Delta x),
\]

 \pause and thus the truncation error for this approximation of the second spatial derivative is of order 2. Similarly the expansion

\[
u_{i}^{n+1} = u_{i}^{n} + u_{t}(n\Delta t,i\Delta x)\Delta t + 
\frac{1}{2}u_{tt}(n\Delta t,i\Delta x)\Delta t^{2} + \Delta t^{2}\epsilon(\Delta t),
\]

where \(\lim_{\Delta t \to 0} \epsilon(\Delta t) = 0\) yields

\[
\frac{u_{i}^{n+1} - u_{i}^{n}}{\Delta t} = u_{t}(n\Delta t,i\Delta x) + \frac{1}{2}u_{tt}(n\Delta t,i\Delta x)\Delta t + \Delta t\epsilon(\Delta t).
\]
\par  \pause The truncation error for the approximation of the first derivative 
in time is of order 1 only. So the scheme is formally consistent with:
\[
u_t = u_{xx},
\]
as $\Delta t \to 0$ and $\Delta x \to 0$.
    }
\end{frame}



\begin{frame}{The standard explicit scheme}

    {\footnotesize \footnotesize
     \par For the heat equation explicit scheme:
\[
u_{i}^{n+1} = u_{i}^{n} + \frac{\Delta t}{\Delta x^{2}} (u_{i+1}^{n} + u_{i-1}^{n} - 2u_{i}^{n}).
\]
\par Rewriting:
\[
u_{i}^{n+1} = u_{i}^{n} \left( 1 - \frac{2\Delta t}{\Delta x^{2}} \right) +
 \frac{\Delta t}{\Delta x^{2}} (u_{i+1}^{n} + u_{i-1}^{n}).
\]
\vspace{1em}
\par   \pause Furthermore, the approximation \(S\) is monotone if and only if \(S\) is 
decreasing in \(u_{i}^{n}, u_{i+1}^{n}\) and \(u_{i-1}^{n}\). First of all, it 
is unconditionally decreasing with
 respect to both $u_{i-1}^n$ and $u_{i+1}^n$. Secondly, it is only decreasing 
 in $u_i^n$ if the following CFL condition is satisfied:

\[
(-1 + 2\frac{\Delta t}{\Delta x^2}) \leq 0
\]

or equivalently

\[
\Delta t \leq \frac{1}{2}\Delta x^2.
\]

    }
\end{frame}

\begin{frame}{The standard explicit scheme}

    {\footnotesize \footnotesize
     Finally, we can prove the boundedness of the approximation independently 
     of the discretization steps and hence the stability 
     of the scheme recursively by showing the maximum principle
\[
\max_i |u_i^n| \leq \max_i |u_i^0|
\]
using the monotonicity property under the CFL condition.

 \pause Indeed, we have
\begin{align*}
    u_i^{n+1} &= u_i^n(1 - 2\frac{\Delta t}{\Delta x^2}) + \frac{\Delta t}{\Delta x^2}(u_{i+1}^n + u_{i-1}^n)\\
    &\leq \max_i |u_i^n|(1 - 2\frac{\Delta t}{\Delta x^2}) + 2\frac{\Delta t}{\Delta x^2} \max_i |u_i^n|\\
    &\leq \max_i |u_i^n|,
\end{align*}
\par to show reverse inequality we can reverse the time index in the logic and will end up bounding both directions.
\par  \pause This symmetry confirms:

\[
\max_i |u_i^{n+1}| = \max_i |u_i^n| = \max_i |u_i^0|.
\]
\par The maximum is non-increasing and bounded across all steps.
    }
\end{frame}

\begin{frame}{The standard implicit scheme}

    {\footnotesize \footnotesize
     \par For many financial applications, the explicit scheme turns out to be very
        inaccurate because the CFL condition forces the time step to be so small
        that the rounding error dominates the total computational error (computational
        error=rounding error+truncation error). Most of the time, an implicit
        scheme is preferred because it is unconditionally convergent, regardless of
        the size of the time step.
        \vspace{1em}
    \par  \pause We now evaluate the second derivative at time $(n+1)\Delta t$ instead of time $n\Delta t$,
\[
\frac{u_{i}^{n+1}-u_{i}^{n}}{\Delta t} = \frac{u_{i+1}^{n+1}+u_{i-1}^{n+1}-2u_{i}^{n+1}}{\Delta x^2}.
\]
\par We also need to initialize this scheme by setting
\[
u_{i}^{0}=u_{0}(i\Delta x).
\]
\par  \pause Implementing an algorithm allowing to compute the approximation is less obvious here. 
This discrete equation may be converted into a linear system of equations and the 
algorithm will then consist in inverting a tridiagonal matrix. The truncation errors for 
smooth functions 
are the same as for the explicit scheme and the consistency follows from this analysis.
    }
\end{frame}


\begin{frame}{The standard implicit scheme}

    {\footnotesize \footnotesize
    \par It turns out that, for any choice of the time step, the implicit scheme is monotone. In order to verify that claim, let us rewrite the implicit scheme using the notation $S$:

\[
S(\Delta t, \Delta x, (n+1)\Delta t, i\Delta x, u_i^{n+1}, [u_{i-1}^{n+1}, u_i^n, u_{i+1}^{n+1}]) =
\]

\[
\frac{u_i^{n+1}-u_i^n}{\Delta t} - \frac{u_{i+1}^{n+1}+u_{i-1}^{n+1}-2u_i^{n+1}}{\Delta x^2}.
\]

    \par Since $S$ is decreasing in $u_i^n, u_{i+1}^{n+1}$ and $u_{i-1}^{n+1}$ 
    the implicit scheme is unconditionally monotone.
    \vspace{1em}
    \par \pause  Besides, the implicit scheme is clearly consistent with the heat equation and is stable because, like the explicit scheme, it satisfies the maximum principle

\[
\max_i |u_i^n| \leq \max_i |u_i^0|.
\]
    }
\end{frame}

\begin{frame}{The standard implicit scheme}

    {\footnotesize \footnotesize
     One may also consider alternately the so-called Crank-Nicolson scheme

\[
\frac{u_{i}^{n+1} - u_{i}^{n}}{\Delta t} = \theta \frac{u_{i+1}^{n+1} + u_{i-1}^{n+1} - 2u_{i}^{n+1}}{\Delta x^2} + (1 - \theta) \frac{u_{i+1}^{n} + u_{i-1}^{n} - 2u_{i}^{n}}{\Delta x^2}.
\]
\vspace{1em}
\par \pause  Observe that the choice $\theta = 0$ leads to the explicit scheme whereas the $\theta = 1$ yields the fully implicit scheme. For $\theta = 0.5$, the truncation error is of second order both in time and space. The condition for monotonicity and stability is here

\[
\Delta t \leq \frac{\Delta x^2}{2(1 - \theta)}
\]

and in particular for $\theta = 0.5$ it becomes

\[
\Delta t \leq \Delta x^2.
\]
    }
\end{frame}


\begin{frame}{The Black-Scholes-Merton PDE}

    {\footnotesize \footnotesize
     The price of a European call $u(t, x)$ for a time to expiration equal to $t$ and for an asset value of $x$ dollars at time $T - t$, satisfies the degenerate linear PDE

\[
u_t + ru - \frac{1}{2} \sigma^2 x^2 u_{xx} - rx u_x = 0 \text{ in } (0, T] \times [0, +\infty)
\]

\[
u(0, x) = (x - K)^+
\]

\par where $T > 0$ is the maturity, $\mu$ is the drift of the underlying asset and $\sigma > 0$ is its volatility.
\vspace{1em}
\par  \pause Note that, in this example, the solution is no longer bounded but is nonnegative and grows at most linearly for large values of $x$. The Barles-Souganidis [5] framework can be slightly modified to accommodate the linear growth of the value function at infinity. One can also get rid of the degeneracy in the above Black-Scholes PDE by making the change of variable $y = \log s$.
\par The Black-Scholes-Merton PDE is linear and its Elliptic operator is degenerate. The first derivative $u_x$ can be easily approximated in a monotone way using a forward Finite Difference
\[
-rx u_x \approx -rx_i \frac{u_{i+1}^{n+1} - u_i^{n+1}}{\Delta x}.
\]
    }
\end{frame}

\begin{frame}{The Black-Scholes-Merton PDE}

    {\footnotesize \footnotesize
    \par The Black-Scholes-Merton PDE is linear and its Elliptic operator is degenerate. The first derivative $u_x$ can be easily approximated in a monotone way using a forward Finite Difference

\[
-rx u_x \approx -rx_i \frac{u_{i+1}^{n+1} - u_i^{n+1}}{\Delta x}.
\]

 \pause One can, for instance, implement the implicit scheme
\begin{align*}
    S(\Delta t, \Delta x, & (n+1)\Delta t, i\Delta x, u_i^{n+1}, [u_{i-1}^{n+1}, u_i^n, u_{i+1}^{n+1}]) \\
    =&\frac{u_i^{n+1} - u_i^n}{\Delta t} + r u_i^{n+1} - \frac{1}{2} \sigma^2 (i \Delta x)^2 \frac{u_{i+1}^{n+1} + u_{i-1}^{n+1} - 2u_i^{n+1}}{\Delta x^2}- r i \Delta x \frac{u_{i+1}^{n+1} - u_i^{n+1}}{\Delta x}.
\end{align*}
\par  \pause The consistency is straightforward, each finite difference approximates its derivative to leading order.
The monotonicity is unconditional and the standard stability 
requirement can be replaced by the following condition to accommodate 
the linear growth of the solution and its approximation at $+\infty$.

\par Stability For every $h$, the scheme has a solution $u_h$ which is nonnegative and grows at most linearly.
    }
\end{frame}

% \begin{frame}{Section 8}

%     {
%     \begin{center}
%         Numerical Methods for Nonlinear PDEs in Finance
%     \end{center}
%      \begin{center}
%         Peter A. Forsyth and Kenneth R. Vetzal
%     \end{center}
%     }
    
% \end{frame}

% \begin{frame}{Section 8}

%     {\footnotesize \footnotesize
%     \par Section 8 provides a concrete numerical example showing that the fully implicit 
%     monotone scheme converges to the correct viscosity solution, while the Crank–Nicolson scheme, though 
%     seemingly reasonable, converges to a totally wrong and arbitrage-violating solution even though it looks stable.
%     \vspace{1em}
%     \par The underlying asset price $S_t$ follows  
%   \[
%   dS_t = \mu S_t dt + \sigma_t S_t dW_t,
%   \]
%   but $\sigma_t$ is unknown, except that:  
%   \[
%   \sigma_t \in [\sigma_{\min}, \sigma_{\max}].
%   \]
%    Let the expiry time time of the claim be $T$, and let $\tau = T - t$, if the volatility $\sigma$ were fixed, Black-Scholes PDE is:  
% \[
% V_\tau = \frac{1}{2} \sigma^2 S^2 V_{SS} + SV_S - rV.
% \]

% Now replace $\sigma$ with a control variable, $Q$:  
% \[
% V_\tau = \frac{1}{2} Q^2 S^2 V_{SS} + SV_S - rV.
% \]
%     }
    
% \end{frame}

% \begin{frame}{Section 8}

%     {\footnotesize \footnotesize
%     For a short position the optimal control problem is given by

% \[
% V_\tau = \sup_{Q \in \hat{Q}} \left\{ \frac{Q^2 S^2}{2} V_{SS} + SV_S - rV \right\} = 0
% \]

% where $\hat{Q} = \{\sigma_{\min}, \sigma_{\max}\}$ and $r$ is the borrowing/lending rate.
%  Replacing the sup by an inf gives the corresponding pricing equation for a long position.
%     }
    
% \end{frame}
% \begin{frame}

%     {\footnotesize \footnotesize

%     }
    
% \end{frame}
% % {\mathbb{P}^*}
% \tilde{\mathbb{P}}
% {\footnotesize \footnotesize
% }
% \tiny
% \scriptsize
% \footnotesize
% \small
% \normalsize (default)
\end{document}