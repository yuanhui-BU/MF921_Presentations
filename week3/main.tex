\documentclass{beamer}

% \usepackage[utf8]{inputenc}
% \usepackage[T1]{fontenc}
\usepackage{lmodern}   % modern Latin Modern fonts
\usepackage{textcomp}  % provides \textquoteright
\usepackage{lmodern} % Latin Modern fonts with T1 shapes


\usepackage{graphicx}
\usepackage{ragged2e} % for generating dummy text
\usepackage[backend=biber,style=authoryear]{biblatex}
% \addbibresource{references.bib}

\usetheme{Madrid}
\usecolortheme{default}
\usefonttheme{professionalfonts} % keeps proper math fonts

\usepackage{amsmath,amssymb,amsfonts} % math symbols (\mathcal, \mathbb, etc.)
\usepackage{mathrsfs}                 % optional: \mathscr for fancy script

% \setbeamercovered{invisible} 
\setbeamercovered{transparent}


\title{MF921 Topics in Dynamic Asset Pricing}
\subtitle{Week 3}
\author{Yuanhui Zhao}
\date{Boston University}
%Today, I will present the paper Valuing Data as an Asset, written by Professor Laura Veldkamp from Columbia Business School. This is a survey paper which summarize the data valuation works that professor Laura Veldkamp and her team did in the past few years. 
\begin{document}
\frame{\titlepage}
% \begin{frame}
% \frametitle{Outline}
% \tableofcontents
% \end{frame}
\section{Part I}
\begin{frame}{Part I}
    \begin{center}
        Option Pricing Under a Double Exponential Jump Diffusion Model
    \end{center}
    \vspace{2em}
    \begin{center}
        S.G. Kou\\
        Hui Wang
    \end{center}
    \vspace{3em}
    \par This paper aims to show that a double exponential jump diffusion model can lead to an analytic approximation for finite-horizon
    American options and analytical solutions for popular path-dependent options (such as lookback, barrier, and perpetual American options). We will focus on lookback and barrier options here.
 \end{frame}


\section{Background}
\begin{frame}{Background}

    {\footnotesize \footnotesize
    \par Recall The Double Exponential Jump Diffusion Model:
    \begin{align*}
        \frac{dS(t)}{S(t^{-})} = \mu dt + \sigma dW(t) + d\left(\sum_{i=1}^{N(t)} (V_i - 1)\right)
    \end{align*}
    \par\begin{itemize}
    \item \( W(t) \): Brownian motion under the real-world measure.
    \item \( N(t) \): Poisson process with rate \(\lambda\).
    \item \( V_i \): multiplicative jump sizes, i.i.d. random variables.
    \item \( Y = \log(V) \), the jump sizes follow double exponential law:
    \end{itemize}   
    \begin{align*}
        f_Y(y) = p \eta_1 e^{-\eta_1 y} \mathbf{1}_{y \geq 0} + q \eta_2 e^{\eta_2 y} \mathbf{1}_{y < 0}
    \end{align*}
    \par with parameters:
    \begin{itemize}
        \item \( p, q \geq 0, p + q = 1 \): probabilities of upward/downward jumps.
        \item \(\eta_1 > 1\): rate for upward jumps.
        \item \(\eta_2 > 0\): rate for downward jumps.
    \end{itemize}
    }
    
\end{frame}

\begin{frame}{Background Con.}

    {\footnotesize \footnotesize
    \par For option pricing, we switch to a risk-neutral measure \( P^* \), so that the discounted price process is a martingale:
    \begin{align*}
        E^{P^*}[e^{-rt}S(t)] = S(0)
    \end{align*}
    \par Under \( P^* \), the dynamics adjust:
    \begin{align*}
        \frac{dS(t)}{S(t^-)} = (r - \lambda^*(t)\zeta^*)dt + \sigma dW^*(t) + d\left(\sum_{i=1}^{N^*(t)}(V_i^*-1)\right)
    \end{align*}
    \par where:
    \begin{itemize}
        \item \( W^*(t) \): Brownian motion under \( P^* \),
        \item \( N^*(t) \): Poisson process with intensity \( \lambda^* \),
        \item \( V^* = e^{Y^*} \): jump multiplier with new parameters \( (p^*, q^*, \eta_1^*, \eta_2^*) \),
        \item \( \zeta^* = E^{P^*}[V^*] - 1 = \frac{p^{*}\eta_{1}^{*}}{\eta_{1}^{*}-1} + \frac{q^{*}\eta_{2}^{*}}{\eta_{2}^{*}+1} - 1\) is mean percentage jump size.
    \end{itemize}
    \par The log-price process:  
    \begin{align*}
        X(t) = \log\left(\frac{S(t)}{S(0)}\right) = \left(r - \frac{1}{2}\sigma^2 - 
    \lambda^*\zeta^*\right)t + \sigma W^*(t) + \sum_{i=1}^{N^*(t)} Y_i^*,\;\;X(0)=0
    \end{align*}
    }
    
    
\end{frame}


\section{Intuition}
\begin{frame}{Intuition of the Pricing Formula}

    \begin{itemize}
    \item Without jumps, the model reduces to geometric Brownian motion. Pricing American, 
    barrier, and lookback options is straightforward. First passage times are tractable, and 
    closed-form formulas are well known (as what we show in last week).
    \vspace{1em}
    \item With jumps, however, analytical pricing becomes difficult 
    because the process can cross barriers by jumping over them (Overshoot Problem).
\end{itemize}
    
\end{frame}

\section{Intuition}
\begin{frame}{Intuition of the Pricing Formula Con}


     {\footnotesize \footnotesize
    \par Define the first passage time:
    \begin{align*}
        \tau_{b} := \inf\{t \geq 0 : X(t) \geq b\}, \quad b > 0
    \end{align*}
    \par In a jump diffusion, when the process crosses \( b \), it may overshoot: $X(\tau_{b}) - b > 0$.
    \vspace{1em}
    \par This overshoot creates complications to compute the distribution of the first passage
times analytically: 
    \begin{itemize}
        \item Need the distribution of overshoot \( X(\tau_{b}) - b \).
        \item Need the joint dependence between overshoot and \( \tau_{b} \).
        \item Need correlation between overshoot and the terminal state \( X(T) \)
    \end{itemize}
    \vspace{1em}
    \par \textbf{Note}: Double exponential distribution assumption has a memoryless property, this property 
    simplifies the overshoot distribution and allows tractable Laplace transforms of first passage times.
    }
    
\end{frame}

\section{Some Useful Formulas}
\begin{frame}{Some Useful Formulas}

    {\footnotesize \footnotesize
    \par The double exponential jump diffusion process is a special 
    case of L\'evy processes with two-sided jumps, whose characteristic exponent admits the (unique) representation: 
    \vspace{1em}
    \begin{align*}
    \phi(\theta) = E[e^{i\theta X_1}] = \exp\left\{i\gamma\theta - \frac{1}{2}A\theta^2 
    + \int_{-\infty}^{\infty}(e^{i\theta y} - 1 - i\theta y I_{\{|y|\leq 1\}})\Pi(dy)\right\}
    \end{align*}\vspace{1em}
    \par where the generating triplet $(\gamma, A, \Pi)$ is given by:
    \vspace{1em}
    \begin{itemize}
        \item $A = \sigma^2$
        \item $\Pi(dy) = \lambda \cdot f_Y(y)dy 
        = \lambda p \eta_1 e^{-\eta_1 y} I_{\{y\geq 0\}} dy + \lambda q \eta_2 e^{\eta_2 y} I_{\{y<0\}} dy$
        \item $\gamma = \mu + \lambda E[V I_{\{|V|\leq 1\}}] = \mu + \lambda p \left( \frac{1 - e^{-\eta_1}}{\eta_1} 
        - e^{-\eta_1} \right) - \lambda q \left( \frac{1 - e^{-\eta_2}}{\eta_2} - e^{-\eta_2} \right)$
    \end{itemize}
    
    
    }

    
\end{frame}

\begin{frame}{Some Useful Formulas Con}

    {\footnotesize \footnotesize
    \par Moment Generating Function of the log-price process, \( X(t) \):
    \begin{align*}
        \mathbb{E}^* \left[ e^{\theta X(t)} \right] = \exp\{G(\theta)t\}
    \end{align*}
    \par Where the function $G(\cdot)$ is defined as:
    \begin{align*}
        G(x) = x \left( r - \frac{1}{2} \sigma^2 - \lambda \zeta \right) + \frac{1}{2} x^2 \sigma^2 
        + \lambda \left( \frac{p \eta_1}{\eta_1 - x} + \frac{q \eta_2}{\eta_2 + x} - 1 \right)
    \end{align*}
    \par \textbf{Note}: Lemma 3.1 in Kou and Wang (2003) shows that the 
    equation \( G(x) = \alpha, \forall \alpha > 0 \), has exactly 
    four roots: \(\beta_{1,\alpha}\), \(\beta_{2,\alpha}\), \(-\beta_{3,\alpha}\), and \(-\beta_{4,\alpha}\), where: 
    \begin{align*}
        0 &< \beta_{1,\alpha} < \eta_1 < \beta_{2,\alpha} < \infty\\
        0 &< \beta_{3,\alpha} < \eta_2 < \beta_{4,\alpha} < \infty
    \end{align*}
    \par These roots determine the structure of Laplace transforms for first passage times.
    }
    
\end{frame}
\begin{frame}{Some Useful Formulas Con}

    {\footnotesize \footnotesize
    \par Infinitesimal Generator of the log-price process, \( X(t) \):
    \vspace{1em}
    \begin{align*}
        (\mathcal{L}V)(x) = \frac{1}{2}\sigma^2 V''(x) + \left(r - \frac{1}{2}\sigma^2 
        - \lambda\zeta\right)V'(x) + \lambda \int_{-\infty}^{\infty}\left(V(x+y) - V(x)\right)f_Y(y)\,dy
    \end{align*}
    \vspace{1em}
    \par The generator describes how expectations of functions of \( X(t) \) evolve in time:
    \vspace{1em}
    \begin{align*}
        \frac{d}{dt}\mathbb{E}[V(X_t)] = \mathbb{E}[(\mathcal{L}V)(X_t)]
    \end{align*}
    \vspace{1em}
    \par They provide the mathematical foundation to derive option pricing formulas.



    }
    
\end{frame}

\section{Pricing Path-Dependent Options}
\begin{frame}{Lookback Options}

    {\footnotesize \footnotesize
    \par Consider a lookback put option with an initial "prefixed maximum" \( M \geq S(0) \):
    \vspace{1em}
    \begin{align*}
        LP(T) &= \mathbb{E}^{\mathbb{P}^*} \left[ e^{-rT} \left( \max\{M, \max_{0 \leq t \leq T} S(t)\} - S(T) \right) \right]\\
        & =   \mathbb{E}^{\mathbb{P}^*}\left[ e^{-rT} \max\{M, \max_{0 \leq t \leq T} S(t)\} \right] - S(0)\\
    \end{align*}

    \par You need the joint distribution of $\max S(t)$ and $S(T)$, which is complicated for jump processes. 
    Laplace transforms convert a complicated path integral over time into a function of roots of $G(x)$
    which we can solve algebraically.
    }
    
\end{frame}

\begin{frame}{Lookback Options Con.}

    \par Theorem:
    {\footnotesize \footnotesize
    
    \vspace{1em}
    \par Using the notations \(\beta_{1,\alpha+r}\) 
    and \(\beta_{2,\alpha+r}\) as in early silde, the Laplace transform of the lookback put is given by:
    \vspace{1em}
    {\footnotesize \tiny
    \begin{align*}
        \hat{L}(T) = \int_0^\infty e^{-\alpha T} \mathrm{LP}(T)  dT = \frac{S(0)A_\alpha}{C_\alpha} \left( \frac{S(0)}{M} \right)^{\beta_{1,\alpha+r}-1} 
        + \frac{S(0)B_\alpha}{C_\alpha} \left( \frac{S(0)}{M} \right)^{\beta_{2,\alpha+r}-1}  
        + \frac{M}{\alpha+r} - \frac{S(0)}{\alpha}
    \end{align*}
    }
    \vspace{1em}
    \par For all \(\alpha > 0\); here:
    \vspace{1em}
    \begin{align*}
        A_\alpha &= \frac{(\eta_1 - \beta_{1,\alpha+r}) \beta_{2,\alpha+r}}{\beta_{1,\alpha+r} - 1} \\
        B_\alpha &= \frac{(\beta_{2,\alpha+r} - \eta_1) \beta_{1,\alpha+r}}{\beta_{2,\alpha+r} - 1}\\
        C_\alpha &= (\alpha + r) \eta_1 (\beta_{2,\alpha+r} - \beta_{1,\alpha+r})
    \end{align*}

    }
    
\end{frame}

\begin{frame}{Lookback Options Con.}


    {\footnotesize \footnotesize
    \par Laplace inversion: 
    \begin{align*}
        LP(T)  = \frac{1}{2\pi i} 
        \int_{c-i\infty}^{c+i\infty} e^{\alpha T} \hat{L}(\alpha)  d\alpha
    \end{align*}
    \par This is the Bromwich inversion integral, intractable in closed form.
     So we approximate it numerically. 
     \par The most widely used numerical methods for Laplace transform inversion is
     Gaver-Stehfest (GS) algorithm. Suppose you know the Laplace transform of a function \( f(t) \):
     \begin{align*}
        F(\alpha) = \int_{0}^{\infty} e^{-\alpha t} f(t)  dt
     \end{align*}
     \par The GS method approximates \( f(t) \) by evaluating \( F(\alpha) \) at carefully 
     chosen points along the real line.

    }
    
\end{frame}

\begin{frame}{Lookback Options Con.}


    {\footnotesize \footnotesize
    \par The formula is:
    \begin{align*}
        f(t) \approx \frac{\ln(2)}{t} \sum_{k=1}^{n} w_k F \left( \frac{k \ln(2)}{t} \right)
    \end{align*}

where:

\begin{itemize}
    \item \( n \) is an even integer (typically converges nicely even for n between 5 and 10).
    \item \( w_k \) are weights (depending only on \( n \) and \( k \)):
    \begin{align*}
        w_k = (-1)^{\frac{n}{2}+k} \sum_{j=\lceil k/2 \rceil}^{\min(k,n/2)} 
        \frac{j^{n/2}(2j)!}{(\frac{n}{2}-j)! j! (j-1)! (k-j)! (2j-k)!}
    \end{align*}
\end{itemize}
\par Intuition: The algorithm generates a sequence \( f_n(x) \) such that \( f_n(x) \to f(x),  n \to \infty \)

    }
    
\end{frame}

\begin{frame}{Barrier Options}


    {\footnotesize \footnotesize
    \par Consider the up-and-in call (UIC) option with the barrier level $H$ ($H>S(0)$):
    \vspace{1em}
    \begin{align*}
        UIC = E^{\mathbb{P}^*}[e^{-rT}(S(T) - K)^+I{\{ \max\limits_{0 \leq t \leq T} S(t) \geq H \}}]
    \end{align*}
    \vspace{1em}
    \par For any given probability \( P \), define:
    \vspace{1em}
    \begin{align*}
        \Psi(\mu, \sigma, \lambda, p, \eta_1, \eta_2; a, b, T):= P[Z(T) \geq a, \max_{0 \leq t \leq T} Z(t) \geq b]
    \end{align*}
    \vspace{1em}
    \par where under \( P, Z(t) \) is a double exponential jump diffusion 
    process with drift \( \mu \), volatility \( \sigma \), and jump rate \( \lambda \), i.e., \( Z(t) = \mu t + \sigma W(t) + \sum_{i=1}^{N(t)} Y_i \), 
    and \( Y \) has a double exponential distribution with 
    density \( f_Y(y) \sim p \cdot \eta_1 e^{-\eta_1 y} 1_{\{y \geq 0\}} + q \cdot \eta_2 e^{y \eta_2} 1_{\{y < 0\}} \).
    }
    
    
\end{frame}
\begin{frame}{Barrier Options Con.}

    \par Theorem:
    \vspace{1em}
    {\footnotesize \footnotesize
    
    \par The price of the UIC option is obtained as:
    \vspace{1em}
    \begin{align*}
        \text{UIC} = & S(0) \Psi \left( r + \frac{1}{2} \sigma^2 - \lambda \zeta, \sigma, \tilde{\lambda}, \tilde{p}, \tilde{\eta}_1, \tilde{\eta}_2; \right. 
        \left. \log \left( \frac{K}{S(0)} \right), \log \left( \frac{H}{S(0)} \right), T \right) \\
        & -Ke^{-rT} \cdot \Psi \left( r - \frac{1}{2} \sigma^2 - \lambda \zeta, \sigma, \lambda, p, \eta_1, \eta_2; \right. 
         \left. \log \left( \frac{K}{S(0)} \right), \log \left( \frac{H}{S(0)} \right), T \right)
    \end{align*}
    \vspace{1em}
    \par where \( \tilde{p} = (p/(1 + \zeta)) \cdot (\eta_1 / (\eta_1 - 1)), \tilde{\eta}_1 
    = \eta_1 - 1, \tilde{\eta}_2 = \eta_2 + 1, \tilde{\lambda} = \lambda(\zeta + 1) \), 
    with \(\zeta = E^{P^*}[V] - 1 = \frac{p\eta_{1}}{\eta_{1}-1} + \frac{q\eta_{2}}{\eta_{2}+1} - 1\). 
    The Laplace transforms of \( \Psi \) is computed explicitly in Kou and Wang (2003).
    }
    
\end{frame}
\begin{frame}{Numerical Results}

    {\footnotesize \footnotesize
    
    \par Setting: For the lookback put option the predetermined maximum is $M = 110$. 
    \par For the UIC option the barrier and the strike price are given by $H = 120$ and $K = 100$. 
    \par For others \( T = 1 ,  r = 5\% ,\sigma = 0.2,  p = 0.3,  \frac{1}{\eta_1} = 0.02,  \frac{1}{\eta_2}  = 0.04, \lambda = 3, S(0) = 100\).
    }
    \begin{figure}
    \centering
    \includegraphics[width=0.7\textwidth]{1}
    % \caption{This is a sample figure caption}
    % \label{fig:example}
    \end{figure}
    
\end{frame}

\section{Part II}
\begin{frame}{Part II}

    \begin{center}
        Pricing Path-Dependent Options with Jump Risk via Laplace Transforms
    \end{center}
    \vspace{2em}
    \begin{center}
        Steven Kou\\
        Giovanni Petrella\\
        Hui Wang
    \end{center}
    \vspace{3em}
    \par  Show the analytical solutions for two-dimensional Laplace transforms of barrier option prices,
     as well as an approximation based on Laplace transforms for the prices of finite-time horizon American options, under a double exponential jump diffusion model.
    
\end{frame}

% \begin{frame}

%     {\footnotesize \footnotesize

%     }
    
% \end{frame}
% % {\mathbb{P}^*}
% \tilde{\mathbb{P}}
% {\footnotesize \footnotesize
% }
% \tiny
% \scriptsize
% \footnotesize
% \small
% \normalsize (default)
\end{document}