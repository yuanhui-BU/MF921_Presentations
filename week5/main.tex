\documentclass{beamer}

% \usepackage[utf8]{inputenc}
% \usepackage[T1]{fontenc}
\usepackage{lmodern}   % modern Latin Modern fonts
\usepackage{textcomp}  % provides \textquoteright
\usepackage{lmodern} % Latin Modern fonts with T1 shapes


\usepackage{graphicx}
\usepackage{ragged2e} % for generating dummy text
\usepackage[backend=biber,style=authoryear]{biblatex}
% \addbibresource{references.bib}

\usetheme{Madrid}
\usecolortheme{default}
\usefonttheme{professionalfonts} % keeps proper math fonts

\usepackage{amsmath,amssymb,amsfonts} % math symbols (\mathcal, \mathbb, etc.)
\usepackage{mathrsfs}                 % optional: \mathscr for fancy script

% \setbeamercovered{invisible} 
\setbeamercovered{transparent}


\title{MF921 Topics in Dynamic Asset Pricing}
\subtitle{Week 5}
\author{Yuanhui Zhao}
\date{Boston University}

\begin{document}
\frame{\titlepage}
% \begin{frame}
% \frametitle{Outline}
% \tableofcontents
% \end{frame}

\section{Chapter 14}
\begin{frame}{Chapter 14}

    \begin{center}
        \par Black-Scholes (II): Dominance-Free Interval and Risk Neutral Pricing
    \end{center}
    
\end{frame}

\begin{frame}{A General Brownian Market Model}

    {\footnotesize \footnotesize
    \par Given a complete probablity space \( (\Omega, \mathcal{F}, \mathbb{P}) \).
    \( W(t) = (W_1(t), \ldots, W_d(t))^\top \), independent \( d \)-dimensional Brownian motion. The filtraion 
    $\mathcal{F}_t^W = \sigma(W(s) : 0 \leq s \leq t)$ which is complete and right-continuous.
    \vspace{1em} 
    \par A financial market $\mathcal{M}$ with 1 bond and $d$ stocks under a finite horizon $[0,T]$:
    \begin{gather*}
    dS_0(t) = r(t)S_0(t)  dt, \quad S_0(0) = 1 \\
    dS_i(t) = S_i(t)\left(b_i(t)\,dt + \sum_{j=1}^d \sigma_{ij}(t)\,dW_j(t)\right),\;\;\text{for $i$ $\in 1,2....,d$}
    \end{gather*}
     \pause 
    \begin{itemize}
        \item \( r(t) \): interest rate
        \item  \( b(t) = (b_1, \ldots, b_d) \): appreciation rates
        \item \( \sigma(t) = (\sigma_{ij}(t)) \): volatility matrix
        \item \( r(t) \), \(b(t) \) and \( \sigma(t)\) all progressively measurable
         with respect to $\{\mathcal{F}_t\}$ and bounded uniformly in $(t,\omega) \in [0,T] \times \Omega$.
    \end{itemize}
    }   
\end{frame}

\begin{frame}{Remarks on the Modell}

    {\footnotesize \footnotesize
    \begin{itemize}
        \item The model is very general: processes can be dependent, time-varying, and even non-Markovian
        \vspace{1em}
        \item The model does not cover stochastic volatility or jump models. 
        Therefore, dominance arguments do not apply for stochastic volatility or jump models.
        \vspace{1em}
        \item  \pause In general, the option price is not pinned to a single number. Instead, there exists an interval $[h_{low},h_{up}]$.
        If the option price is within the interval then there will be no dominance, outside the interval there will be dominance opportunity.
        \vspace{1em}
        \item Ideal market assumption: Infinite divisibility of assets, no transaction costs or taxes, no borrowing/short-selling constraints 
        and same interest rate for borrowing and lending
    \end{itemize}
    }   
\end{frame}

\begin{frame}{Introducing auxiliary processes}

    {\footnotesize \footnotesize
    \par Relative risk (Sharpe ratio): 
    \begin{align*}
            \theta(t) = \sigma^{-1}(t)\left(b(t) - r(t)\mathbf{1}\right),\;\; \mathbf{1} = (1,1,......,1)^T
    \end{align*}
    \par Exponential martingale (RN derivative): 
    \begin{align*}
        Z(t) = \exp\left(-\int_{0}^{t}\theta^{\top}(s)\,dW(s) - \frac{1}{2}\int_{0}^{t}\|\theta(s)\|^{2}ds\right)
    \end{align*}
    \par  \pause Discount factor:
    \begin{align*}
        \gamma(t) = \exp\left(-\int_{0}^{t}r(s)\,ds\right)
    \end{align*}
    \par Brownian motion with drift:
    \begin{align*}
         W_0(t) = W(t) + \int_{0}^{t}\theta(s)\,ds,\;\; 0\leq t\leq T
    \end{align*}
    \par  \pause $\sigma(t)$ invertible, inverses bounded. Ensures bounded $\theta(t)$ and $Z(t)$ is a true martingale.
    These tools set up the risk-neutral framework for pricing options.

    }   
\end{frame}
\begin{frame}{Investor Setup}

    {\footnotesize \footnotesize
    \par  Investor type: ``small investor'' $\rightarrow$ cannot affect market prices.
    \par Trading Strategy defined as:
    \begin{align*}
         \phi(t) = (\phi_1(t), \ldots, \phi_d(t)) 
    \end{align*}
    \par Where $\phi_0(t)$ is number of bonds held,
     $\phi_i(t)$ is number of shares of stock $i$ held. 
     If  $\phi_0 < 0$ or $\phi_i < 0$ interpret as short position(loan). All decisions are adapted to $\mathcal{F}_t$.
    \par  \pause Cumulative consumption process defined as:
    \begin{align*}
         C(t) = \int_0^t c(s)  ds, \quad c(s) \geq 0
    \end{align*}
    \par $C(t)$ is non-decreasing (consumption can only increase).  $C(0) = 0,  C(T) < \infty$ almost surely. 
    Models the total amount consumed by the investor up to time $t$
    \vspace{1em}
    \par  \pause To ensure consistency, we need to specify what trading strategies are allowed in the market. Two rules impose in the market, 
    self-financing condition and exclusion of doubling strategies.
    }   
\end{frame}

\begin{frame}{Self-Financing Condition}

    {\footnotesize \footnotesize
    \par Self-Financing Condition:
    \begin{align*}
        \sum_{i=0}^d \phi_i(t)S_i(t) = \sum_{i=0}^d \phi_i(0)S_i(0) + \sum_{i=0}^d \int_0^t \phi_i(u) dS_i(u) - C(t), \quad 0 \leq t \leq T
    \end{align*}
    \par Changes in wealth = trading gains – consumption, no outside cash flows.
    \vspace{1em}
    \par  \pause A portfolio process is defined as $\pi(\cdot) = (\pi_1(\cdot), \ldots, \pi_d(\cdot))$, where $\pi_i(t) = \phi_i(t) S_i(t)$ means that 
    total amount of money invested in the $i$th risky asset. The total wealth $X(t)$ is equal to:
    \begin{align*}
        X(t) = \sum_{i=0}^d \phi_i(t) S_i(t) = \sum_{i=1}^d \pi_i(t) + \phi_0(t) S_0(t)
    \end{align*}
    \par Both the wealth process $X(\cdot)$ and the portfolio $\pi(\cdot)$ can clearly take both positive and negative values.
    }   
\end{frame}

\begin{frame}{Self-Financing Condition}

    {\footnotesize \footnotesize
    \par For a given initial capital \( x \) and a portfolio process \( \pi(\cdot) \), the self-financing condition is translated to:
    \begin{align*}
        X(t) = X(0) + \sum_{i=0}^{d} \int_{0}^{t} \phi_i(u) dS_i(u) - C(t)
    \end{align*}
    \par In terms of the differential form, we have:
    \begin{align*}
        dX(t) &= \sum_{i=0}^{d} \phi_i(t) dS_i(t) - dC(t) \\
    \end{align*}
    \vspace{-2em}
    \par Express in terms of portfolio $\pi$ and plug in dynamics of assets:
    \begin{align*}
        dX(t) &= \frac{X(t) - \sum_{i=1}^{d} \pi_i(t)}{S_0(t)} dS_0(t) + \sum_{i=1}^{d} \frac{\pi_i(t)}{S_i(t)} dS_i(t) - dC(t) \\
        &= X(t) r(t) dt + \sum_{i=1}^{d} \pi_i(t) \left[ (b_i(t) - r(t)) dt + \sum_{j=1}^{d} \sigma_{ij}(t) dW_j(t) \right] - dC(t)
    \end{align*}
    }   
\end{frame} 

\begin{frame}{Self-Financing Condition}

    {\footnotesize \footnotesize
    \par By vector notation, Brownian motion with drift and Sharpe ratio, we can rewrite as:
    \begin{align*}
        dX(t) = X(t)r(t)dt + \pi^\top(t)\sigma(t)dW_0(t) - dC(t), \quad X(0) = x
    \end{align*}
    \par Since $d\gamma(t) \cdot dX(t) = 0$, by Itô formula:
    \begin{align*}
        d(\gamma(t)X(t)) &= \gamma(t)dX(t) - r(t)\gamma(t)X(t)dt \\
        &= \gamma(t)\pi^\top(t)\sigma(t)dW_0(t) - \gamma(t)dC(t)
    \end{align*}
    \par  \pause Therefore, we have the wealth equation:
    \begin{align*}
        \gamma(t)X(t) = x - \int_0^t \gamma(s)dC(s) + \int_0^t \gamma(s)\pi^\top(s)\sigma(s)dW_0(s)
    \end{align*}
    \par For a triple $(x, \pi, C)$, if there exists a unique strong solution $X(\cdot)$, it's called the wealth process.
    For the stochastic integral to be well-defined:
    \begin{align*}
        \int_{0}^{T} \|\pi(t)\|^{2}  dt < \infty
    \end{align*}
    }   
\end{frame} 

\begin{frame}{Risk-Neutral Probability via Girsanov Theorem}

    {\footnotesize \footnotesize
    \par Definition of Risk-Neutral Measure:
    \begin{align*}
         \mathbb{P}^{0}(A) := \mathbb{E}[Z(T)1_{A}], \quad A \in \mathcal{F}_{T}
    \end{align*}
    \par $Z(T)$ is the exponential martingale. By Girsanov's Theorem, $ W_{0}(t) = W(t) + \int_{0}^{t} \theta(s) ds$ 
    is a standard Brownian motion under $\mathbb{P}^{0}$.
    \par  \pause Thus rewirte the wealth process:
    \begin{align*}
         N_{0}(t) &= \gamma(t)X(t) + \int_{0}^{t} \gamma(s)dC(s) \\
    &= x + \int_{0}^{t} \gamma(s)\pi^{\top}(s)\sigma(s) dW_{0}(s)
    \end{align*}
    \par A continuous $\mathbb{P}^{0}$-local martingale.
    }   
\end{frame} 

\begin{frame}{Risk-Neutral Probability via Girsanov Theorem}

    {\footnotesize \footnotesize
    \par Stock dynamics under risk-neutral measure:
    \begin{align*}
        dS_i(t) = S_i(t) \left[ b_i(t)dt - \sum_{j=1}^d \sigma_{ij}(t)\theta_j(t)dt + \sum_{j=1}^d \sigma_{ij}(t)dW_0^{(j)}(t) \right]
    \end{align*}
    \par Since $b(t) - \sigma(t)\theta(t) = r(t)\mathbf{1}$:
    \begin{align*}
        dS_i(t) = S_i(t) \left[r(t)dt + \sum_{j=1}^d \sigma_{ij}(t)dW_0^{(j)}(t) \right], \quad i = 1, \ldots, d
    \end{align*}
    \par  \pause Since $d\gamma(t) \cdot dS_i(t) = 0$, apply Itô:
    \begin{align*}
        d(\gamma(t)S_i(t)) &= S_i(t)dr(t) + \gamma(t)dS_i(t) \\
        &= \gamma(t)S_i(t) \sum_{j=1}^d \sigma_{ij}(t)dW_0^{(j)}(t)
    \end{align*}
    \par  \pause  Hence, the discounted stock processes $\gamma(\cdot)S_i(\cdot)$ are local martingales.
     This also confirms our intuition that every asset $S_i(t)$ should have a growth rate $r(t)$ in the risk neutral world.
    }   
\end{frame} 

\begin{frame}{Doubling Strategies and Admissibility}

    {\footnotesize \footnotesize
    \par Doubling Strategy: Double investment after each loss, leads to arbitrarily large wealth at $T$. 
    Requires wealth process $X(t)$ unbounded from below. Need to exclude, because creates arbitrage opportunities and 
    violates no-arbitrage principle.
    \vspace{1em}
    \par  \pause A uniform boundedness condition is needed to prevent the doubling strategy. Wealth must satisfy:
    \begin{align*}
        X^{x,\pi,C}(t) \geq -\Lambda, \quad 0 \leq t \leq T
    \end{align*}
    \par  With $\mathbb{E}^0[\Lambda^p] < \infty$ for some $p > 1$.
    \vspace{1em}
    \par  \pause In summary, the admissibility on $(\pi,C)$ essentially requires the portfolio to
    be self-financing and not to be a doubling strategy.
    }
\end{frame} 

\begin{frame}{Some Properties}

    {\footnotesize \footnotesize
    \par (i) Supermartingale Property:
    \vspace{1em}
    \par If $(\pi,C)$ is admissible, $N_0(t)$ is bound below.  $N_0(t)$ is a $\mathbb{P}^0$-supermartingale. Consequently:
    \begin{align*}
          \mathbb{E}^0 \left[ \gamma(T)X(T) + \int_0^T \gamma(t)dC(t) \right] \leq x
    \end{align*}
    \par  Expected discounted terminal wealth and consumption less than or equal to initial wealth to ensures no arbitrage.
    \vspace{1em}
    \par  \pause (ii) Scaling Property:
    \vspace{1em}
    \par Wealth dynamics are linear in $(x, \pi, C)$. For any $a \neq 0$:
    \begin{align*}
         X^{ax, a\pi, aC}(t) = a \cdot X^{x, \pi, C}(t)
    \end{align*}
    \par In particular,  $a > 0$ outcome is scaled wealt and $a = -1$ is mirror strategy with wealth is $-X^{x, \pi, C}(t)$. 
    The intuition is that the model is homogeneous of degree 1 in wealth.
    }
\end{frame} 

\begin{frame}{Dominant Opportunities and Dominance-Free Interval}

    {\footnotesize \footnotesize
    \par Consider European Contingent Claim (ECC), payoff at maturity is $\psi(T) \geq 0$. For example the Call option payoff $(S_1(T) - K)^+$.
    To ensure that option price is finite, we assume that  $\mathbb{E}[(\psi(T))^{1+\epsilon}] < \infty, \;\forall \epsilon > 0$.
    \vspace{1em}
    \par  \pause Price at time 0 is $\psi(0)$. Question: what is the fair value of $\psi(0)$?
     Price too low $\rightarrow$ buyer arbitrage, price too high $\rightarrow$ seller arbitrage. 
    Therefore, the correct price must lie in an interval that rules out dominance.
    \vspace{1em}
    \par  \pause The main purpose of this section is to find out what $\psi(0)$ should be in the market $\mathcal{M}$ with the ECC,  
    denoted by $(M, \psi)$ for short, with $\psi$ standing for the pair $(\psi(0), \psi(T))$.
    \vspace{1em}
     \par  \pause A dominance opportunity exists in market $(\mathcal{M}, \psi)$ if  we
     start with some initial wealth $x \geq 0$ (or $x \leq 0$ depending on position). We take an admissible strategy $(\pi, C)$ and 
     add a position in the ECC (long if $a = -1$, short if $a = +1$).
    } 
\end{frame} 
\begin{frame}{A Definition of Dominate Opportunity}

    {\footnotesize \footnotesize
    \par Condition:
    \begin{align*}
         x + a \cdot \psi(0) < 0
    \end{align*}
    \par yet at maturity:
    \begin{align*}
        \mathbb{P}\{X^{x,\pi,C}(T) + a \cdot \psi(T) \geq 0\} = 1
    \end{align*}
    \par You start with negative initial wealth, but end up with a guaranteed nonnegative payoff. 
    This is a riskless profit strictly better than bond returns. 
    \vspace{1em}
    \par  \pause Moreover, 
    the scaling property will result the unlimited arbitrage.
     Therefore, dominance opportunities must be excluded in rational, well-behaved markets.
     \vspace{1em}
    \par Definition: The admissible price $\psi(0)$ must lie in an dominance-free interval $[a, b]$, such that:
    \begin{itemize}
        \item If $\psi(0) > b$: dominance opportunity exists (price too high)
        \item If $\psi(0) < a$: dominance opportunity exists (price too low)
        \item If $a \leq \psi(0) \leq b$: no dominance opportunities exist
    \end{itemize}
    }
\end{frame} 


\begin{frame}{Lower and Upper Hedging Classes}

    {\footnotesize \footnotesize
    \par Upper Hedging Class $\mathcal{U}$:
    \begin{align*}
         \mathcal{U} := \{x \geq 0 : \exists (\hat{\pi}, \hat{C}) \in \mathcal{A}, \, X^{x,\hat{\pi},\hat{C}}(0)
          = x, \, X^{x,\hat{\pi},\hat{C}}(T) \geq \psi(T) \ a.s.\}
    \end{align*}
    \par Starting with capital $x$,
     you can construct an admissible strategy whose terminal wealth is always at least as large as the claim payoff $\psi(T)$
     So, the minimum of $\mathcal{U}$ gives the upper bound for the fair price. $\mathcal{U}$ may be empty (think about quadratic payoff).
     \vspace{1em}
     \par  \pause Lower Hedging Class $\mathcal{L}$:
     \begin{align*}
         \mathcal{L} := \{x \geq 0 : \exists (\check{\pi}, \check{C}) \in \mathcal{A}, \, X^{x,\check{\pi},\check{C}}(0) 
         = -x, \, X^{x,\check{\pi},\check{C}}(T) \geq -\psi(T) \ a.s.\}
     \end{align*}
     \par  With initial wealth $-x$ (i.e. receiving $x$ up front),
      you can construct a strategy whose terminal wealth is always greater than or equal to $-\psi(T)$. 
      So, the maximum of $\mathcal{L}$ gives the lower bound for the fair price.
      \vspace{1em}
      \par  \pause Observe that both sets are intervals (connected):
     \begin{itemize}
        \item If $x_1 \in \mathcal{L}$ and $0 \leq y_1 \leq x_1$, then $y_1 \in \mathcal{L}$
        \item If $x_2 \in \mathcal{U}$ and $y_2 \geq x_2$, then $y_2 \in \mathcal{U}$
    \end{itemize}
    \par Thus it would be interesting to look at the endpoints of the intervals.
    }
\end{frame} 

\begin{frame}{Upper and Lower Hedging Prices}

    {\footnotesize \footnotesize
    \par \textbf{Upper bound: $h_{\text{up}} := \inf \mathcal{U}.$}  Because
     it's the cheapest initial capital needed to guarantee covering the claim $\psi(T)$. $\inf \mathcal{U}$ is 
     the minimal fair one for the seller.
      \vspace{1em}
     \par \textbf{Lower bound:} $h_{\text{low}} := \sup \mathcal{L}.$   Because it's the maximum initial 
     amount from which a buyer can still hedge against the debt of paying $\psi(T)$. 
     So $\sup \mathcal{L}$ is the highest ``safe'' price for the buyer.
     \vspace{1em}
     \par  \pause The intuition suggests that the lower dominate price cannot be bigger than the upper hedging price.
     \vspace{1em}
      \par Let $u_0 := \mathbb{E}^0[\gamma(T)\psi(T)]$. Show that $u_0 < \infty$. In fact, we can show a strong result that

        \[
        \mathbb{E}^0[\{\gamma(T)\psi(T)\}^a] < \infty, \quad 1 < a < 1 + \epsilon.
        \]
    }
\end{frame} 

\begin{frame}{Upper and Lower Hedging Prices}

    {\footnotesize \footnotesize
    \par Proof:
    \par Note that  $u_0 = \mathbb{E}^0[\gamma(T)\psi(T)] = \mathbb{E}[\gamma(T)\psi(T)Z(T)].$ Let $c < \infty$ be a fixed constant such that
    $\|\theta(t)\| \leq c, \;\gamma(T) \leq c.$ Also let $p = 1 + \epsilon$, $1/p + 1/q = 1$. We have:
     \pause 
    \begin{align*}
        & \mathbb{E} \left[ e^{ -q \int_0^T \theta^\top(s)dW(s) - \frac{1}{2} q \int_0^T \|\theta(s)\|^2 ds }  \right] \\
        &= \mathbb{E} \left[e^{ -q \int_0^T \theta^\top(s)dW(s) - \frac{1}{2} q^2 \int_0^T \|\theta(s)\|^2 ds } \cdot e ^{ \frac{1}{2}(q^2 - q) \int_0^T \|\theta(s)\|^2 ds } \right] \\
        &\leq \mathbb{E} \left[ e^ { -q \int_0^T \theta^\top(s)dW(s) - \frac{1}{2} q^2 \int_0^T \|\theta(s)\|^2 ds } \right] e^{ \frac{1}{2}(q^2 - q)c^2T } (\text{Typo: c}) \\
        &\leq e^{q(q-1)c^2T/2}
    \end{align*}
    \par where the last inequality holds because inside of $\mathbb{E}(\cdot)$ is a martingale. Therefore, by the Hölder inequality:
    }
\end{frame} 

\begin{frame}{Upper and Lower Hedging Prices}

    {\footnotesize \footnotesize
    \par \begin{align*}
        u_0 &\leq c\mathbb{E}(\psi(T)Z(T)) \\
        &\leq c(\mathbb{E}(\psi(T))^p)^{1/p} \cdot (\mathbb{E}(Z(T))^q)^{1/q} \\
        &= c(\mathbb{E}(\psi(T))^p)^{1/p} \cdot \left( \mathbb{E}\left[ e^{ -q \int_0^T \theta^\top(s)dW(s) -
         \frac{1}{2} q \int_0^T \|\theta(s)\|^2 ds } \right] \right)^{1/q} \\
        &\leq c(\mathbb{E}(\psi(T))^p)^{1/p} \cdot e^{(q-1)c^2 T/2} < \infty
        \end{align*}
    \par  \pause For $\mathbb{E}^0[\{\gamma(T)\psi(T)\}^a]$. Follow the same steps, change the measure and choose Hölder exponents, 
    $ p = \frac{1 + \varepsilon}{a} > 1, \quad q = \frac{p}{p - 1}$. Then:
    \begin{align*}
        \mathbb{E}[\psi^a Z] &\leq \left( \mathbb{E}[\psi^{ap}] \right)^{1/p} \left( \mathbb{E}[Z^q] \right)^{1/q}=
         \left( \mathbb{E}[\psi^{1 + \varepsilon}] \right)^{1/p} \cdot \left( \mathbb{E}[Z^q] \right)^{1/q}
    \end{align*}
    \par  The first factor is finite by assumption; the second is bounded exactly as in previous part. Thus:
    \begin{align*}
        \mathbb{E}^0[\{\gamma(T)\psi(T)\}^a] \leq c^a \left( \mathbb{E}[\psi(T)^{1 + \varepsilon}] \right)^{1/p} \exp \left\{ \frac{1}{2}(q - 1)c^2T \right\} < \infty
        ,\;\;\forall 1 < a < 1 + \varepsilon
    \end{align*}
    }
\end{frame} 


\begin{frame}{An Inequality for the Upper and Lower Prices}

    {\footnotesize \footnotesize
    \par At any time $t \in [0, T]$, we have $ 0 \leq h_{\text{low}} \leq u_0 \leq h_{\text{up}}\;\; a.s.,$ where $u_0 = \mathbb{E}^0[\gamma(T)\psi(T)]$.
    \vspace{1em}
    \par \textbf{For upper bound}: If $\mathcal{U}$ is empty, $h_{\text{up}} = +\infty$ and inequality is trivial. If $\mathcal{U} \neq \emptyset$, then by definition of $\mathcal{U}$, for every $x \in \mathcal{U}$,
     there exists some admissible portfolio $(\hat{\pi}, \hat{C})$ with:
     \begin{align*}
        X^{\hat{\pi}, \hat{C}}(0) = x, \quad X^{\hat{\pi}, \hat{C}}(T) \geq \psi(T)
     \end{align*}
     \par  \pause Apply the supermartingale property to the discounted wealth process:
     \begin{align*}
        x \geq \mathbb{E}^0 \left[ \gamma(T) X^{\hat{\pi}, \hat{C}}(T) + \int_0^T \gamma(s) d\hat{C}(s) \right]
     \end{align*}
     \par Since $X^{\hat{\pi}, \hat{C}}(T) \geq \psi(T)$, we get $ x \geq \mathbb{E}^0 [\gamma(T) \psi(T)]$. 
     This mean that any initial capital in $\mathcal{U}$ must be at least $u_0$, $ u_0 \leq h_{\text{up}}$.
    
    }
\end{frame} 

\begin{frame}{An Inequality for the Upper and Lower Prices}

    {\footnotesize \footnotesize
    \par \textbf{For lower bound}: Similarly, we can show that $0 \leq h_{\text{low}} \leq u_0$. Indeed, since the set $\mathcal{L}$ contains $x = 0$,
     it is nonempty. For any $x \geq 0$ in this set, again by the supermartingale property, almost surely.
      \pause 
     \begin{align*}
        -x &\geq \mathbb{E}^0 \left[ \gamma(T) X^{\check{\pi}, \check{C}}(T) + \int_0^T \gamma(s) d\hat{C}(s) \right] \\
        &\geq \mathbb{E}^0 \left[ \gamma(T) (-\psi(T)) + \int_0^T \exp \left( - \int_0^s r(u) du \right) d\hat{C}(s) \right] \\
        &\geq \mathbb{E}^0 \left[ \gamma(T) (-\psi(T)) \right]
        \end{align*}
        \vspace{1em}
    \par By the definition of $\mathcal{L}$. Hence, $x \leq \mathbb{E}^0 \left[ \gamma(T) \psi(T) \right]$ and $0 \leq h_{\text{low}} \leq u_0$, almost surely.
    

    }
\end{frame} 


\begin{frame}{Dominance Outside the Interval}

    {\footnotesize \footnotesize
    \par  Claim: for any ECC price $\psi(0) > h_{\text{up}}$, 
    there exists a dominant opportunity in $(M, \psi)$; similarly for any ECC price $\psi(0) < h_{\text{low}}$.
    \vspace{1em}
    \par Suppose that $\psi(0) > h_{\text{up}}$. Then for any $x_1 \in (h_{\text{up}}, \psi(0))$ we know that $x_1 \in \mathcal{U}$, since $h_{\text{up}}$ is 
    the left endpoint of a connected interval $\mathcal{U}$. By the definition of $\mathcal{U}$, there exists 
    a $(\hat{\pi}, \hat{C}) \in \mathcal{A}$ such that:
     \pause 
    \begin{align*}
        X^{\hat{\pi}, \hat{C}}(0) - \psi(0) = x_1 - \psi(0) < 0
    \end{align*}
    \par because $x_1 < \psi(0)$, and
    \begin{align*}
        X^{\hat{\pi}, \hat{C}}(T) - \psi(T) \geq \psi(T) - \psi(T) = 0
    \end{align*}
    \par Hence the definition of the dominance is satisfied with $a = -1$. 
    }
\end{frame} 

\begin{frame}{Dominance Outside the Interval}

    {\footnotesize \footnotesize
    \par  Suppose $\psi(0) < h_{\text{low}}$. Then for any $x_1 \in (\psi(0),h_{\text{low}})$ we 
    know that $x_1 \in \mathcal{L}$, since $h_{\text{low}}$ is 
    the right endpoint of a connected interval $\mathcal{L}$. By definition of $\sup \mathcal{L}$, 
     there exists an admissible strategy $(\check{\pi}, \check{C})$ such that: 
     \begin{align*}
         X^{\hat{\pi}, \hat{C}}(0) = -x_1, \quad X^{\hat{\pi}, \hat{C}}(T) \geq -\psi(T)
     \end{align*}
     \par  \pause Consider: 
     \begin{align*}
         X^{\hat{\pi}, \hat{C}}(0) + \psi(0) = -x_1 + \psi(0) < 0
     \end{align*}
     \par because $x_1 > \psi(0)$, and
     \begin{align*}
        X^{\hat{\pi}, \hat{C}}(T) + \psi(T) \geq -\psi(T) + \psi(T) = 0
    \end{align*}
    \par Hence the definition of the dominance is satisfied with $a = 1$. 
    % caron
    }
\end{frame} 

\begin{frame}{No dominance within the Interval}

    {\footnotesize \footnotesize
    \par Show that for any $\psi(0) \in [h_{\text{low}}, h_{\text{up}}]$ there is no dominant opportunity in $(\mathcal{M}, \psi)$.
    \par Proof:
    \par Suppose there is a dominant opportunity with $\psi(0) \in [h_{\text{low}}, h_{\text{up}}]$.
    \par  \pause \textbf{Case 1}: The dominant opportunity satisfies the definition with $a = -1$. In this case, there exist 
    an initial wealth $x \in [0, \infty)$ and a pair $(\pi_1, C_1) \in \mathcal{A}$, such that:
    \begin{align*}
        x - \psi(0) = X^{\pi_1, C_1}(0) - \psi(0) < 0
    \end{align*}
    \par where $x < \psi(0)$, and
    \begin{align*}
        X^{\pi_1, C_1}(T) - \psi(T) \geq 0, \text{ a.s.}
    \end{align*}
    \par From the definition of $\mathcal{U}$ we know that $x \in \mathcal{U}$, 
    where $x \geq h_{\text{up}}$, by the definition of $h_{\text{up}}$. 
    Therefore, $h_{\text{up}} \leq x < \psi(0)$; a contradiction, since by assumption $h_{\text{up}} \geq \psi(0)$.
    \vspace{1em}
     \par  \pause \textbf{Case 2}: The dominant opportunity satisfies the definition with $a = 1$. In this case, there exist 
    an initial wealth $x \in (\infty,0]$ and a pair $(\pi_2, C_2) \in \mathcal{A}$, such that:

    \begin{align*}
        x + \psi(0) = X^{\pi_2, C_2}(0) + \psi(0) < 0
    \end{align*}
    }

\end{frame} 

\begin{frame}{No dominance within the Interval}

    {\footnotesize \footnotesize
    \par where $-x > \psi(0)$, and
    \begin{align*}
        X^{\pi_2, C_2}(T) + \psi(T) \geq 0, \text{ a.s.}
    \end{align*}
    \par \pause  Define $y := -x$, then we have $ X^{\pi_2, C_2}(0) = -y, \; X^{\pi_2, C_2}(T) + \psi(T) \geq 0, \text{ a.s.}$. 
    From the definition of $\mathcal{L}$ we know that $y \in \mathcal{L}$, 
    where $y \leq h_{\text{low}}$, by the definition of $h_{\text{low}}$. Therefore, $\psi(T) < y \leq h_{\text{low}}$
    ; a contradiction, since by assumption $h_{\text{low}} \leq \psi(0)$.
    \vspace{1em}
    \par  \pause The interval $[h_{\text{low}}, h_{\text{up}}]$ is the  range of possible fair prices  for ECC. 
    Inside this interval, no dominance opportunities exist. Therefore, this interval is called the dominance-free interval.
    \vspace{1em}
    \par \pause  \textbf{Remark}: $\text{No arbitrage} \Rightarrow \text{No dominance} \Rightarrow \text{Law of one price}$.
    But the reverse is not always true. In particular, 
    there are examples showing that at the two endpoints $h_{\text{low}}$ and $h_{\text{up}}$ 
    there may be arbitrage opportunities but no dominant opportunities.

    }

\end{frame} 


\begin{frame}{The Unique Dominance-Free Price in the Ideal Market}

    {\footnotesize \footnotesize
    \par In an ideal (complete, unconstrained) market where every contingent claim can be perfectly replicated
    and unlimited long/short positions are allowed. The interval of possible fair prices, $ [h_{\text{low}}, h_{\text{up}}]$, 
    collapses to a single price:
    \begin{gather*}
        h_{\text{low}} = h_{\text{up}} = u_0\\
          \psi(0) = u_0 = \mathbb{E}^0[\gamma(T)\psi(T)]
    \end{gather*}
    \par  \pause Furthermore, we can show that, corresponding to the Black-Scholes price $u_0$, there is a ``hedging portfolio'' process $\pi(\cdot)$ (hence also a corresponding trading process $\phi(\cdot)$) and a consumption process $C(\cdot) \equiv 0$, such that

    \[
    X^{u_0,\pi,0}(T) = \psi(T) \tag{*}
    \]

    \par  \pause and with the opposite portfolio $-\pi(\cdot)$ (hence the opposite trading strategy $-\phi(\cdot)$), we have

    \[
    X^{-u_0,-\pi,0}(T) = -\psi(T) \tag{**}
    \]

    \par  \pause Hence the price for the ECC has to be $u_0$, if no dominance is allowed in $\mathcal{M}$. This price is called the \emph{dominance-free price}, also known as \emph{Black-Scholes price}.


    }

\end{frame} 

\begin{frame}{Martingale Representation Theorem}

    {\footnotesize \footnotesize
    \par To prove $(*)$ and $(**)$, we need the martingale representation theorem.
    \vspace{1em}
    \par  \pause Suppose $W_t$ is a $d$-dimensional Brownian motion. Let $\mathcal{F}_t$ be the history (filtration)
     generated by $W_t$. If $M$ is a $d$-dimensional continuous martingale with $M_0 = 0$, then there exists
      a vector of (progressively measurable) processes:
      \begin{align*}
        \theta_s = (\theta^{(1)}_s, \ldots, \theta^{(d)}_s)
      \end{align*}
      \par  \pause such that with probability one:
      \begin{align*}
        M_t = \sum_{j=1}^d \int_0^t \theta^{(j)}_s dW^{(j)}_s
      \end{align*}
      \par  \pause Any square-integrable martingale can be written as a stochastic integral of Brownian motion. In general, the martingale representation
theorem is only an existence results.
    }

\end{frame} 

\begin{frame}{Martingale Representation Theorem}

    {\footnotesize \footnotesize
    \par Consider a (Doob) martingale $X_t = \mathbb{E}(W_T^3 | \mathcal{F}_t), \, T \geq t$.
    \begin{itemize}
    \item[(i)] Compute the conditional expectation to show that  
    \[
    X_t = 3W_t \cdot (T - t) + W_t^3.
    \]
    
    \item[(ii)] Compute the stochastic integral to show that  
    \[
    X_t = \int_0^t 3(T - s + W_s^2) dW_s.
    \]
    \end{itemize}

    \par Therefore, in the martingale representation theorem $\theta_s = 3(T - s + W_s^2)$.
    \par  \pause Proof:
    \par (i)
    {\footnotesize \tiny
    \begin{align*}
         \mathbb{E}(W_T^3 | \mathcal{F}_t) & = \mathbb{E}((W_t + (W_T-W_t))^3 | \mathcal{F}_t)\\
         & = \mathbb{E}(W_t ^3 | \mathcal{F}_t) + \mathbb{E}(3W_t^2(W_T-W_t) | \mathcal{F}_t)
         + \mathbb{E}(3W_t(W_T-W_t)^2 | \mathcal{F}_t) + \mathbb{E}((W_T-W_t) ^3 | \mathcal{F}_t)\\
         &= W_t^3 + 3W_t (T - t)
    \end{align*}
    }
    \par Using that $W_t$ is $\mathcal{F}_t$-measurable and the independent increment $W_T - W_t \perp \mathcal{F}_t$ 
    with odd moments vanish.
    }

\end{frame} 

\begin{frame}{Martingale Representation Theorem}

    {\footnotesize \footnotesize
    \par (ii)
    \par \par Define $f(t, x) := 3x(T - t) + x^3$. Then $X_t = f(t, W_t)$.
    \par Apply Itô's formula to $f(t, W_t)$:
    \begin{align*}
        dX_t &= \left( f_t + \frac{1}{2} f_{xx} \right) dt + f_x dW_t \\
    &= \left( -3W_t + 3W_t \right) dt + \left( 3(T - t) + 3W_t^2 \right) dW_t \\
    &= 3 \left( T - t + W_t^2 \right) dW_t
    \end{align*}
    \par \pause  Integrate from $0$ to $t$ and $X_0 = \mathbb{E}[W_T^3] = 0$, thus:
    \begin{align*}
        X_t = \int_0^t 3 \left( T - s + W_s^2 \right) dW_s
    \end{align*}
    \par  \pause If you can make the discounted payoff a martingale under some measure, then by the MRT 
    there exists a process $\theta_s$ so that the trading strategy replicates the payoff exactly with probability 1.
    MRT requires continuous martingale. If the market has jumps, MRT generally fails, 
    so you can't in general replicate an option payoff exactly.
    }

\end{frame} 


\begin{frame}{Generalized Martingale Representation Theorem}

    {\footnotesize \footnotesize
    \par In this setting, every square-integrable martingale $M_t$ has the form:
    \begin{align*}
        M_t = M_0 + \int_0^t \theta_s dW_s + \int_0^t \int_{\mathbb{R}} \phi_s(x) \tilde{N}(ds, dx)
    \end{align*}
    \par  where:
    \begin{itemize}
        \item $\theta_s$ is the Brownian-driven part
        \item $\tilde{N}(ds, dx) = N(ds, dx) - \nu(dx) ds$ is the compensated Poisson measure
        \item $\phi_s(x)$ tells you how sensitive $M_t$ is to jumps of size $x$
    \end{itemize}
    \par  \pause In a jump-diffusion market, you generally cannot replicate payoffs perfectly using only continuous trading. 
    You would need extra instruments tied to the jump risk (like insurance contracts, catastrophe bonds, or options directly on jumps).
    }

\end{frame} 


\begin{frame}{Proofs of $(*)$ and $(**)$}

    {\footnotesize \footnotesize
    \par Consider the discounted final payoff of the option: $ Q(T) := \gamma(T)\psi(T)$. Note that from 
    the definition of $u_0$, we have $\mathbb{E}^0[Q(T)] = u_0$. Define for $s \in [0, T]$:
    \begin{align*}
        \zeta(s) := \frac{1}{\gamma(s)}\mathbb{E}^0[Q(T) | \mathcal{F}_s] \geq 0
    \end{align*}
    \par  \pause Then almost surely at time 0 and at time $T$ we have:
    \begin{align*}
        \zeta(0) = u_0, \quad \zeta(T) = \frac{1}{\gamma(T)}\mathbb{E}^0[\gamma(T)\psi(T) | \mathcal{F}_T] = \psi(T)
    \end{align*}
    \par It is tempting to apply the martingale representation theorem to $\mathbb{E}^0[Q(T) | \mathcal{F}_s]$, 
    which is martingale adapted to $\{W_0(t)\}$, to find a replication strategy for the option payoff $\psi(T)$.
     However, what we need is a replication strategy adapted to $\{W(t)\}$ not $\{W_0(t)\}$.
    }

\end{frame}

\begin{frame}{Proofs of $(*)$ and $(**)$}

    {\footnotesize \footnotesize
   
    \par \textbf{Theorem}:
     \vspace{1em}
    \par There exists a (progressively measurable) process \(\eta(\cdot)\), 
    \(\int_0^T \|\eta(u)\|^2 du < \infty\), adapted to \(\{W(t)\}\) such that, almost surely,
    \begin{align*}
        \gamma(s) \zeta(s) = E^0[Q(T) | \mathcal{F}_s] = u_0 + \int_0^s \eta^*(u) dW_0(u), \quad s \in [0,T]\; (\text{Typo?}W(u))
    \end{align*}
     \vspace{1em}
    \par  \pause Note that the standard martingale representation theorem does not
     apply here, as we want \(\eta(\cdot)\) to be adapted to \(\mathcal{F}_t\), 
     the P-argumented filtration generated by \(\{W(t)\}\) not by \(\{W_0(t)\}\). 
     The filtration generated by \(\{W_0(t)\}\) is different from \(\mathcal{F}_t\) due to the null sets of \(\{W_0(t)\}\).
     \vspace{1em}
    \par  \pause \textbf{Proof}: Let $M^0(t) = \gamma(t) \zeta(t) = E^0[Q(T) | \mathcal{F}_t]$, where \(\mathcal{F}_t\) is 
    the filtration generated by the process \(\{W(t)\}\). 
    Then \(M^0(t)\) is a martingale with respect to \(\mathcal{F}_t\) under the new probability \(P^0\), because for \(t \geq s\)
    \begin{align*}
        E^0[M^0(t) | \mathcal{F}_s] = E^0[E^0[Q(T) | \mathcal{F}_t] | \mathcal{F}_s] = E^0[Q(T) | \mathcal{F}_s] = M^0(s)
    \end{align*}
    }

\end{frame}


\begin{frame}{Proofs of $(*)$ and $(**)$}

    {\footnotesize \footnotesize
   
    \par Define $M(t) = Z(t) M^0(t)$. Clearly \(M(t)\) is adapted to \(\{W(t)\}\), because \(Z(t)\) and \(M^0(t)\) are 
    all adapted to \(\{W(t)\}\). We shall show that \(M(t)\) is a martingale under the original probability. To do this, recall Bayes formula (which is given previously when we discussed Girsanov theorem) 
    says that for any \( Y \in \mathcal{F}_t \), the new conditional expectation under \( E^0 \) is given by  :
    \begin{align*}
        E^0 [Y | \mathcal{F}_s] = \frac{1}{Z(s)} E[Y \cdot Z(t) | \mathcal{F}_s]
    \end{align*}
    \par  \pause Since \( M^0(t) \in \mathcal{F}_t \), we have by Bayes formula for \( t \geq s \)  
    \begin{align*}
        E[M(t) | \mathcal{F}_s] = E[Z(t)M^0(t) | \mathcal{F}_s] = Z(s)E^0[M^0(t) | \mathcal{F}_s] = Z(s)M^0(s) = M(s)
    \end{align*}
    \par from which the conclusion follows.  
    \par Next, by the standard martingale representation theorem, we can find a vector process \( \xi(t) \), adapted to \( W(t) \), such that :
    \begin{align*}
        dM(t) = \xi^*(t) dW(t)
    \end{align*}
   
    }

\end{frame}

\begin{frame}{Proofs of $(*)$ and $(**)$}

    {\footnotesize \footnotesize
   
    \par Now since \( M^0(t) = M(t) \cdot 1/Z(t) \), we have:
    \begin{align*}
        d(M^0(t)) = \frac{1}{Z(t)} dM(t) + M(t) d\left(\frac{1}{Z(t)}\right) + dM(t) d\left(\frac{1}{Z(t)}\right)
    \end{align*}
    \par By Itô's formula:
    \begin{align*}
        d \left( \frac{1}{Z(t)} \right) =  \frac{1}{Z(t)} \left[ \theta^*(t) dW(t) + \| \theta(t) \|^2 dt \right]
    \end{align*}
     \par  \pause Therefore:
    \begin{align*}
        d(M^0(t)) = \frac{1}{Z(t)} \left\{ \xi^*(t) dW(t) + M(t) \theta^*(t) dW(t) + M(t) \| \theta(t) \|^2 dt + \xi^*(t) \theta(t) dt \right\}
    \end{align*}
    \par Since
    \begin{align*}
        M(t) \theta^*(t) dW(t) + M(t) \| \theta(t) \|^2 dt = M(t) \theta^*(t) \cdot [dW(t) + \theta(t) dt] = M(t) \theta^*(t) dW_0(t)
    \end{align*}
    }

\end{frame}

\begin{frame}{Proofs of $(*)$ and $(**)$}

    {\footnotesize \footnotesize
   \begin{align*}
    \xi^*(t)dW(t) + \xi^*(t)\theta(t)dt = \xi^*(t)[dW(t) + \theta(t)dt] = \xi^*(t)dW_0(t)
   \end{align*}
   \par we further have:
   \begin{align*}
    d(M^0(t)) = \frac{1}{Z(t)} \left\{ \xi^*(t) + M(t)\theta^*(t) \right\} dW_0(t)
   \end{align*}
   \par  \pause Thus, the martingale representation for \(M^0(t)\) holds with respect to \(W_0(t)\)
   \begin{align*}
    d(M^0(t)) = \eta^*(t)dW_0(t)
   \end{align*}
   \par With 
   \begin{align*}
    \eta^*(t) = \frac{1}{Z(t)} \left\{ \xi^*(t) + M(t)\theta^*(t) \right\}
   \end{align*}
   \par which is adapted to \(\{W_t\}\).
    }

\end{frame}

\begin{frame}{The Unique price}

    {\footnotesize \footnotesize
    \par To prove \( h_{\text{low}} = h_{\text{up}} = u_0 \), we only need to show that \( u_0 \in U \) and \( u_0 \in L \).
    \vspace{1em}
    \par First we shall show that \( u_0 \in U \). By theorem, Define a process \(\hat{\pi}\) by $\hat{\pi}^*(u) := \frac{1}{\gamma(u)} \eta^*(u)(\sigma(u))^{-1}$. 
    Then $\eta^*(u) = \gamma(u) \hat{\pi}^*(u) \sigma(u)$, and from theorem $\gamma(s) \zeta(s) = u_0 + \int_0^s \gamma(u) \hat{\pi}^*(u) \sigma(u) dW_0(u)$
    \vspace{1em}
    \par  \pause Recall that the wealth equation is given by:
    \begin{align*}
        \gamma(t) X(t) = x - \int_0^t \gamma(s) dC(s) + \int_0^t \gamma(s) \pi^*(s) \sigma(s) dW_0(s)
    \end{align*}
    \par Therefore, \(\zeta(s)\) satisfies 
    the wealth equation with the initial wealth \( u_0 \), the 
    consumption \(\hat{C}(s) \equiv 0\), and the portfolio strategy \(\hat{\pi}\):
    \begin{align*}
        X^{u_0, \hat{\pi}, \hat{C}}(s) = \zeta(s), \quad \int_0^T ||\hat{\pi}(t)||^2 dt < \infty
    \end{align*}
    \par  \pause Since almost surely \(\zeta(s) \geq 0\), we have \(\left( \hat{\pi}, \hat{C} \right) \in A(u_0)\), 
    hanks to the assumption of the ideal market that there is no constraints on \(\hat{\pi}\). Therefore, we get \( u_0 \in U \).
    }

\end{frame}

\begin{frame}{The Unique price}

    {\footnotesize \footnotesize
    \par To show \( u_0 \in L \), notice that with $\tilde{\pi} = -\hat{\pi}$. we have $X^{\tilde{\pi},0}(\cdot) = -X^{\hat{\pi},0}(\cdot)$
    To check the admissibility, it only remains to prove that \( X^{\tilde{\pi},0}(\cdot) \) is bounded 
    away from below, thanks again to the ideal market assumption that there is no constraints on \(\pi\). 
    \par  \pause To achieve this, note that:
    \begin{align*}
        \inf_{0 \leq s \leq T} X^{\tilde{\pi},0}(s) &= -\sup_{0 \leq s \leq T} (X^{\hat{\pi},0}(s)) = -\sup_{0 \leq s \leq T} (\zeta(s)) \\
    &= -\sup_{0 \leq s \leq T} \left\{ \frac{1}{\gamma(s)} E^0[Q(T) | \mathcal{F}_s] \right\} \\
    &\geq -\text{const.} \sup_{0 \leq s \leq T} E^0[Q(T) | \mathcal{F}_s]
    \end{align*}
    \par  \pause where the inequality follows via the boundedness of \(\gamma(t)\). Let $M(t) = E^0[Q(T) | \mathcal{F}_t].$ Then \( M(t) \) is
     a martingale with respect to \( P^0 \). By the Doob maximal inequality for martingale, we get, for some \( p > 1 \).
    }

\end{frame}
\begin{frame}{The Unique price}

    {\footnotesize \footnotesize
    \begin{align*}
E^0 \left[ \left( \sup_{0 \leq s \leq T} E^0[Q(T) | \mathcal{F}_s] \right)^p \right] &=
E^0 \left[ \left( \sup_{0 \leq s \leq T} M(s) \right)^p \right] \\
&\leq \left( \frac{p}{p-1} \right)^p E^0[(M(T))^p] \\
&= \left( \frac{p}{p-1} \right)^p E^0[(Q(T))^p] < \infty
\end{align*}

\par  \pause where the last inequality has already been proven in Problem 1. Thus, \(\inf_{0 \leq s \leq T} X^{\tilde{\pi},0}(s)\) is 
bounded below by a random variable \(-\Lambda\), such that \( E^0(\Lambda^p) < \infty \), for some \( p > 1 \).
    }

\end{frame}


\begin{frame}{BS Formula in Markets with Constant Coefficients}

    {\footnotesize \footnotesize
    \par  Market with constant drift \( b \), interest rate \( r \), and volatility $\sigma$. One stock \( S(t) \) (\( d = 1 \)). 
     To price a European call option with payoff $\psi(T) = (S(T) - K)^+. $ The option price is:
     \begin{align*}
        u_0 = E^0 \left( e^{-rT} (S(T) - K)^+\right)
     \end{align*}
     \par where \( E^0 \) is expectation under the risk-neutral measure \( P^0 \). Under \( P^0 \):
     \begin{align*}
        dS(t) = S(t)[rdt + \sigma dW_0(t)]
     \end{align*}
     \par  \pause Therefore, \( S(T) \) is equal, in distribution,
      to \( S(0) \exp \left\{ \left( r - \frac{\sigma^2}{2} \right) T + Z\sigma\sqrt{T} \right\} \), 
     where \( Z \) is a standard normal random variable.
     \begin{align*}
        u_0 = e^{-rT} \int_{-\infty}^{\infty} \left( S(0) \exp \left\{ \left( r - \frac{\sigma^2}{2} \right) T 
        + z\sigma\sqrt{T} \right\} - K \right)^+ \varphi(z) dz
     \end{align*}
     \par where \( \varphi(z) \) is the standard normal density.
    }
    
\end{frame}

\begin{frame}{BS Formula in Markets with Constant Coefficients}

    {\footnotesize \footnotesize
    \par We can evaluate the integral to show that
    \begin{align*}
        u_0 = E^0 \left( e^{-rT} (S(T) - K)^+ \right) = S(0) \cdot \Phi(\mu_+) - Ke^{-rT} \Phi(\mu_-)
    \end{align*}
    \par which is the celebrated Black-Scholes formula for European call option. Here
    \begin{align*}
        \mu_{\pm} := \frac{1}{\sigma\sqrt{T}} [\log(S(0)/K) + (r \pm (\sigma^2/2))T]
    \end{align*}
    \par and we set \( \Phi(z) = \frac{1}{\sqrt{2\pi}} \int_{-\infty}^{z} e^{-u^2/2} du \) .
    \par  \pause Indeed, since \( S(0)e^{(r-\sigma^2/2)T+\sigma\sqrt{T}z} \geq K \) is equivalent to \( z \geq -\mu_- \), we get
    \begin{align*}
    u_0 &= e^{-rT} \int_{-\mu_-}^{\infty} \left( S(0)e^{(r-\sigma^2/2)T+\sigma\sqrt{T}z} - K \right) \varphi(z) dz \\
    &= S(0) \frac{1}{\sqrt{2\pi}} \int_{-\mu_-}^{\infty} e^{-\sigma^2T/2+\sigma\sqrt{T}z} e^{-z^2/2} dz - Ke^{-rT} \Phi(\mu_-) \\
    &= S(0) \frac{1}{\sqrt{2\pi}} \int_{-\mu_-}^{\infty} \exp \left( - \left( z - \sigma\sqrt{T} \right)^2 / 2 \right) dz - Ke^{-rT} \Phi(\mu_-)
\end{align*}
    }
    
\end{frame}



\begin{frame}{BS Formula in Markets with Constant Coefficients}

    {\footnotesize \footnotesize
    \par Letting \( x = z - \sigma\sqrt{T} \) yields

    \begin{align*}
    u_0 &= S(0) \frac{1}{\sqrt{2\pi}} \int_{-\mu_- - \sigma\sqrt{T}}^{\infty} \exp \left( -x^2 / 2 \right) dx - Ke^{-rT} \Phi(\mu_-) \\
    &= S(0) \Phi \left( \mu_- + \sigma\sqrt{T} \right) - Ke^{-rT} \Phi(\mu_-)
    \end{align*}

    \par  \pause from which the conclusion follows, because

\[\mu_+ = \mu_- + \sigma\sqrt{T}\]
    }
    
\end{frame}
% \begin{frame}

%     {\footnotesize \footnotesize

%     }
    
% \end{frame}
% % {\mathbb{P}^*}
% \tilde{\mathbb{P}}
% {\footnotesize \footnotesize
% }
% \tiny
% \scriptsize
% \footnotesize
% \small
% \normalsize (default)
\end{document}