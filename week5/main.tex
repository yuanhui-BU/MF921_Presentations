\documentclass{beamer}

% \usepackage[utf8]{inputenc}
% \usepackage[T1]{fontenc}
\usepackage{lmodern}   % modern Latin Modern fonts
\usepackage{textcomp}  % provides \textquoteright
\usepackage{lmodern} % Latin Modern fonts with T1 shapes


\usepackage{graphicx}
\usepackage{ragged2e} % for generating dummy text
\usepackage[backend=biber,style=authoryear]{biblatex}
% \addbibresource{references.bib}

\usetheme{Madrid}
\usecolortheme{default}
\usefonttheme{professionalfonts} % keeps proper math fonts

\usepackage{amsmath,amssymb,amsfonts} % math symbols (\mathcal, \mathbb, etc.)
\usepackage{mathrsfs}                 % optional: \mathscr for fancy script

% \setbeamercovered{invisible} 
\setbeamercovered{transparent}


\title{MF921 Topics in Dynamic Asset Pricing}
\subtitle{Week 5}
\author{Yuanhui Zhao}
\date{Boston University}

\begin{document}
\frame{\titlepage}
% \begin{frame}
% \frametitle{Outline}
% \tableofcontents
% \end{frame}

\section{Chapter 14}
\begin{frame}{Chapter 14}

    \begin{center}
        \par Black-Scholes (II): Dominance-Free Interval and Risk Neutral Pricing
    \end{center}
    
\end{frame}

\begin{frame}{A General Brownian Market Model}

    {\footnotesize \footnotesize
    \par Given a complete probablity space \( (\Omega, \mathcal{F}, \mathbb{P}) \).
    \( W(t) = (W_1(t), \ldots, W_d(t))^\top \), independent \( d \)-dimensional Brownian motion. The filtraion 
    $\mathcal{F}_t^W = \sigma(W(s) : 0 \leq s \leq t)$ which is complete and right-continuous.
    \vspace{1em} 
    \par A financial market $\mathcal{M}$ with 1 bond and $d$ stocks under a finite horizon $[0,T]$:
    \begin{gather*}
    dS_0(t) = r(t)S_0(t)  dt, \quad S_0(0) = 1 \\
    dS_i(t) = S_i(t)\left(b_i(t)\,dt + \sum_{j=1}^d \sigma_{ij}(t)\,dW_j(t)\right),\;\;\text{for $i$ $\in 1,2....,d$}
    \end{gather*}
    \begin{itemize}
        \item \( r(t) \): interest rate
        \item  \( b(t) = (b_1, \ldots, b_d) \): appreciation rates
        \item \( \sigma(t) = (\sigma_{ij}(t)) \): volatility matrix
    \end{itemize}
    }   
\end{frame}
% \begin{frame}

%     {\footnotesize \footnotesize

%     }
    
% \end{frame}
% % {\mathbb{P}^*}
% \tilde{\mathbb{P}}
% {\footnotesize \footnotesize
% }
% \tiny
% \scriptsize
% \footnotesize
% \small
% \normalsize (default)
\end{document}