\documentclass{beamer}

% \usepackage[utf8]{inputenc}
% \usepackage[T1]{fontenc}
\usepackage{lmodern}   % modern Latin Modern fonts
\usepackage{textcomp}  % provides \textquoteright
\usepackage{lmodern} % Latin Modern fonts with T1 shapes


\usepackage{graphicx}
\usepackage{ragged2e} % for generating dummy text
\usepackage[backend=biber,style=authoryear]{biblatex}
% \addbibresource{references.bib}

\usetheme{Madrid}
\usecolortheme{default}
\usefonttheme{professionalfonts} % keeps proper math fonts

\usepackage{amsmath,amssymb,amsfonts} % math symbols (\mathcal, \mathbb, etc.)
\usepackage{mathrsfs}                 % optional: \mathscr for fancy script

% \setbeamercovered{invisible} 
\setbeamercovered{transparent}


\title{MF921 Topics in Dynamic Asset Pricing}
\subtitle{Week 6}
\author{Yuanhui Zhao}
\date{Boston University}

\begin{document}
\frame{\titlepage}
% \begin{frame}
% \frametitle{Outline}
% \tableofcontents
% \end{frame}


\begin{frame}{Chapter 22}

    {
    \begin{center}
        Chapter 22 American Options (II)
    \end{center}
    }
    
\end{frame}

\begin{frame}{Brownian Market Setup}

    {\footnotesize \footnotesize
    \par Given a complete probablity space \( (\Omega, \mathcal{F}, \mathbb{P}) \).
    \( W(t) = (W_1(t), \ldots, W_d(t))^\top \), independent \( d \)-dimensional Brownian motion. The filtraion 
    $\mathcal{F}_t^W = \sigma(W(s) : 0 \leq s \leq t)$ which is complete and right-continuous.
    \vspace{1em} 
    \par A financial market $\mathcal{M}$ with 1 bond and $d$ stocks under a finite horizon $[0,T]$:
    \begin{gather*}
    dS_0(t) = r(t)S_0(t)  dt, \quad S_0(0) = 1 \\
    dS_i(t) = S_i(t)\left(b_i(t)\,dt + \sum_{j=1}^d \sigma_{ij}(t)\,dW_j(t)\right),\;\;\text{for $i$ $\in 1,2....,d$}
    \end{gather*}
     \pause 
    \begin{itemize}
        \item \( r(t) \): interest rate
        \item  \( b(t) = (b_1, \ldots, b_d) \): appreciation rates
        \item \( \sigma(t) = (\sigma_{ij}(t)) \): volatility matrix
        \item \( r(t) \), \(b(t) \) and \( \sigma(t)\) all progressively measurable
         with respect to $\{\mathcal{F}_t\}$ and bounded uniformly in $(t,\omega) \in [0,T] \times \Omega$.
    \end{itemize}
    }   
\end{frame}

\begin{frame}{Introducing auxiliary processes}

    {\footnotesize \footnotesize
    \par Relative risk (Sharpe ratio): 
    \begin{align*}
            \theta(t) = \sigma^{-1}(t)\left(b(t) - r(t)\mathbf{1}\right),\;\; \mathbf{1} = (1,1,......,1)^T
    \end{align*}
    \par Exponential martingale (RN derivative): 
    \begin{align*}
        Z(t) = \exp\left(-\int_{0}^{t}\theta^{\top}(s)\,dW(s) - \frac{1}{2}\int_{0}^{t}\|\theta(s)\|^{2}ds\right)
    \end{align*}
    \par  \pause Discount factor:
    \begin{align*}
        \gamma(t) = \exp\left(-\int_{0}^{t}r(s)\,ds\right)
    \end{align*}
    \par Brownian motion with drift:
    \begin{align*}
         W_0(t) = W(t) + \int_{0}^{t}\theta(s)\,ds,\;\; 0\leq t\leq T
    \end{align*}
    \par  \pause $\sigma(t)$ invertible, inverses bounded. Ensures bounded $\theta(t)$ and $Z(t)$ is a true martingale.
    These tools set up the risk-neutral framework for pricing options.
    }   
\end{frame}

\begin{frame}{Self-Financing Condition}

    {\footnotesize \footnotesize
    \par A portfolio process is defined as $\pi(\cdot) = (\pi_1(\cdot), \ldots, \pi_d(\cdot))$, where $\pi_i(t) = \phi_i(t) S_i(t)$ means that 
    total amount of money invested in the $i$th risky asset. The self-financing condition leads to the wealth equation:
    \begin{align*}
        \gamma(t)X(t) = x - \int_0^t \gamma(s)dC(s) + \int_0^t \gamma(s)\pi^\top(s)\sigma(s)dW_0(s)
    \end{align*}
    \par where $C(t)$ is the cumulative consumption process, for the stochastic integral to be well-defined:
    \begin{align*}
        \int_{0}^{T} \|\pi(t)\|^{2}  dt < \infty
    \end{align*}
    }   
\end{frame}
 
 

\begin{frame}{Risk-Neutral Probability via Girsanov Theorem}

    {\footnotesize \footnotesize
    \par Definition of Risk-Neutral Measure:
    \begin{align*}
         \mathbb{P}^{0}(A) := \mathbb{E}[Z(T)1_{A}], \quad A \in \mathcal{F}_{T}
    \end{align*}
    \par $Z(T)$ is the exponential martingale. By Girsanov's Theorem, $ W_{0}(t) = W(t) + \int_{0}^{t} \theta(s) ds$ 
    is a standard Brownian motion under $\mathbb{P}^{0}$.
    \par  \pause Thus rewirte the wealth process:
    \begin{align*}
         N_{0}(t) &= \gamma(t)X(t) + \int_{0}^{t} \gamma(s)dC(s) \\
    &= x + \int_{0}^{t} \gamma(s)\pi^{\top}(s)\sigma(s) dW_{0}(s)
    \end{align*}
    \par A continuous $\mathbb{P}^{0}$-local martingale.
    }   
\end{frame} 

\begin{frame}{Risk-Neutral Probability via Girsanov Theorem}

    {\footnotesize \footnotesize
    \par Stock dynamics under risk-neutral measure:
    \begin{align*}
        dS_i(t) = S_i(t) \left[ b_i(t)dt - \sum_{j=1}^d \sigma_{ij}(t)\theta_j(t)dt + \sum_{j=1}^d \sigma_{ij}(t)dW_0^{(j)}(t) \right]
    \end{align*}
    \par Since $b(t) - \sigma(t)\theta(t) = r(t)\mathbf{1}$:
    \begin{align*}
        dS_i(t) = S_i(t) \left[r(t)dt + \sum_{j=1}^d \sigma_{ij}(t)dW_0^{(j)}(t) \right], \quad i = 1, \ldots, d
    \end{align*}
    \par  \pause Since $d\gamma(t) \cdot dS_i(t) = 0$, apply Itô:
    \begin{align*}
        d(\gamma(t)S_i(t)) &= S_i(t)dr(t) + \gamma(t)dS_i(t) \\
        &= \gamma(t)S_i(t) \sum_{j=1}^d \sigma_{ij}(t)dW_0^{(j)}(t)
    \end{align*}
    \par  \pause  Hence, the discounted stock processes $\gamma(\cdot)S_i(\cdot)$ are local martingales.
     This also confirms our intuition that every asset $S_i(t)$ should have a growth rate $r(t)$ in the risk neutral world.
    }   
\end{frame} 

\begin{frame}{Doubling Strategies and Admissibility}

    {\footnotesize \footnotesize
    \par Doubling Strategy: Double investment after each loss, leads to arbitrarily large wealth at $T$. 
    Requires wealth process $X(t)$ unbounded from below. Need to exclude, because creates arbitrage opportunities and 
    violates no-arbitrage principle.
    \vspace{1em}
    \par A uniform boundedness condition is needed to prevent the doubling strategy. Wealth must satisfy:
    \begin{align*}
        X^{x,\pi,C}(t) \geq -\Lambda, \quad 0 \leq t \leq T
    \end{align*}
    \par  With $\mathbb{E}^0[\Lambda^p] < \infty$ for some $p > 1$.
    \vspace{1em}
    \par   \pause Supermartingale Property:
    \vspace{1em}
    \par If $(\pi,C)$ is admissible, $N_0(t)$ is bound below.  $N_0(t)$ is a $\mathbb{P}^0$-supermartingale. Consequently:
    \begin{align*}
          \mathbb{E}^0 \left[ \gamma(T)X(T) + \int_0^T \gamma(t)dC(t) \right] \leq x
    \end{align*}
    \par  Expected discounted terminal wealth and consumption less than or equal to initial wealth to ensures no arbitrage.
    }
\end{frame} 

\begin{frame}{Dominance-Free Interval for American Options}

    {\footnotesize \footnotesize
   \par At \( t = 0 \), two agents agree: \textbf{Seller} promises to pay the buyer \(\psi(\tau, \omega) \geq 0\)
     at a stopping time \(\tau \in \mathcal{S}\) (chosen by the buyer). 
     \textbf{Buyer} pays upfront amount \(x \geq 0\) to seller.
     \vspace{1em}
    \par \(\psi(t, \omega)\): F-adapted, continuous process representing possible payoff. Integrability condition:
    \[
    \mathbb{E} \left[ \sup_{0 \leq t \leq T} \left( \gamma_0(t) \psi(t) \right)^{1 + \varepsilon} \right] < \infty, \quad \varepsilon > 0
    \]
    ensures finite expected discounted payoff. Such a process \(\psi(\cdot)\) defines an American Contingent Claim (ACC).
    \vspace{1em}
    \par Question: \textit{What should the buyer pay at \(t = 0\) for this option?}  
    i.e. find the arbitrage-free price allowing both sides to hedge. We will look at the situation of each agent
    separately
    \vspace{1em}
    \par Seller receives \(x\) at \(t = 0\) and seeks a self-financing portfolio \((\hat{\pi}, \hat{C})\) such that he can always meet the buyer's demand:

\[
X^{x,\hat{\pi},\hat{C}}(\tau) \geq \psi(\tau), \quad \forall \tau \in \mathcal{S}, \text{ a.s.}
\]
    }
\end{frame} 

\begin{frame}{Dominance-Free Interval for American Options}

    {\footnotesize \footnotesize
    The smallest initial capital \(x\) that makes this possible is

    \[
    h_{up} := \inf \left\{ x \geq 0 : \exists (\hat{\pi}, \hat{C}) \in \mathcal{A}_0(x) \text{ s.t. }
     X^{x,\hat{\pi},\hat{C}}(\tau) \geq \psi(\tau), \, \forall \tau \in \mathcal{S} \right\}
    \]

    Buyer pays \(x\) at \(t = 0 \Rightarrow\) and searches for a stopping time \(\hat{\tau}\) 
    and portfolio \((\hat{\pi}, \hat{C})\) such that the payment that he receives allows him to recover
    the debt he incurred at $t=0$ by purchasing the ACC:

    \[
    X^{-x,\hat{\pi},\hat{C}}(\hat{\tau}) + \psi(\hat{\tau}) \geq 0, \quad \text{a.s.}
    \]

    The largest \(x\) that allows this is

    \[
    h_{low} := \sup \left\{ x \geq 0 : \exists (\hat{\pi}, \hat{C}) \in \mathcal{A}_0(-x) \text{ s.t. } 
    X^{-x,\hat{\pi},\hat{C}}(\hat{\tau}) + \psi(\hat{\tau}) \geq 0 \right\}
    \]
    \par Note: Seller needs to hedge against any stopping time \(\tau \in \mathcal{S}\), 
    whereas the buyer need only hedge for some stopping time \(\tilde{\tau} \in \mathcal{S}\). We need to justify
    defined "upper" and "lower" hedging price.
    }
\end{frame} 
\begin{frame}{Dominance-Free Interval for American Options}

    {\footnotesize \footnotesize
    Consider the decreasing function
\[
u(t) =: \sup_{\tau \in S_{t,T}} E^0[\gamma_0(\tau)\psi(\tau)], \quad 0 \leq t \leq T
\]
We must have

\[
0 \leq \psi(0) \leq h_{\text{low}} \leq u(0) \leq h_{\text{up}} \leq \infty
\]
If $h_{up}$ is empty set, then $h_{\text{up}} = \infty$ and $h_{\text{up}} \geq u(0)$ holds trivially; if not,
let x be an arbitrary element of this set. Under the risk-neutral measure \( P^0 \), 
from the supermartingale property of discounted wealth and seller hedging condition.
\[
x \geq E^0 \left[ \gamma_0(\tau)X^{x,\tilde{\pi},\tilde{C}}(\tau) + \int_{(0,\tau]} 
\gamma_0(t)d\tilde{C}(t) \right] \geq E^0[\gamma_0(\tau)\psi(\tau)]; \quad \forall \tau \in S
\]
Taking the supremum over all stopping times gives $x \geq \sup_{\tau \in \mathcal{S}} E^0 [\gamma_0(\tau) \psi(\tau)] = u(0).$
Thus, any admissible initial capital \( x \) that allows perfect hedging must satisfy \( x \geq u(0) \). 
Therefore $h_{\text{up}} \geq u(0).$
    }
\end{frame} 

\begin{frame}{Dominance-Free Interval for American Options}

    {\footnotesize \footnotesize
     On the other hand, the number \( \psi(0) \) clearly belongs to the set of $h_{low}$ ( 
     take \( x = \psi(0) \geq 0, \, \tilde{\pi} = 0, 
     \, \tilde{\pi}(\cdot) \equiv 0, \, \tilde{C}(\cdot) \equiv 0 \)). 
     \vspace{1em}
     Applying again the discounted wealth supermartingale property and buyer hedging condition, we get:
     \begin{align*}
        -x \geq E^0 \left[ \gamma_0(\tilde{\tau})X^{-x,\tilde{\pi},\tilde{C}}(\tilde{\tau}) + \int_{(0,\tilde{\tau}]} 
        \gamma_0(t)d\tilde{C}(t) \right] \geq -E^0[\gamma_0(\tilde{\tau})\psi(\tilde{\tau})] \geq -u(0)
     \end{align*}
     $x \leq  u(0)$, therefore $h_{low} \leq u(0)$.
    
     The integrability condition, and the boundedness of the process \( \theta(\cdot) \) implies that  

\[
\text{E}^0 \left[ \sup_{0 \leq t \leq T} (\gamma_0(t)\psi(t)) \right] = \text{E} \left[ Z_0(T) \cdot \sup_{0 \leq t \leq T} (\gamma_0(t)\psi(t)) \right]
\]
\[
\leq (\text{E}(Z_0(T))^q)^{1/q} \cdot \left( \text{E} \sup_{0 \leq t \leq T} (\gamma_0(t)\psi(t))^p \right)^{1/p} < \infty
\]

with \( p = 1 + \epsilon > 1, \, \frac{1}{p} + \frac{1}{q} = 1 \). In particular, $u(0) < \infty$. 
It can be shown that \( [h_{\text{low}}, h_{\text{up}}] \) forms a dominance-free interval.
    }
\end{frame} 


\begin{frame}{The Unique Price for American Options in an Ideal Market}

    {\footnotesize \footnotesize
     \textbf{Theorem}  In an ideal market

    \[
    h_{\text{up}} = h_{\text{low}} = u(0) =: \sup_{\tau \in S} E^0[\gamma_0(\tau)\psi(\tau)] < \infty
    \]

    Furthermore, there exists a pair \((\hat{\pi}, \hat{C}) \in A_0(u(0))\) such that, with

    \[
    \hat{X}_0(t) := \frac{1}{\gamma_0(t)} \text{ess}\sup_{\tau \in S_{t,T}} 
    E^0[\gamma_0(\tau)\psi(\tau)|F(t)], \quad 0 \leq t \leq T
    \]

    \[
    \check{\sigma} =: \inf\{t \in [0, T)/\hat{X}_0(t) = \psi(t)\} \land T
    \]

    and \(\hat{\pi}(\cdot) \equiv -\hat{\pi}(\cdot)\), we have almost surely:

    \begin{align*}
    X^{u(0), \hat{\pi}, \hat{C}}(t) &= \hat{X}_0(t) \geq \psi(t), \quad \forall \, 0 \leq t \leq T \\
    X^{u(0), \hat{\pi}, \hat{C}}(t) &= -X^{-u(0), \hat{\pi}, 0}(t) > \psi(t), \quad \forall \, 0 \leq t < tilde{\tau}\\
    \hat{C}(\tilde{\tau}) = 0, \quad &X^{u(0), \hat{\pi}, \hat{C}}(\tilde{\tau}) = -X^{-u(0), \hat{\pi}, 0}(\tilde{\tau}) = \psi(\tilde{\tau})
    \end{align*}
    }
\end{frame} 

% \begin{frame}

%     {\footnotesize \footnotesize

%     }
    
% \end{frame}
% % {\mathbb{P}^*}
% \tilde{\mathbb{P}}
% {\footnotesize \footnotesize
% }
% \tiny
% \scriptsize
% \footnotesize
% \small
% \normalsize (default)
\end{document}