\documentclass{beamer}

% \usepackage[utf8]{inputenc}
% \usepackage[T1]{fontenc}
\usepackage{lmodern}   % modern Latin Modern fonts
\usepackage{textcomp}  % provides \textquoteright
\usepackage{lmodern} % Latin Modern fonts with T1 shapes


\usepackage{graphicx}
\usepackage{ragged2e} % for generating dummy text
\usepackage[backend=biber,style=authoryear]{biblatex}
% \addbibresource{references.bib}
\DeclareMathOperator*{\esssup}{ess\,sup}
\usetheme{Madrid}
\usecolortheme{default}
\usefonttheme{professionalfonts} % keeps proper math fonts

\usepackage{amsmath,amssymb,amsfonts} % math symbols (\mathcal, \mathbb, etc.)
\usepackage{mathrsfs}                 % optional: \mathscr for fancy script

% \setbeamercovered{invisible} 
\setbeamercovered{transparent}


\title{MF921 Topics in Dynamic Asset Pricing}
\subtitle{Week 6}
\author{Yuanhui Zhao}
\date{Boston University}

\begin{document}
\frame{\titlepage}
% \begin{frame}
% \frametitle{Outline}
% \tableofcontents
% \end{frame}


\begin{frame}{Chapter 22}

    {
    \begin{center}
        Chapter 22 American Options (II)
    \end{center}
    }
    
\end{frame}

\begin{frame}{Brownian Market Setup}

    {\footnotesize \footnotesize
    \par Given a complete probablity space \( (\Omega, \mathcal{F}, \mathbb{P}) \).
    \( W(t) = (W_1(t), \ldots, W_d(t))^\top \), independent \( d \)-dimensional Brownian motion. The filtraion 
    $\mathcal{F}_t^W = \sigma(W(s) : 0 \leq s \leq t)$ which is complete and right-continuous.
    \vspace{1em} 
    \par A financial market $\mathcal{M}$ with 1 bond and $d$ stocks under a finite horizon $[0,T]$:
    \begin{gather*}
    dS_0(t) = r(t)S_0(t)  dt, \quad S_0(0) = 1 \\
    dS_i(t) = S_i(t)\left(b_i(t)\,dt + \sum_{j=1}^d \sigma_{ij}(t)\,dW_j(t)\right),\;\;\text{for $i$ $\in 1,2....,d$}
    \end{gather*}
     \pause 
    \begin{itemize}
        \item \( r(t) \): interest rate
        \item  \( b(t) = (b_1, \ldots, b_d) \): appreciation rates
        \item \( \sigma(t) = (\sigma_{ij}(t)) \): volatility matrix
        \item \( r(t) \), \(b(t) \) and \( \sigma(t)\) all progressively measurable
         with respect to $\{\mathcal{F}_t\}$ and bounded uniformly in $(t,\omega) \in [0,T] \times \Omega$.
    \end{itemize}
    }   
\end{frame}

\begin{frame}{Introducing auxiliary processes}

    {\footnotesize \footnotesize
    \par Relative risk (Sharpe ratio): 
    \begin{align*}
            \theta(t) = \sigma^{-1}(t)\left(b(t) - r(t)\mathbf{1}\right),\;\; \mathbf{1} = (1,1,......,1)^T
    \end{align*}
    \par Exponential martingale (RN derivative): 
    \begin{align*}
        Z(t) = \exp\left(-\int_{0}^{t}\theta^{\top}(s)\,dW(s) - \frac{1}{2}\int_{0}^{t}\|\theta(s)\|^{2}ds\right)
    \end{align*}
    \par  \pause Discount factor:
    \begin{align*}
        \gamma(t) = \exp\left(-\int_{0}^{t}r(s)\,ds\right)
    \end{align*}
    \par Brownian motion with drift:
    \begin{align*}
         W_0(t) = W(t) + \int_{0}^{t}\theta(s)\,ds,\;\; 0\leq t\leq T
    \end{align*}
    \par  \pause $\sigma(t)$ invertible, inverses bounded. Ensures bounded $\theta(t)$ and $Z(t)$ is a true martingale.
    These tools set up the risk-neutral framework for pricing options.
    }   
\end{frame}

\begin{frame}{Self-Financing Condition}

    {\footnotesize \footnotesize
    \par A portfolio process is defined as $\pi(\cdot) = (\pi_1(\cdot), \ldots, \pi_d(\cdot))$, where $\pi_i(t) = \phi_i(t) S_i(t)$ means that 
    total amount of money invested in the $i$th risky asset. The self-financing condition leads to the wealth equation:
    \begin{align*}
        \gamma(t)X(t) = x - \int_0^t \gamma(s)dC(s) + \int_0^t \gamma(s)\pi^\top(s)\sigma(s)dW_0(s)
    \end{align*}
    \par where $C(t)$ is the cumulative consumption process, for the stochastic integral to be well-defined:
    \begin{align*}
        \int_{0}^{T} \|\pi(t)\|^{2}  dt < \infty
    \end{align*}
    }   
\end{frame}
 
 

\begin{frame}{Risk-Neutral Probability via Girsanov Theorem}

    {\footnotesize \footnotesize
    \par Definition of Risk-Neutral Measure:
    \begin{align*}
         \mathbb{P}^{0}(A) := \mathbb{E}[Z(T)1_{A}], \quad A \in \mathcal{F}_{T}
    \end{align*}
    \par $Z(T)$ is the exponential martingale. By Girsanov's Theorem, $ W_{0}(t) = W(t) + \int_{0}^{t} \theta(s) ds$ 
    is a standard Brownian motion under $\mathbb{P}^{0}$.
    \par  \pause Thus rewirte the wealth process:
    \begin{align*}
         N_{0}(t) &= \gamma(t)X(t) + \int_{0}^{t} \gamma(s)dC(s) \\
    &= x + \int_{0}^{t} \gamma(s)\pi^{\top}(s)\sigma(s) dW_{0}(s)
    \end{align*}
    \par A continuous $\mathbb{P}^{0}$-local martingale.
    }   
\end{frame} 

\begin{frame}{Risk-Neutral Probability via Girsanov Theorem}

    {\footnotesize \footnotesize
    \par Stock dynamics under risk-neutral measure:
    \begin{align*}
        dS_i(t) = S_i(t) \left[ b_i(t)dt - \sum_{j=1}^d \sigma_{ij}(t)\theta_j(t)dt + \sum_{j=1}^d \sigma_{ij}(t)dW_0^{(j)}(t) \right]
    \end{align*}
    \par Since $b(t) - \sigma(t)\theta(t) = r(t)\mathbf{1}$:
    \begin{align*}
        dS_i(t) = S_i(t) \left[r(t)dt + \sum_{j=1}^d \sigma_{ij}(t)dW_0^{(j)}(t) \right], \quad i = 1, \ldots, d
    \end{align*}
    \par  \pause Since $d\gamma(t) \cdot dS_i(t) = 0$, apply Itô:
    \begin{align*}
        d(\gamma(t)S_i(t)) &= S_i(t)dr(t) + \gamma(t)dS_i(t) \\
        &= \gamma(t)S_i(t) \sum_{j=1}^d \sigma_{ij}(t)dW_0^{(j)}(t)
    \end{align*}
    \par  \pause  Hence, the discounted stock processes $\gamma(\cdot)S_i(\cdot)$ are local martingales.
     This also confirms our intuition that every asset $S_i(t)$ should have a growth rate $r(t)$ in the risk neutral world.
    }   
\end{frame} 

\begin{frame}{Doubling Strategies and Admissibility}

    {\footnotesize \footnotesize
    \par Doubling Strategy: Double investment after each loss, leads to arbitrarily large wealth at $T$. 
    Requires wealth process $X(t)$ unbounded from below. Need to exclude, because creates arbitrage opportunities and 
    violates no-arbitrage principle.
    \vspace{1em}
    \par A uniform boundedness condition is needed to prevent the doubling strategy. Wealth must satisfy:
    \begin{align*}
        X^{x,\pi,C}(t) \geq -\Lambda, \quad 0 \leq t \leq T
    \end{align*}
    \par  With $\mathbb{E}^0[\Lambda^p] < \infty$ for some $p > 1$.
    \vspace{1em}
    \par   \pause Supermartingale Property:
    \vspace{1em}
    \par If $(\pi,C)$ is admissible, $N_0(t)$ is bound below.  $N_0(t)$ is a $\mathbb{P}^0$-supermartingale. Consequently:
    \begin{align*}
          \mathbb{E}^0 \left[ \gamma(T)X(T) + \int_0^T \gamma(t)dC(t) \right] \leq x
    \end{align*}
    \par  Expected discounted terminal wealth and consumption less than or equal to initial wealth to ensures no arbitrage.
    }
\end{frame} 

\begin{frame}{Dominance-Free Interval for American Options}

    {\footnotesize \footnotesize
   \par At \( t = 0 \), two agents agree: \textbf{Seller} promises to pay the buyer \(\psi(\tau, \omega) \geq 0\)
     at a stopping time \(\tau \in \mathcal{S}\) (chosen by the buyer). 
     \textbf{Buyer} pays upfront amount \(x \geq 0\) to seller.
     \vspace{1em}
    \par \(\psi(t, \omega)\): F-adapted, continuous process representing possible payoff. Integrability condition:
    \[
    \mathbb{E} \left[ \sup_{0 \leq t \leq T} \left( \gamma_0(t) \psi(t) \right)^{1 + \varepsilon} \right] < \infty, \quad \varepsilon > 0
    \]
    ensures finite expected discounted payoff. Such a process \(\psi(\cdot)\) defines an American Contingent Claim (ACC).
    \vspace{1em}
    \par \pause  Question: \textit{What should the buyer pay at \(t = 0\) for this option?}  
    i.e. find the arbitrage-free price allowing both sides to hedge. We will look at the situation of each agent
    separately
    \vspace{1em}
    \par Seller receives \(x\) at \(t = 0\) and seeks a self-financing portfolio \((\hat{\pi}, \hat{C})\) such that he can always meet the buyer's demand:

\[
X^{x,\hat{\pi},\hat{C}}(\tau) \geq \psi(\tau), \quad \forall \tau \in \mathcal{S}, \text{ a.s.}
\]
    }
\end{frame} 

\begin{frame}{Dominance-Free Interval for American Options}

    {\footnotesize \footnotesize
    The smallest initial capital \(x\) that makes this possible is

    \[
    h_{up} := \inf \left\{ x \geq 0 : \exists (\hat{\pi}, \hat{C}) \in \mathcal{A}_0(x) \text{ s.t. }
     X^{x,\hat{\pi},\hat{C}}(\tau) \geq \psi(\tau), \, \forall \tau \in \mathcal{S} \right\}
    \]

    Buyer pays \(x\) at \(t = 0 \Rightarrow\) and searches for a stopping time \(\hat{\tau}\) 
    and portfolio \((\hat{\pi}, \hat{C})\) such that the payment that he receives allows him to recover
    the debt he incurred at $t=0$ by purchasing the ACC:

    \[
    X^{-x,\hat{\pi},\hat{C}}(\hat{\tau}) + \psi(\hat{\tau}) \geq 0, \quad \text{a.s.}
    \]

     \pause The largest \(x\) that allows this is

    \[
    h_{low} := \sup \left\{ x \geq 0 : \exists (\hat{\pi}, \hat{C}) \in \mathcal{A}_0(-x) \text{ s.t. } 
    X^{-x,\hat{\pi},\hat{C}}(\hat{\tau}) + \psi(\hat{\tau}) \geq 0 \right\}
    \]
    \par Note: Seller needs to hedge against any stopping time \(\tau \in \mathcal{S}\), 
    whereas the buyer need only hedge for some stopping time \(\tilde{\tau} \in \mathcal{S}\). We need to justify
    defined "upper" and "lower" hedging price.
    }
\end{frame} 
\begin{frame}{Dominance-Free Interval for American Options}

    {\footnotesize \footnotesize
    Consider the decreasing function
\[
u(t) =: \sup_{\tau \in S_{t,T}} E^0[\gamma_0(\tau)\psi(\tau)], \quad 0 \leq t \leq T
\]
We must have

\[
0 \leq \psi(0) \leq h_{\text{low}} \leq u(0) \leq h_{\text{up}} \leq \infty
\]
 \pause If $h_{up}$ is empty set, then $h_{\text{up}} = \infty$ and $h_{\text{up}} \geq u(0)$ holds trivially; if not,
let x be an arbitrary element of this set. Under the risk-neutral measure \( P^0 \), 
from the supermartingale property of discounted wealth and seller hedging condition.
\[
x \geq E^0 \left[ \gamma_0(\tau)X^{x,\tilde{\pi},\tilde{C}}(\tau) + \int_{(0,\tau]} 
\gamma_0(t)d\tilde{C}(t) \right] \geq E^0[\gamma_0(\tau)\psi(\tau)]; \quad \forall \tau \in S
\]
 \pause Taking the supremum over all stopping times gives $x \geq \sup_{\tau \in \mathcal{S}} E^0 [\gamma_0(\tau) \psi(\tau)] = u(0).$
Thus, any admissible initial capital \( x \) that allows perfect hedging must satisfy \( x \geq u(0) \). 
Therefore $h_{\text{up}} \geq u(0).$
    }
\end{frame} 

\begin{frame}{Dominance-Free Interval for American Options}

    {\footnotesize \footnotesize
     On the other hand, the number \( \psi(0) \) clearly belongs to the set of $h_{low}$ ( 
     take \( x = \psi(0) \geq 0, \, \tilde{\pi} = 0, 
     \, \tilde{\pi}(\cdot) \equiv 0, \, \tilde{C}(\cdot) \equiv 0 \)). 
     \vspace{1em}
     Applying again the discounted wealth supermartingale property and buyer hedging condition, we get:
     \begin{align*}
        -x \geq E^0 \left[ \gamma_0(\tilde{\tau})X^{-x,\tilde{\pi},\tilde{C}}(\tilde{\tau}) + \int_{(0,\tilde{\tau}]} 
        \gamma_0(t)d\tilde{C}(t) \right] \geq -E^0[\gamma_0(\tilde{\tau})\psi(\tilde{\tau})] \geq -u(0)
     \end{align*}
     $x \leq  u(0)$, therefore $h_{low} \leq u(0)$.
    
      \pause The integrability condition, and the boundedness of the process \( \theta(\cdot) \) implies that  

\[
\text{E}^0 \left[ \sup_{0 \leq t \leq T} (\gamma_0(t)\psi(t)) \right] = \text{E} \left[ Z_0(T) \cdot \sup_{0 \leq t \leq T} (\gamma_0(t)\psi(t)) \right]
\]
\[
\leq (\text{E}(Z_0(T))^q)^{1/q} \cdot \left( \text{E} \sup_{0 \leq t \leq T} (\gamma_0(t)\psi(t))^p \right)^{1/p} < \infty
\]

with \( p = 1 + \epsilon > 1, \, \frac{1}{p} + \frac{1}{q} = 1 \). In particular, $u(0) < \infty$. 
It can be shown that \( [h_{\text{low}}, h_{\text{up}}] \) forms a dominance-free interval.
    }
\end{frame} 


\begin{frame}{The Unique Price for American Options in an Ideal Market}

    {\footnotesize \footnotesize
     \textbf{Theorem}  In an ideal market

    \[
    h_{\text{up}} = h_{\text{low}} = u(0) =: \sup_{\tau \in S} E^0[\gamma_0(\tau)\psi(\tau)] < \infty
    \]

    Furthermore, there exists a pair \((\hat{\pi}, \hat{C}) \in A_0(u(0))\) such that, with

    \[
    \hat{X}_0(t) := \frac{1}{\gamma_0(t)} \esssup_{\tau \in S_{t,T}} 
    E^0[\gamma_0(\tau)\psi(\tau)|\mathcal{F}_t], \quad 0 \leq t \leq T
    \]

    \[
    \check{\sigma} =: \inf\{t \in [0, T)/\hat{X}_0(t) = \psi(t)\} \land T
    \]

     \pause and \(\check{\pi}(\cdot) \equiv -\hat{\pi}(\cdot)\), we have almost surely:

    \begin{align*}
    X^{u(0), \hat{\pi}, \hat{C}}(t) &= \hat{X}_0(t) \geq \psi(t), \quad \forall \, 0 \leq t \leq T \\
    X^{u(0), \hat{\pi}, \hat{C}}(t) &= -X^{-u(0), \check{\pi}, 0}(t) > \psi(t), \quad \forall \, 0 \leq t < \check{\tau}\\
    \hat{C}(\check{\tau}) = 0, \quad &X^{u(0), \hat{\pi}, \hat{C}}(\check{\tau}) = -X^{-u(0), \check{\pi}, 0}(\check{\tau}) 
    = \psi(\check{\tau})
    \end{align*}
    
    The portfolio \(\hat{\pi}(\cdot)\) (respectively, \(\check{\pi}(\cdot)\)) is the optimal hedging portfolio 
    for the seller (respectively, the buyer). 
    The stopping time of \(\check{\tau}\) is the optimal exercise time for the buyer.
    }
\end{frame} 


\begin{frame}{Results from Optimal Stopping}

    {\footnotesize \footnotesize
     The Snell envelope of the discounted payoff process of \(\gamma_0(\cdot) \psi(\cdot)\):
    \[
    Z(t) := \esssup_{\tau \in S_{t,T}} E^0 [\gamma_0(\tau) \psi(\tau) | \mathcal{F}_t], \quad 0 \leq t \leq T
    \]
    At time \(t\), given the current information \(\mathcal{F}_t\), \(Z(t)\) is the maximum expected discounted payoff
    achievable if you can choose the future stopping time \(\tau \geq t\). This is a conditional version of the optimal stopping value:
    \[
    u(t) = E^0[Z(t)] = \sup_{\tau \in S_{t,T}} E^0[\gamma_0(\tau)\psi(\tau)] 
    \]

    $Z(\cdot)$ is $\mathcal{F}_t$-adapted and right-continuous with left limits (RCLL) with the following properties:
     \pause \begin{itemize}
        \item \( Z(\cdot) \) is a \(P^0\)-supermartingale, dominates \(\gamma_0(\cdot) \psi(\cdot)\).
        If you wait longer (go from \( t \) to \( s > t \)), your best expected
        discounted payoff can only decrease, because there are fewer opportunities 
        to stop.
        And among all supermartingales that dominate the discounted payoff process,  
    \( Z(\cdot) \) is the smallest one.
        
        
    \end{itemize}
    }
\end{frame} 
\begin{frame}{Results from Optimal Stopping}

    {\footnotesize \footnotesize
    \begin{itemize}
        \item For every \( t \in [0, T] \) and with
        \[
        \check{\tau}_t := \inf\{u \in [t, T) : Z(u) = \gamma_0(u) \psi(u)\} \land T
        \]
        the process \( Z(\cdot \land \check{\tau}_t) \) is a \(P^0\)-martingale on \([t, T]\), and we have
        \[
        E^0Z(t) = u(t) := \sup_{\tau \in S_{t,T}} E^0[\gamma_0(\tau) \psi(\tau)] = 
        E^0 [\gamma_0(\check{\tau}_t) \psi(\check{\tau}_t)]
        \]
        
         \item \pause  The \(P^0\)-supermartingale \( Z(\cdot) \) has a unique Doob-Meyer decomposition
        \[
        Z(t) = u(0) + M(t) - A(t), \quad 0 \leq t \leq T
        \]
        Here \( M(\cdot) \) is a uniformly integrable \(P^0\)-martingale, and \( A(\cdot) \) a continuous,
         increasing F-adapted process with \( A(0) = M(0) = 0 \), 
         \( E^0 A(T) < \infty \).
        
        \item The process \( A(\cdot) \) satisfies, a.s.
        \[
        \int_{0}^{T} (Z(t) - \gamma_0(t) \psi(t)) dA(t) = 0, \quad A(\tilde{\tau}_t) 
        = A(t) \text{ for } 0 \leq t \leq T
        \]
        In other words, \( A(\cdot) \) is flat off the set \(\{0 \leq t \leq T / Z(t) = \gamma_0(t) \psi(t)\}\).
    \end{itemize}
    }
    
\end{frame}

\begin{frame}{Proof of Theorem 1}

    {\footnotesize \footnotesize
    From above we know that the Snell envelope \( Z(t) \) satisfies:
    \[
    Z(t) = u(0) + M(t) - A(t)
    \]
    By the martingale representation theorem (under Brownian filtration), every square-integrable \( P^0\)-martingale can be written as a stochastic integral:
    \begin{align*}
        M(t) &= E^0[M(T)|\mathcal{F}_t] = E^0[\gamma_0(T)\psi(T) + A(T)|\mathcal{F}_t] - u(0)\\
    &= \int_0^t \psi^*(s) \, dW_0(s), \quad 0 \leq t \leq T
    \end{align*}

    \pause  for some \( \mathcal{F}(t) \)-progressively measurable process 
    \( \psi : [0,T] \times \Omega \to \mathbb{R}^d \) with

    \[
    \int_0^T \|\psi(t)\|^2 dt < \infty
    \]
    almost surely. Note that, just like the case of European option, a Bayes formula has to be used to justify
     the existence of \(\psi(t)\), because \(\mathcal{F}(t) \) is a filtration generated by \(W(t)\) not \(W_0(t)\).


    }

    
\end{frame}

\begin{frame}{Proof of Theorem 1}

    {\footnotesize \footnotesize
    The nonnegative process \(\hat{X}_0(\cdot) = Z(t)/\gamma_0(t)\)  is seen to satisfy almost surely

    \[
    \gamma_0(t)\hat{X}_0(t) = Z(t) = u(0) + \int_0^t \psi^*(s)dW_0(s) - A(t), \quad 0 \leq t \leq T
    \]
    
    Thus, we can define a portfolio/consumption process pair \((\hat{\pi}, \hat{C})\):

    \[
    \hat{\pi}(t) =: \frac{1}{\gamma_0(t)} (\sigma^*(t))^{-1} \psi(t),\;\;\;\hat{C}(t) =: \int_0^t \frac{1}{\gamma_0(s)} dA(s), \quad 0 \leq t \leq T
    \]


     \pause We have \(\int_0^T \|\hat{\pi}(t)\|^2 dt < \infty\) a.s., and can rewrite above equation as

    \[
    \gamma_0(t)\hat{X}(t) = u(0) + \int_0^t \gamma_0(s) \hat{\pi}^*(s) \sigma(s) dW_0(s) 
    - \int_0^t \gamma_0(s) d\hat{C}(s), \quad 0 \leq t \leq T
    \]

    This is exactly the discounted wealth equation for a self-financing portfolio with consumption 
    $\hat{C}$. Thus, \(\hat{X}_0(\cdot) \equiv X^{u(0), \hat{\pi}, \hat{C}}(\cdot)\). Clearly
     \(\check{\tau} = \check{\tau}_0\) and \(\hat{C}(\check{\tau}) = 0\) a.s. and 
     the other claims also follow. The theorem is proved.
    }

    
\end{frame}


\begin{frame}{The early exercise premium}

    {\footnotesize \footnotesize
     The early exercise premium

    \[
    e(t) =: \frac{1}{\gamma_0(t)} \left[ \sup_{\tau \in S_{t,T}} E^0[\gamma_0(\tau)\psi(\tau)|\mathcal{F}_t] - 
    E^0[\gamma_0(T)\psi(T)|\mathcal{F}_t] \right], \quad 0 \leq t \leq T
    \]

    is the difference between the prices, at time \(t\), of the American Contingent Claim \(\psi(\cdot)\)
    and the European Contingent Claim \(\psi(T)\). 
    
     \pause It is easily seen that

    \[
    e(t) = \frac{1}{\gamma_0(t)} E^0[A(T) - A(t)|\mathcal{F}_t] = \frac{1}{\gamma_0(t)} E^0 \left[ \int_t^T 
    \gamma_0(s)d\hat{C}(s)|\mathcal{F}_t\right]
    \]

    in other words, the consumption process \(\hat{C}(\cdot)\) can be interpreted as a ``cash-flow'', 
    whose expected discounted value \(E^0 \int_0^T \gamma_0(t)d\hat{C}(t)\)
    yields the early exercise premium \(e(0)\) at \(t = 0\).
    }

    
\end{frame}
       
\begin{frame}{American Call-Option}

    {\footnotesize \footnotesize
    Consider an American Call-Option on Non-dividend payting stock
    \[
    \psi(t) = (S_i(t) - K)^+, \quad 0 \leq t \leq T
    \]
    With \(r(\cdot) \geq 0\) and \(0 \leq K < \infty\). In this case

    \[
    \gamma_0(t)S_i(t) \text{ is a } P^0\text{-martingale}, \quad K\gamma_0(t) \text{ is decreasing}
    \]
    So that \(\gamma_0(\cdot)(S_i(\cdot) - K)\) is a \(P^0\)-submartingale. 
    
     \pause By Jensen's inequality, \(\gamma_0(\cdot)\psi(\cdot) = 
    \gamma_0(\cdot)(S_i(\cdot)-K)^+\) is again a \(P^0\)-submartingale, 
    and the optimal stopping problem admits a trivial solution:
    \[
    \check{\tau} = T, \quad \text{and} \quad e(\cdot) \equiv 0
    \]

    The American call-option (on a stock without dividends) is, under these conditions, equivalent to the European call-option, and thus

    \[
    u(0) = E^0[\gamma_0(T)(S_i(T)-K)^+]
    \]

    }

    
\end{frame}   

\begin{frame}{American Put Option}

    {\footnotesize \footnotesize
    We consider an American put option with payoff

    \[
    \psi(t) = (K - S(t))^+, \quad 0 \leq t \leq T
    \]

    on a single stock \(S(t)\) following (under \(P^0\)):

    \[
    dS(t) = S(t)[r \, dt + \sigma \, dW^0(t)], \quad r > 0, \quad \sigma > 0, \quad K > 0
    \]

     \pause The discounted value process is

    \[
    \hat{X}_0(t) = \esssup_{\tau \geq t} E^0 
    \left[ e^{-r(\tau-t)} (K - S(\tau))^+ \mid \mathcal{F}_t \right] =: V(t, S(t))
    \]

    Then

    \[
    V(x, t) := \sup_{\tau \geq t} E^0 \left[ e^{-r(\tau-t)} (K - S(\tau))^+ \mid S(t) = x \right]
    \]

    is the price function of the American put at time \(t\).

    }

    
\end{frame}   

\begin{frame}{American Put Option}

    {\footnotesize \footnotesize
    It can be shown that this function \( V : (0, \infty) \times (0, \infty) \rightarrow [0, K] \) 
    is strictly positive; continuous; decreasing in the variable \( t \); decreasing, 
    convex and continuously differentiable in the variable \( x \); and the 
    function \( x \mapsto x + V(t, x) \) is increasing for every \( t > 0 \). 
    On the other hand, the continuation region is given as  
    \[
    \mathcal{C} = (b(t), \infty), \quad 0 < b(t) < K
    \]  
    
     \pause In other words, the optimal exercise time is  
    \[
    \check{\tau} = \inf\{t \in [0, T) : S(t) \leq b(t)\} \land T
    \]  

    for the buyer. The optimal hedging portfolio in terms of percentage shares is  
    \[
    \hat{p}(t) = S(t) \cdot \frac{\partial V}{\partial x} (S(t), t)/V(S(t), t), \quad 0 \leq t \leq T
    \]  

    for the seller, the cash-flow process is  
    \[
    \hat{C}(t) = rK \int_{0}^{t} 1_{\{S(u) \leq b(u)\}} du, \quad 0 \leq t \leq T
    \]
    }

    
\end{frame} 

\begin{frame}{American Put Option}

    {\footnotesize \footnotesize
    The resulting moving boundary function \( b : (0, \infty) \to (0, K) \) turns out to be continuous and increasing, with \( b(T - ) = K \). As \( T \to \infty \), the function \( V \) goes to the value of the infinite time optimal stopping solution with the exercise boundary being \( \bar{b} \),

    \[
    \lim_{T \to \infty} V(x, t) = (K - \bar{b}) \left( \frac{\bar{b}}{x} \right)^\gamma; \quad x > \bar{b},
    \]

    \[
    \lim_{T \to \infty} V(x, t) = K - x; \quad 0 < x < \bar{b},
    \]

    and

    \[
    \bar{b} = \frac{K \gamma}{1 + \gamma}, \quad \gamma = \frac{1}{\sigma} \left[ \left( \frac{r}{\sigma} - \frac{\sigma}{2} \right) + \sqrt{\left( \frac{r}{\sigma} - \frac{\sigma}{2} \right)^2 + 2r} \right].
    \]
    }
\end{frame} 

\begin{frame}{Numerical Methods for American Options}

    {\footnotesize \footnotesize
     Start again from the definition:

    \[
    V(x,t) := \sup_{\tau \geq t} E^0 \left[ e^{-r(\tau - t)} \varphi(S(\tau)) \mid S(t) = x \right],
    \]

    where \( S(t) = \) stock price, \(\varphi(\cdot) = \) payoff function (e.g. \((K - S)^+\) for 
    a put), \(\tau = \) stopping time (the exercise time). So \( V(x,t) \) is the value function
    represent the highest discounted expected payoff you can achieve by optimally choosing when to stop.
    \vspace{1em}
    
     \pause In the region where you don't exercise (continuation region \(x > b(t)\)), 
    the option behaves like a European one — no payoff is taken yet. Under the risk-neutral measure,
    
    \[
    dS(t) = rS(t) \, dt + \sigma S(t) \, dW^0(t)
    \]
    so the generator (infinitesimal operator) of \(S(t)\) is 
    \[
    \mathcal{A} V = \frac{\partial V}{\partial t} + rx \frac{\partial V}{\partial x} + 
    \frac{1}{2} \sigma^2 x^2 \frac{\partial^2 V}{\partial x^2}
    \]

    Since the discounted value \(e^{-rt} V(S(t), t)\) must be 
    a martingale while waiting (no early exercise), its drift must be zero:

    \[
    \mathcal{A} V - rV = 0, \quad x > b(t), \, t \in [0, T]
    \]
    }
\end{frame} 

\begin{frame}{Numerical Methods for American Options}

    {\footnotesize \footnotesize
      This is the Black-Scholes-type PDE for the continuation region. At \( t = T \), 
      the American and European options coincide $V(x, T) = \varphi(x), \; x > 0.$ 
      For very large stock prices (deep out-of-the-money put or call), $\lim_{x \to \infty} V(x, t) = 0.$
      For \( x = 0 \), the option is worth its full strike payoff. $V(0, t) = \varphi(0), \;t \in [0, T]$.
      \vspace{1em}
      
       \pause The key difference between American and European options. At the exercise boundary \( x = b(t) \):
      \begin{itemize}
        \item Value matching (continuity):\[
        V(b(t), t) = \varphi(b(t))
        \]
        This ensures that the value is continuous — no jump when you switch from continuation to exercise.
        \item  \pause Smooth pasting (first derivative continuity): \[
        \frac{\partial V}{\partial x}(b(t), t) = \varphi'(b(t))
        \]
        This ensures the optimality of the stopping rule — if the slope were different, 
        a small move in \( x \) would make it better to stop slightly earlier or later.
      \end{itemize}
    }
\end{frame}


\begin{frame}{Numerical Methods for American Options}

    {\footnotesize \footnotesize
      So the finite-horizon American option satisfies:

        \[
        \begin{cases}
        \mathcal{A} V - rV = 0, & x > b(t), \quad 0 \leq t \leq T\\
        V(b(t), t) = \varphi(b(t)), & (\text{value matching}) \\
        V_x(b(t), t) = \varphi'(b(t)), & (\text{smooth pasting}) \\
        V(x, T) = \varphi(x), & x > 0 \\
        V(0, t) = \varphi(0), & V(\infty, t) = 0
        \end{cases}
        \]
        \vspace{1em}
    \par  \pause The challenge: \( b(t) \) is not known in advance, 
    and the PDE domain itself depends on it. That's why this is called a 
    free-boundary problem — both \( V \) and \( b(t) \) are unknowns.
    \vspace{1em}

    To solve it numerically, you must iterate between solving the PDE for a given boundary guess
     and adjusting the boundary until the smooth pasting condition holds.

    \vspace{1em}
     \pause One way to solve the free boundary problem is to approximate the moving boundary \( b(\cdot) \) by piecewise exponential function. This is used, 
    for example, in Kou, Petrall, and Wang (2005) to price American options in a more general jump-diffusion model.

    }
\end{frame}


\begin{frame}{Numerical Methods for American Options}

    {\footnotesize \footnotesize
    Second, the free boundary problem can be written as a partial differential complementarity problem (without the free boundary) as
    \begin{gather*}
        \mathcal{A} V - rV  \leq  0, \quad x > 0, \quad t \in [0,T) \\
    (\mathcal{A} V - rV) \{ V(x,t) - \varphi(x) \}  =  0, \quad x > 0, \quad t \in [0,T) \\
    V(x,t)  \geq \varphi(x), \quad x > 0, \quad t \in [0,T)
    \end{gather*}
    
    along with the terminal and boundary conditions

    \begin{gather*}
        V(x,T)  =  \varphi(x), \quad x > 0, \\
    \lim_{x \to \infty} V(x,t)  =  0, \quad V(0,t) = \varphi(0), \quad t \in [0,T)
    \end{gather*}

     \pause This is called the complementarity problem because the two inequalities cannot be strict 
    inequalities simultaneously. The partial differential complementarity problem can be solved 
    numerically by converting it to a matrix linear 
    complementarity problem by using the finite difference method.
        }
\end{frame}

\begin{frame}{Numerical Methods for American Options}

    {\footnotesize \footnotesize
    The unknown exercise boundary \( b(t) \) is implicit: wherever \( V = \varphi \), you are in the exercise region;  
    wherever \( \mathcal{A} V - rV = 0 \), you are in the continuation region. You solve for a single function \( V \); the  
    boundary is recovered as the contact set \(\{(x, t) : V = \varphi\}\).
    \vspace{1em}

      \pause Third, the partial differential complementarity problem can be rewritten as a variational inequality problem:

    \[
    \min(-\mathcal{A} V + rV, V(x,t) - \varphi(x)) = 0, \quad x > 0, \quad t \in [0,T) \tag{*}
    \]

    along with the terminal and boundary conditions

    \[
    V(x,T) = \varphi(x), \quad x > 0,
    \]

    \[
    \lim_{x \to \infty} V(x,t) = 0, \quad V(0,t) = \varphi(0), \quad t \in [0,T).
    \]

    \pause  Indeed, there are only two possibility by the partial differential complementarity 
    problem: either \(-\mathcal{A} V + rV > 0\) or \(-\mathcal{A} V + rV = 0\). In the first case we must 
    have \(V(x,t) = \varphi(x)\), and in the second case \(V(x,t) \geq \varphi(x)\). Thus, 
    in both cases we have $(*)$. Hence the partial differential complementarity problem 
    implies the variational inequality problem. 
        }
\end{frame}

\begin{frame}{Numerical Methods for American Options}

    {\footnotesize \footnotesize
    Conversely, $(*)$ implies that \(V(x,t) \geq \varphi(x)\) 
    and \(-\mathcal{A} V + rV \geq 0\). Furthermore, if \(-\mathcal{A}V + rV > 0\) then \(V(x,t) - \varphi(x) = 0\) by $(*)$. 
    Therefore, the variational inequality problem also implies the partial differential complementarity problem.
    \vspace{1em}

       General way to solve the variational inequality or the complementarity problem is to write

    \[
    \max(\mathcal{A} V - rV, \varphi(x) - V(x,t)) = 0, \quad x > 0, \quad t \in [0,T), \tag{**}
    \]

     \pause so that we can use the penalty method:

    \[
    \mathcal{A} V - rV + \lambda \left\{ \varphi(x) - V \right\}^+ = 0, \quad \lambda \to \infty,
    \]
    along with the terminal and boundary conditions.
     To see this, note that \(\varphi(x) - V \leq 0\); otherwise, 
     for all large \(\lambda\) the last term in the penalty method goes to infinity.
      Secondly, if \(\varphi(x) < V\) then the last term is always zero, and we must have \(\mathcal{A} V - rV = 0\). 
    Thirdly, we must have \(\mathcal{A} V - rV \leq 0\). In summary, we recover all the conditions in $(**)$.
        }
\end{frame}

\begin{frame}{The Linear Complementarity Problem}

    {\footnotesize \footnotesize
    An efficient way to solve the partial differential complementarity problem is
    to transform it ot a linear complementarity problem via the finite difference
    method.
    \vspace{1em}

    Consider the following (matrix) linear complementarity problem: 
    Find a vector \( x \in \mathbb{R}^{1 \times n} \)

    \[
    Ax \geq b, \, x \geq c, \, (x - c)^T (Ax - b) = 0,
    \]

    for given \( A \in \mathbb{R}^{n \times n}, \, c, b \in \mathbb{R}^{1 \times n} \), where, 
    for two column vectors \( x \) and \( y, \, x \geq y \) means \( x_i \geq y_i \) for each \( i \).
    The linear complementarity problem has a unique solution for all
     column vectors \( b \) and \( c \) if and only if \( A \in \mathbb{R}^{n \times n} \) is a P-matrix.
    \vspace{1em}
    
     \pause Note that if \( A \) is symmetric, then \( A \) is a P-matrix if and only if \( A \) is positive definite.
     Many matrices that arise in finite-difference and finite-element methods are diagonally dominant.
    }
    
\end{frame}


\begin{frame}{The Linear Complementarity Problem}

    {\footnotesize \footnotesize
      More precisely, the matrix \( A \) is diagonally dominant if

    \[
    |a_{ii}| \geq \sum_{j \neq i} |a_{ij}|, \, \forall i,
    \]

    where \( a_{ij} \) denotes the entry of \( A \) in the \( i \)th row and \( j \)th column. 
    If the above inequality is strict, it is called strictly diagonally dominant. A well-known result is that a 
    symmetric strictly diagonally dominant matrix with real positive diagonal entries is positive definite.

    \vspace{1em}

     \pause The linear complementarity problem can be solved by using many methods, 
    including pivoting methods (e.g., Lemke's algorithm), quadratic programming, 
    successive over relaxation (SOR), projected SOR, etc. There are also several 
    Matlab and Python codes available online.

    }
    
\end{frame}


\begin{frame}{The Linear Complementarity Problem}

    {\footnotesize \footnotesize
     The linear complementarity problem is a particular case of the nonlinear complementarity problem, which is to find a vector \( z \in \mathbb{R}^{1 \times n} \)

\[
f(x) \geq b, \, z \geq c, \, (z - c)^T(f(x) - b) = 0,
\]

where \( f \) is a given function \( \mathbb{R}^n \longmapsto \mathbb{R}^n, \, c, b \in \mathbb{R}^{1 \times n} \). 
One way to solve the nonlinear complementarity problem is to use the linear approximation of \( f(x) \) to get
an iterative algorithm by solving a sequence of the complementarity problem.

\vspace{1em}
 \pause The nonlinear complementarity problem is also related to a variational inequality problem: 
Given a non-empty set \( K \in \mathbb{R}^n \), a function $g: \mathbb{R}^n \longmapsto \mathbb{R}^n, \, b \in \mathbb{R}^{1 \times n},$
\(  \, \text{find a } 
z^* \in K \text{ such that} \)

\[
\min_{y \in K}(y - z^*)^T g(z^*) \geq 0.
\]

It can be shown that with

\[
K = \mathbb{R}_+^{1 \times n}, \, z^* = z - c, \, g(x) = f(x) - b,
\]

the nonlinear complementarity problem and the variational inequality problem have the same solution.

    }
    
\end{frame}


\begin{frame}{Finite Difference Methods}

    {\footnotesize \footnotesize
    The partial differential complementarity problem for \(\psi(x,t)\) without the free boundary,

    \[
    \frac{\partial \psi}{\partial t} + \frac{\sigma^2 x^2}{2} \frac{\partial^2 \psi}{\partial x^2} + rx \frac{\partial \psi}{\partial x} - r\psi \leq 0
    \]

    \[
    \left( \frac{\partial \psi}{\partial t} + \frac{\sigma^2 x^2}{2} \frac{\partial^2 \psi}{\partial x^2} + rx \frac{\partial \psi}{\partial x} - r\psi \right) \{ \psi(t,x) - g(x) \} = 0
    \]

    \[
    \psi(x,t) \geq g(x)
    \]

    \[
    \psi(x,T) = g(x)
    \]

    \[
    \lim_{x \to \infty} \psi(x,t) = 0, \quad \psi(0,t) = g(0),
    \]

   Where $g(x)$ is the payoff function. The first three equations are exactly the variational inequality 
   conditions in differential form.
    For a general payoff \(g(x)\) can be solved by converting it to a 
    linear complementarity problem.

        }
    
\end{frame}

% \begin{frame}

%     {\footnotesize \footnotesize

%     }
    
% \end{frame}
% % {\mathbb{P}^*}
% \tilde{\mathbb{P}}
% {\footnotesize \footnotesize
% }
% \tiny
% \scriptsize
% \footnotesize
% \small
% \normalsize (default)
\end{document}