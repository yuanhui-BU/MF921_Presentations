\documentclass{beamer}

% \usepackage[utf8]{inputenc}
% \usepackage[T1]{fontenc}
\usepackage{lmodern}   % modern Latin Modern fonts
\usepackage{textcomp}  % provides \textquoteright
\usepackage{lmodern} % Latin Modern fonts with T1 shapes


\usepackage{graphicx}
\usepackage{ragged2e} % for generating dummy text
\usepackage[backend=biber,style=authoryear]{biblatex}
% \addbibresource{references.bib}

\usetheme{Madrid}
\usecolortheme{default}
\usefonttheme{professionalfonts} % keeps proper math fonts

\usepackage{amsmath,amssymb,amsfonts} % math symbols (\mathcal, \mathbb, etc.)
\usepackage{mathrsfs}                 % optional: \mathscr for fancy script

% \setbeamercovered{invisible} 
\setbeamercovered{transparent}


\title{MF921 Topics in Dynamic Asset Pricing}
\subtitle{Week 4}
\author{Yuanhui Zhao}
\date{Boston University}

\begin{document}
\frame{\titlepage}
% \begin{frame}
% \frametitle{Outline}
% \tableofcontents
% \end{frame}

\section{Background}
\begin{frame}{Background}

    {\footnotesize \footnotesize
    \par Recall The Double Exponential Jump Diffusion Model:
    \begin{align*}
        \frac{dS(t)}{S(t^{-})} = \mu dt + \sigma dW(t) + d\left(\sum_{i=1}^{N(t)} (V_i - 1)\right)
    \end{align*}
    \par\begin{itemize}
    \item \( W(t) \): Brownian motion under the real-world measure.
    \item \( N(t) \): Poisson process with rate \(\lambda\).
    \item \( V_i \): multiplicative jump sizes, i.i.d. random variables.
    \item \( Y = \log(V) \), the jump sizes follow double exponential law:
    \end{itemize}   
    \begin{align*}
        f_Y(y) = p \eta_1 e^{-\eta_1 y} \mathbf{1}_{y \geq 0} + q \eta_2 e^{\eta_2 y} \mathbf{1}_{y < 0}
    \end{align*}
    \par with parameters:
    \begin{itemize}
        \item \( p, q \geq 0, p + q = 1 \): probabilities of upward/downward jumps.
        \item \(\eta_1 > 1\): rate for upward jumps.
        \item \(\eta_2 > 0\): rate for downward jumps.
    \end{itemize}
    }
    
\end{frame}
\begin{frame}{Background Con.}

    {\footnotesize \footnotesize
    \par For option pricing, we switch to a risk-neutral measure \( P^* \), so that the discounted price process is a martingale:
    \begin{align*}
        E^{P^*}[e^{-rt}S(t)] = S(0)
    \end{align*}
    \par Under \( P^* \), the dynamics adjust:
    \begin{align*}
        \frac{dS(t)}{S(t^-)} = (r - \lambda^*(t)\zeta^*)dt + \sigma dW^*(t) + d\left(\sum_{i=1}^{N^*(t)}(V_i^*-1)\right)
    \end{align*}
    \par where:
    \begin{itemize}
        \item \( W^*(t) \): Brownian motion under \( P^* \),
        \item \( N^*(t) \): Poisson process with intensity \( \lambda^* \),
        \item \( V^* = e^{Y^*} \): jump multiplier with new parameters \( (p^*, q^*, \eta_1^*, \eta_2^*) \),
        \item \( \zeta^* = E^{P^*}[V^*] - 1 = \frac{p^{*}\eta_{1}^{*}}{\eta_{1}^{*}-1} + \frac{q^{*}\eta_{2}^{*}}{\eta_{2}^{*}+1} - 1\) is mean percentage jump size.
    \end{itemize}
    \par The log-price process:  
    \begin{align*}
        X(t) = \log\left(\frac{S(t)}{S(0)}\right) = \left(r - \frac{1}{2}\sigma^2 - 
    \lambda^*\zeta^*\right)t + \sigma W^*(t) + \sum_{i=1}^{N^*(t)} Y_i^*,\;\;X(0)=0
    \end{align*}
    }
\end{frame}
\begin{frame}{Some Useful Formulas Con}

    {\footnotesize \footnotesize
    \par Moment Generating Function of the log-price process, \( X(t) \):
    \begin{align*}
        \mathbb{E}^{\mathbb{P}^*} \left[ e^{\theta X(t)} \right] = \exp\{G(\theta)t\}
    \end{align*}
    \par Where the function $G(\cdot)$ is defined as:
    \begin{align*}
        G(x) = x \left( r - \frac{1}{2} \sigma^2 - \lambda \zeta \right) + \frac{1}{2} x^2 \sigma^2 
        + \lambda \left( \frac{p \eta_1}{\eta_1 - x} + \frac{q \eta_2}{\eta_2 + x} - 1 \right)
    \end{align*}
    \par \textbf{Note}: Lemma 3.1 in Kou and Wang (2003) shows that the 
    equation \( G(x) = \alpha, \forall \alpha > 0 \), has exactly 
    four roots: \(\beta_{1,\alpha}\), \(\beta_{2,\alpha}\), \(-\beta_{3,\alpha}\), and \(-\beta_{4,\alpha}\), where: 
    \begin{align*}
        0 &< \beta_{1,\alpha} < \eta_1 < \beta_{2,\alpha} < \infty\\
        0 &< \beta_{3,\alpha} < \eta_2 < \beta_{4,\alpha} < \infty
    \end{align*}
    \par These roots determine the structure of Laplace transforms for first passage times.
    }
    
\end{frame}
\begin{frame}{Some Useful Formulas Con}

    {\footnotesize \footnotesize
    \par Infinitesimal Generator of the log-price process, \( X(t) \):
    \vspace{1em}
    \begin{align*}
        (\mathcal{L}V)(x) = \frac{1}{2}\sigma^2 V''(x) + \left(r - \frac{1}{2}\sigma^2 
        - \lambda\zeta\right)V'(x) + \lambda \int_{-\infty}^{\infty}\left(V(x+y) - V(x)\right)f_Y(y)\,dy
    \end{align*}
    \vspace{1em}
    \par The generator describes how expectations of functions of \( X(t) \) evolve in time:
    \vspace{1em}
    \begin{align*}
        \frac{d}{dt}\mathbb{E}[V(X_t)] = \mathbb{E}[(\mathcal{L}V)(X_t)]
    \end{align*}
    \vspace{1em}
    \par They provide the mathematical foundation to derive option pricing formulas.



    }
    
\end{frame}

\begin{frame}{Part I}
    \begin{center}
        Option Pricing Under a Double Exponential Jump Diffusion Model
    \end{center}
    \vspace{2em}
    \begin{center}
        S.G. Kou\\
        Hui Wang
    \end{center}
    \vspace{3em}
    \par Proof of two Theorems. The Laplace transform of lookback option and barrier option.
 \end{frame}
\section{Pricing Path-Dependent Options}
\begin{frame}{Lookback Options}

    {\footnotesize \footnotesize
    \par Consider a lookback put option with an initial "prefixed maximum" \( M \geq S(0) \):
    \vspace{1em}
    \begin{align*}
        LP(T) &= \mathbb{E}^{\mathbb{P}^*} \left[ e^{-rT} \left( \max\{M, \max_{0 \leq t \leq T} S(t)\} - S(T) \right) \right]\\
        & =   \mathbb{E}^{\mathbb{P}^*}\left[ e^{-rT} \max\{M, \max_{0 \leq t \leq T} S(t)\} \right] - S(0)\\
    \end{align*}

    \par You need the joint distribution of $\max S(t)$ and $S(T)$, which is complicated for jump processes. 
    Laplace transforms convert a complicated path integral over time into a function of roots of $G(x)$
    which we can solve algebraically.
    }
    
\end{frame}

\begin{frame}{Lookback Options Con.}

    \par Theorem:
    {\footnotesize \footnotesize
    
    \vspace{1em}
    \par Using the notations \(\beta_{1,\alpha+r}\) 
    and \(\beta_{2,\alpha+r}\) as in early silde, the Laplace transform of the lookback put is given by:
    \vspace{1em}
    {\footnotesize \tiny
    \begin{align*}
        \hat{L}(T) = \int_0^\infty e^{-\alpha T} \mathrm{LP}(T)  dT = \frac{S(0)A_\alpha}{C_\alpha} \left( \frac{S(0)}{M} \right)^{\beta_{1,\alpha+r}-1} 
        + \frac{S(0)B_\alpha}{C_\alpha} \left( \frac{S(0)}{M} \right)^{\beta_{2,\alpha+r}-1}  
        + \frac{M}{\alpha+r} - \frac{S(0)}{\alpha}
    \end{align*}
    }
    \vspace{1em}
    \par For all \(\alpha > 0\); here:
    \vspace{1em}
    \begin{align*}
        A_\alpha &= \frac{(\eta_1 - \beta_{1,\alpha+r}) \beta_{2,\alpha+r}}{\beta_{1,\alpha+r} - 1} \\
        B_\alpha &= \frac{(\beta_{2,\alpha+r} - \eta_1) \beta_{1,\alpha+r}}{\beta_{2,\alpha+r} - 1}\\
        C_\alpha &= (\alpha + r) \eta_1 (\beta_{2,\alpha+r} - \beta_{1,\alpha+r})
    \end{align*}

    }
    
\end{frame}

\begin{frame}{Proof}

    {\footnotesize \footnotesize
    \par Lemma :  \(\lim\limits_{y\to\infty}e^{y}\mathbb{P}^{*}[M_{X}(T)\geq y]=0\), \(\forall T\geq 0\). \(M_{X}(T):=\max\limits_{0\leq t\leq T}X(t)\)
    \par [Proof]
    \vspace{1em}
    \par Given \(\theta\in(-\eta_{2},\,\eta_{1})\), $\mathbb{E}^{\mathbb{P}^*}[e^{\theta X(t)}]<\infty$, 
    by stationary independent increments we can get that the process \(\{e^{\theta X(t)-G(\theta)t}; t\geq 0\}\) is a martingale.
     \vspace{1em}
    \par Observe that $G(x)$ is continuous and \(G(1)=r>0\), thus we can fix an \(\theta\in(1,\,\eta_{1})\) such that \(G(\theta)>0\).
     Let $\tau_y = \inf\{t \geq 0 : X(t) \geq y\}$. By Optional Sampling Theorem:
      \vspace{1em}
    \begin{align*}
      1 = \mathbb{E}^{\mathbb{P}^*}[M_{\tau_y \land T}]  \mathbb{E}^{\mathbb{P}^*} \left[ e^{\theta X(\tau_y \land T)} e^{-G(\theta)(\tau_y \land T)} \right] 
      \geq e^{\theta y} e^{-G(\theta)T} \mathbb{P}^* (\tau_y \leq T)
    \end{align*}
     \vspace{1em}
    \par Thus $e^{\theta y} \mathbb{P}^* (\tau_y \leq T) \leq e^{G(\theta)T}$. Since $\theta > 1$, then we have:
     \vspace{1em}
    \begin{align*}
      e^y \mathbb{P}^* (M_X(T) \geq y) = 
      e^{(1-\theta)y} \left[ e^{\theta y} \mathbb{P}^* (\tau_y \leq T) \right] \leq e^{(1-\theta)y} e^{G(\theta)T} \xrightarrow{y \to \infty} 0
    \end{align*}
    \par This will use to justify the boundary term vanishing in the integration-by-parts step later.
    }
    
\end{frame}
\begin{frame}{Proof Con}

    {\footnotesize \footnotesize
    \par Given $s = S(0)$ and $M$ are constants, \(\max\limits_{0 \leq t \leq T} S(t) = s  e^{M_X(T)}\), the lookback put as:
    \vspace{1em}
    \begin{align*}
      LP(T) = \mathbb{E}^{\mathbb{P}^*} \left[ e^{-rT} \max\{M, se^{M_X(T)}\} \right] - s
    \end{align*}
     \vspace{1em}
    \par  Letting \(z=\log(M/s)\geq 0\), define:
     \vspace{1em}
    \begin{align*}
      L(s, M; T)  :&= \mathbb{E}^{\mathbb{P}^*} \left[ e^{-rT} \max\{M, se^{M_X(T)}\} \right] \\
      &= s  \mathbb{E}^{\mathbb{P}^*} \left[ e^{-rT} \max\{e^z, e^{M_X(T)}\} \right]\\
      &  = s  \mathbb{E}^{\mathbb{P}^*} \left[ e^{-rT} \left( e^{M_X(T)} - e^z \right) \mathbf{1}_{\{M_X(T) \geq z\}} \right] + s  e^z  e^{-rT}\\
    \end{align*}
    }
    
\end{frame}
\begin{frame}{Proof Con}

    {\footnotesize \footnotesize
    \par Integration by parts:
    {\footnotesize \tiny
    \begin{align*}
      \mathbb{E}^{\mathbb{P}^*} \left[ e^{-rT} e^{M_X(T)} \mathbf{1}_{\{M_X(T) \geq z\}} \right] 
      & = e^{-rT} \int_{z}^{\infty} e^y  d(1-\mathbb{P}^* [M_X(T) \geq y])\\
      & = -e^{-rT} \int_{z}^{\infty} e^y  d\mathbb{P}^* [M_X(T) \geq y]\\
      & = -e^{-rT} \left\{ \left(-e^y \mathbb{P}^* [M_X(T) \geq y]\right)\bigg|_0^{\infty} - 
      \int_{z}^{\infty} \mathbb{P}^* [M_X(T) \geq y] e^y  dy \right\}\\
      &= -e^{-rT} \left\{ -e^z \mathbb{P}^* [M_X(T) \geq z] - \int_{z}^{\infty} \mathbb{P}^* [M_X(T) \geq y] e^y  dy \right\}\\
      &= \mathbb{E}^{\mathbb{P}^*} \left[ e^{-rT} e^z \mathbf{1}_{\{M_X(T) \geq z\}} \right] + e^{-rT} \int_{z}^{\infty} e^y \mathbb{P}^* [M_X(T) \geq y]  dy;\\
    \end{align*}
    }
    \par Plug back into $ L(s, M; T)$:
    {\footnotesize \tiny
    \begin{align*}
       L(s, M; T) & = s  \mathbb{E}^{\mathbb{P}^*} \left[ e^{-rT} \left( e^{M_X(T)} - e^z \right) \mathbf{1}_{\{M_X(T) \geq z\}} \right] + s  e^z  e^{-rT}\\
       & =  se^{-rT} \int_{z}^{\infty} e^{y} \mathbb{P}^*[M_X(T) \geq y] dy + Me^{-rT} \\
    \end{align*}
    }
    }
    
\end{frame}
\begin{frame}{Proof Con}

    {\footnotesize \footnotesize
    \par Then take Laplace in maturity and use Fubini Theorem:
    % {\footnotesize \tiny
    \begin{align*}
      \int_{0}^{\infty} e^{-\alpha T} L(s, M; T) dT 
    &= s \int_{0}^{\infty} e^{-\alpha T} e^{-rT} \int_{z}^{\infty} e^{y} \mathbb{P}^*[M_X(T) \geq y] dy dT + \frac{M}{\alpha + r} \\
    &= s \int_{z}^{\infty} e^{y} \int_{0}^{\infty} e^{-(\alpha + r)T} \mathbb{P}^*[M_X(T) \geq y] dT dy + \frac{M}{\alpha + r}
    \end{align*}
    
    \par Follows from Kou and Wang (2003) that:
    \[
    \int_{0}^{\infty} e^{-(\alpha + r)T} \mathbb{P}^*[M_X(T) \geq y] dT = A_1 e^{-y\beta_{1,\alpha+r}} + B_1 e^{-y\beta_{2,\alpha+r}}
    \]

    \[
    A_1 = \frac{1}{\alpha + r} \frac{\eta_1 - \beta_{1,\alpha+r}}{\eta_1} \cdot \frac{\beta_{2,\alpha+r}}{\beta_{2,\alpha+r} - \beta_{1,\alpha+r}}, \quad
    B_1 = \frac{1}{\alpha + r} \frac{\beta_{2,\alpha+r} - \eta_1}{\eta_1} \cdot \frac{\beta_{1,\alpha+r}}{\beta_{2,\alpha+r} - \beta_{1,\alpha+r}}.
    \]
    }
    
\end{frame}
\begin{frame}{Proof Con}

    {\footnotesize \footnotesize
    \par Note that \(\beta_{2,\alpha+r} > \eta_1 > 1\), \(\beta_{1,\alpha+r} > \beta_{1,r} = 1\). 
    Therefore:
\begin{align*}
\int_{0}^{\infty} e^{-\alpha T} L(s, M; T) dT 
&= s \int_{z}^{\infty} e^{y} A_1 e^{-y\beta_{1,\alpha+r}} dy + s \int_{z}^{\infty} e^{y} B_1 e^{-y\beta_{2,\alpha+r}} dy + \frac{M}{\alpha + r} \\
&= s A_1 \frac{e^{-z(\beta_{1,\alpha+r}-1)}}{\beta_{1,\alpha+r}-1} + s B_1 \frac{e^{-z(\beta_{2,\alpha+r}-1)}}{\beta_{2,\alpha+r}-1} + \frac{M}{\alpha + r} \\
&= s \frac{A_{\alpha}}{C_{\alpha}} e^{-z(\beta_{1,\alpha+r}-1)} + s \frac{B_{\alpha}}{C_{\alpha}} e^{-z(\beta_{2,\alpha+r}-1)} + \frac{M}{\alpha + r}.
\end{align*}

This yields the Laplace transform we obtained in the Theorem:

    \begin{align*}
    \int_{0}^{\infty} e^{-\alpha T} L(s, M; T) dT 
    &= s \int_{z}^{\infty} e^{y} A_1 e^{-y\beta_{1,\alpha+r}} dy + s \int_{z}^{\infty} e^{y} B_1 e^{-y\beta_{2,\alpha+r}} dy + \frac{M}{\alpha + r} \\
    &= s A_1 \frac{e^{-z(\beta_{1,\alpha+r}-1)}}{\beta_{1,\alpha+r}-1} + s B_1 \frac{e^{-z(\beta_{2,\alpha+r}-1)}}{\beta_{2,\alpha+r}-1} + \frac{M}{\alpha + r} \\
    &= s \frac{A_{\alpha}}{C_{\alpha}} e^{-z(\beta_{1,\alpha+r}-1)} + s \frac{B_{\alpha}}{C_{\alpha}} e^{-z(\beta_{2,\alpha+r}-1)} + \frac{M}{\alpha + r}
    \end{align*}

    \par This yields the Laplace transform we obtained in the Theorem.
    }
    
\end{frame}

\begin{frame}{Barrier Options}


    {\footnotesize \footnotesize
    \par Consider the up-and-in call (UIC) option with the barrier level $H$ ($H>S(0)$):
    \vspace{1em}
    \begin{align*}
        UIC = E^{\mathbb{P}^*}[e^{-rT}(S(T) - K)^+I{\{ \max\limits_{0 \leq t \leq T} S(t) \geq H \}}]
    \end{align*}
    \vspace{1em}
    \par For any given probability \( P \), define:
    \vspace{1em}
    \begin{align*}
        \Psi(\mu, \sigma, \lambda, p, \eta_1, \eta_2; a, b, T):= P[Z(T) \geq a, \max_{0 \leq t \leq T} Z(t) \geq b]
    \end{align*}
    \vspace{1em}
    \par where under \( P, Z(t) \) is a double exponential jump diffusion 
    process with drift \( \mu \), volatility \( \sigma \), and jump rate \( \lambda \), i.e., \( Z(t) = \mu t + \sigma W(t) + \sum_{i=1}^{N(t)} Y_i \), 
    and \( Y \) has a double exponential distribution with 
    density \( f_Y(y) \sim p \cdot \eta_1 e^{-\eta_1 y} 1_{\{y \geq 0\}} + q \cdot \eta_2 e^{y \eta_2} 1_{\{y < 0\}} \).
    }
    
    
\end{frame}
\begin{frame}{Barrier Options Con.}

    \par Theorem:
    \vspace{1em}
    {\footnotesize \footnotesize
    
    \par The price of the UIC option is obtained as:
    \vspace{1em}
    \begin{align*}
        \text{UIC} = & S(0) \Psi \left( r + \frac{1}{2} \sigma^2 - \lambda \zeta, \sigma, \tilde{\lambda}, \tilde{p}, \tilde{\eta}_1, \tilde{\eta}_2; \right. 
        \left. \log \left( \frac{K}{S(0)} \right), \log \left( \frac{H}{S(0)} \right), T \right) \\
        & -Ke^{-rT} \cdot \Psi \left( r - \frac{1}{2} \sigma^2 - \lambda \zeta, \sigma, \lambda, p, \eta_1, \eta_2; \right. 
         \left. \log \left( \frac{K}{S(0)} \right), \log \left( \frac{H}{S(0)} \right), T \right)
    \end{align*}
    \vspace{1em}
    \par where \( \tilde{p} = (p/(1 + \zeta)) \cdot (\eta_1 / (\eta_1 - 1)), \tilde{\eta}_1 
    = \eta_1 - 1, \tilde{\eta}_2 = \eta_2 + 1, \tilde{\lambda} = \lambda(\zeta + 1) \), 
    with \(\zeta = E^{P^*}[V] - 1 = \frac{p\eta_{1}}{\eta_{1}-1} + \frac{q\eta_{2}}{\eta_{2}+1} - 1\). 
    The Laplace transforms of \( \Psi \) is computed explicitly in Kou and Wang (2003).
    }
    
\end{frame}
\begin{frame}{Proof}

    
    {\footnotesize \footnotesize
    \par Theorem 3.1 in Kou and Wang (2003):
    \par For any \(\alpha \in (0, \infty)\), let \(\beta_{1,\alpha}\) and \(\beta_{2,\alpha}\) be the only two positive roots of the equation  
    \[
    \alpha = G(\beta),
    \]  
    where \(0 < \beta_{1,\alpha} < \eta_1 < \beta_{2,\alpha} < \infty\). Then we have the following results concerning the Laplace transforms of \(\tau_b\) and \(X_{\tau_b}\):  

    \[
    \mathbb{E}[e^{-\alpha\tau_b}] = \frac{\eta_1 - \beta_{1,\alpha}}{\eta_1} \frac{\beta_{2,\alpha}}{\beta_{2,\alpha} - \beta_{1,\alpha}} e^{-b\beta_{1,\alpha}} 
    + \frac{\beta_{2,\alpha} - \eta_1}{\eta_1} \frac{\beta_{1,\alpha}}{\beta_{2,\alpha} - \beta_{1,\alpha}} e^{-b\beta_{2,\alpha}}
    \]

    \[
    \mathbb{E}[e^{-\alpha\tau_b} \mathbf{1}_{\{X_{\tau_b} - b > y\}}] = e^{-\eta_1 y} \frac{(\eta_1 - \beta_{1,\alpha})(\beta_{2,\alpha} - \eta_1)}{\eta_1 (\beta_{2,\alpha} 
    - \beta_{1,\alpha})} [e^{-b\beta_{1,\alpha}} - e^{-b\beta_{2,\alpha}}] \text{ for all } y \geq 0
    \]

    \[
    \mathbb{E}[e^{-\alpha\tau_b} \mathbf{1}_{\{X_{\tau_b} = b\}}] = \frac{\eta_1 - \beta_{1,\alpha}}{\beta_{2,\alpha} - \beta_{1,\alpha}} e^{-b\beta_{1,\alpha}} +
     \frac{\beta_{2,\alpha} - \eta_1}{\beta_{2,\alpha} - \beta_{1,\alpha}} e^{-b\beta_{2,\alpha}}
    \]
    \par The definition of $\tau_b$ are same and definition of $X$ equilivent to $Z$ here.




    }
    
\end{frame}
\begin{frame}{Proof Con.}

    
    {\footnotesize \footnotesize
    \par Based on the Theorem 3.1, we can get explicit formula of the Laplace transform 
    for $\Psi(\mu, \sigma, \lambda, p, \eta_1, \eta_2; a, b, T)$ (write as $\Psi(a, b, T)$ for simplicity). Define the Laplace transform in \( T \):

    \[
    \Phi_\alpha(a, b) := \int_0^\infty e^{-\alpha T} \Psi(a, b, T)  dT = 
    \int_0^\infty e^{-\alpha T} \mathbb{E}[1_{\{\tau_b \leq T\}} 1_{\{X(T) \geq a\}}]  dT
    \]

    \par We first use Fubini Theorem and strong Markov property can rewrite the formula, then split by overshoot and use tail-integration eventuality
    we will end up with the following:
    {\footnotesize \tiny
    \begin{align*}
        \Phi_\alpha(a,b) &= A_1 e^{-\beta_{1,\alpha}(a-b)} \left( \mathbb{E} \left[ e^{-\alpha\tau_b} \mathbf{1}_{\{X_{\tau_b}=b\}} \right] +
         \beta_{1,\alpha} \int_0^{a-b} e^{\beta_{1,\alpha}y} \mathbb{E} \left[ e^{-\alpha\tau_b} \mathbf{1}_{\{X_{\tau_b}-b>y\}} \right] dy \right) \\
        &\quad + B_1 e^{-\beta_{2,\alpha}(a-b)} \left( \mathbb{E} \left[ e^{-\alpha\tau_b} \mathbf{1}_{\{X_{\tau_b}=b\}} \right]
        + \beta_{2,\alpha} \int_0^{a-b} e^{\beta_{2,\alpha}y} \mathbb{E} \left[ e^{-\alpha\tau_b} \mathbf{1}_{\{X_{\tau_b}-b>y\}} \right] dy \right)
    \end{align*}
    

    }
    \par where:
    {\footnotesize \tiny
    \begin{align*}
        A_1 = \frac{1}{\alpha} \frac{\eta_1 - \beta_{1,\alpha}}{\eta_1} \cdot \frac{\beta_{2,\alpha}}{\beta_{2,\alpha} - \beta_{1,\alpha}}, 
        \quad B_1 = \frac{1}{\alpha} \frac{\beta_{2,\alpha} - \eta_1}{\eta_1} \cdot \frac{\beta_{1,\alpha}}{\beta_{2,\alpha} - \beta_{1,\alpha}}
    \end{align*}
    }
    }
    
\end{frame}

\begin{frame}{Proof Con.}

    
    {\footnotesize \footnotesize
    \par Back to the proof of the Theorem for Barrier option. First rewrite UIC as:
    {\footnotesize \tiny
    \begin{align*}
        UIC & = E^{\mathbb{P}^*}[e^{-rT}(S(T) - K)^+I{\{ \max\limits_{0 \leq t \leq T} S(t) \geq H \}}]\\
        &= \mathbb{E}^{\mathbb{P}^*} \left[ e^{-rT} S(T) \mathbf{1}_{\{S(T) \geq K, \max_{0 \leq t \leq T} S(t) \geq H\}} \right]\\
        &- Ke^{-rT} \mathbb{P}^* \left[ S(T) \geq K, \max_{0 \leq t \leq T} S(t) \geq H \right]\\
        &= I - Ke^{-rT} \cdot II 
    \end{align*}
    }
    \par Since the log–price \( X(t) = \log(S(t)/S(0)) \):
    {\footnotesize \tiny
    \begin{align*}
        S(T) \geq K \iff X(T) \geq \log \frac{K}{S(0)},
        \quad \max_{0 \leq t \leq T} S(t) \geq H \iff \max_{0 \leq t \leq T} X(t) \geq \log \frac{H}{S(0)}.
    \end{align*}
    }
    \par Based on the definition of $\Psi(\mu, \sigma, \lambda, p, \eta_1, \eta_2; a, b, T)$:
    {\footnotesize \tiny
    \begin{align*}
        II = \Psi \left( r - \frac{1}{2} \sigma^2 - \lambda \zeta, \sigma, \lambda, p, \eta_1, \eta_2; \log \left( \frac{K}{S(0)} \right), \log \left( \frac{H}{S(0)} \right), T \right)
    \end{align*}
    }
    }
    
\end{frame}

\begin{frame}{Proof Con.}

    
    {\footnotesize \scriptsize
    \par For the first term, we can use a change of numeraire argument. More precisely, introduce a new probability \(\tilde{\mathbb{P}}\) defined as:
     {\footnotesize \tiny
    \begin{align*}
        \frac{d \tilde{\mathbb{P}}}{d \mathbb{P}^*} \bigg|_{t=T} = e^{-rT} \frac{S(T)}{S(0)} = e^{-rT} e^{X(T)} 
    = \exp \left\{ \left( -\frac{1}{2} \sigma^2 - \lambda \zeta \right) T + \sigma W(T) + \sum_{i=1}^{N(T)} Y_i \right\}
    \end{align*}
    }
    \par Note that this is a well-defined probability as $\mathbb{E}^{\mathbb{P}^*}[e^{-rt}(S(t)/S(0))] = 1$. Then we 
    reparametrize everything we need under  \(\tilde{\mathbb{P}}\). The Brownian part by Girsanov is $\tilde{W}(t) = W(t) - \sigma t$. 
    The diffusion drift shifts by \(+\sigma^2\), $\tilde{\mu} = r + \frac{1}{2}\sigma^2 - \lambda\zeta$. 
    The modified jump rate and the jump distribution:
    {\footnotesize \tiny
    \begin{align*}
        \text{New rate : } \tilde{\lambda} = \lambda\mathbb{E}^{\mathbb{P}^*}[e^Y] 
        = \lambda(1 + \zeta), \quad \text{where } \zeta = \mathbb{E}^{\mathbb{P}^*}[e^Y] - 1 = \frac{p\eta_1}{\eta_1 - 1} + \frac{q\eta_2}{\eta_2 + 1} - 1
    \end{align*}
    }
    \vspace{-2em}
    {\footnotesize \tiny
    \begin{align*}
        \text{New jump density: }\tilde{f}_Y(y) = \frac{e^y}{\mathbb{E}^{\mathbb{P}^*}[e^Y]} f_Y(y) 
        = \tilde{p}\tilde{\eta}_1 e^{-\tilde{\eta}_1 y}\mathbf{1}_{y\geq 0} + \tilde{q}\tilde{\eta}_2 e^{\tilde{\eta}_2 y}\mathbf{1}_{y<0}
    \end{align*}
    }
     \vspace{-2em}
    {\footnotesize \tiny
    \begin{align*}
        \text{With: } \tilde{\eta}_1 = \eta_1 - 1,\;\tilde{\eta}_2 = \eta_2 + 1,\;\tilde{p} = p\left(\frac{p\eta_1}{\eta_1 - 1}
         + \frac{q\eta_2}{\eta_2 + 1}\right)^{-1} \frac{\eta_1}{\eta_1 - 1},
         \; \tilde{q} = q\left(\frac{p\eta_1}{\eta_1 - 1} + \frac{q\eta_2}{\eta_2 + 1}\right)^{-1} \frac{\eta_2}{\eta_2 + 1}
    \end{align*}
    }
    \par For I:
    \vspace{-2em}
     {\footnotesize \tiny
    \begin{align*}
        I 
        &= S(0) \mathbb{E}^{\mathbb{P}^*} \left[ e^{-rT} \frac{S(T)}{S(0)} \cdot \mathbf{1}_{\{S(T) \geq K, \max_{0 \leq t \leq T} S(t) \geq H\}} \right]\\
        &= S(0) \tilde{\mathbb{P}} \left[ S(T) \geq K, \min_{0 \leq t \leq T} S(t) \leq H \right]\\
        & = S(0) \Psi \left( r + \frac{1}{2} \sigma^2 - \lambda \zeta, \sigma, \tilde{\lambda}, \tilde{p}, \tilde{\eta}_1, \tilde{\eta}_2; \log \left( \frac{K}{S(0)} \right), \log \left( \frac{H}{S(0)} \right), T \right)
    \end{align*}
    }

    }
    
\end{frame}

 \section{Part II}
\begin{frame}{Part II}

    \begin{center}
        Pricing Path-Dependent Options with Jump Risk via Laplace Transforms
    \end{center}
    \vspace{2em}
    \begin{center}
        Steven Kou\\
        Giovanni Petrella\\
        Hui Wang
    \end{center}
    \vspace{3em}
    \par  Derive the Laplace transforms for 
    casepricing of European call and put options. Derive the two dimensional Laplace transform for barrier option.
    
\end{frame}



\begin{frame}{European call and put options}

    {\footnotesize \footnotesize
    \par The price of a European call and put with maturity \( T \) and strike \( K \), is given:
    
    \begin{align*}
        C_T(k) &= e^{-rT} \mathbb{E}^{\mathbb{P}^*} [(S(T) - K)^+] = e^{-rT} \mathbb{E}^{\mathbb{P}^*} \left[ (S(0)e^{X(T)} - e^{-k})^+ \right]\\
        P_T(k')& = e^{-rT} \mathbb{E}^{\mathbb{P}^*} [(K - S(T))^+] = e^{-rT} \mathbb{E}^{\mathbb{P}^*} \left[ (e^{k'} - S(0)e^{X(T)})^+ \right]
    \end{align*}
    
    \par By the change of numeraire argument w.r.t $S(t)$:
    
    \begin{align*}
       C_T(k) &= S(0) \tilde{\Psi}_C(k) - e^{-k} e^{-rT} \Psi_C(k)\\
        P_T(k') &= e^{k'} e^{-rT} \Psi_P(k') - S(0) \tilde{\Psi}_P(k')\\
    \end{align*}
    
    \vspace{-2em}
    \par where: 
    \begin{align*}
      \Psi_C(k) &= \mathbb{P}^*(S(T) \geq e^{-k}),\;\tilde{\Psi}_C(k) = \tilde{\mathbb{P}}(S(T) \geq e^{-k})\\
      \Psi_P(k') &= \mathbb{P}^*(S(T) < e^{k'}), \; \tilde{\Psi}_P(k') = \tilde{\mathbb{P}}(S(T) < e^{k'})
    \end{align*}
    }
    
\end{frame}

\begin{frame}{European call and put options Con.}

    {\footnotesize \footnotesize
    \par Lemma. The Laplace transform with respect to \( k \) of \( C_T(k) \) and with respect to \( k' \) for 
    the put option \( P_T(k') \) are given by:
    \begin{align*}
        \tilde{f}_C(\xi) &:= \int_{-\infty}^{\infty} e^{-\xi k} C_T(k)  dk 
        = e^{-rT} \frac{S(0)^{\xi+1}}{\xi(\xi+1)} \exp(G(\xi+1)T), \quad \xi > 0\\
        \tilde{f}_P(\xi) &:= \int_{-\infty}^{\infty} e^{-\xi k'} P_T(k')  dk' 
        = e^{-rT} \frac{S(0)^{-\xi-1}}{\xi(\xi-1)} \exp(G(-(\xi-1))T), \quad \xi > 1
    \end{align*}
    \par The Laplace transforms with respect to \( k \)  of \( \Psi_C(k) \) and \( k' \) of \( \Psi_P(k') \) are:
    \begin{align*}
        \tilde{f}_{\Psi_C}(\xi) &:= \int_{-\infty}^{\infty} e^{-\xi k} \Psi_C(k)  dk = \frac{S(0)^{\xi}}{\xi} \exp(G(\xi)T), \quad \xi > 0\\
        \tilde{f}_{\Psi_P}(\xi) &:= \int_{-\infty}^{\infty} e^{-\xi k'} \Psi_P(k')  dk' = e^{-rT} \frac{S(0)^{-\xi}}{\xi} \exp(G(-\xi)T), \quad \xi > 0
    \end{align*}
    }
    
\end{frame}


\begin{frame}{European call and put options Con.}

    {\footnotesize \scriptsize
    \par Proof:
    \par The Laplace transform for the call option is:
    \begin{align*}
        \hat{f}_C(\xi) = e^{-rT} \int_{-\infty}^{\infty} e^{-\xi k} \mathbb{E}^{\mathbb{P}^*} \left[ (S(0)e^{X(T)} - e^{-k})^+ \right]  dk
    \end{align*}
    \par Applying the Fubini theorem yields for every \(\xi > 0\):
    \begin{align*}
        \hat{f}_C(\xi) &= e^{-rT} \mathbb{E}^{\mathbb{P}^*} \left[ \int_{-\infty}^{\infty} e^{-\xi k} \left( S(0)e^{X(T)} - e^{-k} \right)^+  dk \right]\\
        &= e^{-rT} \mathbb{E}^{\mathbb{P}^*} \left[ \int_{-X(T)-\log S(0)}^{\infty} e^{-\xi k} \left( S(0)e^{X(T)} - e^{-k} \right)  dk \right]\\
        &= e^{-rT} \mathbb{E}^{\mathbb{P}^*} \left[ S(0)e^{X(T)} e^{\xi (X(T)+\log S(0))} \frac{1}{\xi} - e^{(\xi+1)(X(T)+\log S(0))} \frac{1}{\xi+1} \right]\\
        &= e^{-rT} \frac{S(0)^{\xi+1}}{\xi (\xi+1)} \mathbb{E}^{\mathbb{P}^*} \left[ e^{(\xi+1)X(T)} \right] \overset{\text{MGF}}{=}
          e^{-rT} \frac{S(0)^{\xi+1}}{\xi (\xi+1)} e^{G(\xi+1)T}
    \end{align*}
    
    \par Similary, we can get the  Laplace transform for the put option. 
    }
    
\end{frame}
\begin{frame}{European call and put options Con.}

    {\footnotesize \scriptsize
    \par  The Laplace transforms with respect to \( k \) of \( \Psi_C(k) \):
    \begin{align*}
        \hat{f}_{\Psi_C}(\xi) &= \int_{-\infty}^{\infty} e^{-\xi k} \mathbb{E}^ {\mathbb{P}^*} \left[ \mathbf{1}_{\{S(T) \geq e^{-k}\}} \right]  dk \\
       &  = \int_{-\infty}^{\infty} e^{-\xi k} \mathbb{E}^{\mathbb{P}^*} \left[ \mathbf{1}_{\{k \geq -\log S(T)\}} \right]  dk\\
       & =\mathbb{E}^{\mathbb{P}^*} \left[ \int_{-\log S(T)}^{\infty} e^{-\xi k}  dk \right] \\
       &= \frac{1}{\xi} \mathbb{E}^{\mathbb{P}^*} \left[ S(T)^{\xi} \right] = 
       \frac{S(0)^{\xi}}{\xi} \mathbb{E}^{\mathbb{P}^*} \left[ e^{\xi X(T)} \right] \overset{\text{MGF}}{=} \frac{S(0)^{\xi}}{\xi} e^{G(\xi)T}
    \end{align*}
    \par Similary, we can get the  Laplace transform for $\Psi_P(k')$. 
    }
    
\end{frame}
\begin{frame}{Barrier Options}

    {\footnotesize \footnotesize
    \par Rewrite the pricing formula of up-and-in call option(UIC) in early silde:
    \begin{align*}
        UIC(k,T) = E^{\mathbb{P}^*} \left[ e^{-rT} \left( S(T) - e^{-k} \right)^+ I{[\tau_b < T]} \right]
    \end{align*}
    \par where \( H > S(0) \) is the barrier level, \( k = -\log(K) \) the transformed strike and \( b = \log(H/S(0)) \). In previous paper, we obtain:
    \begin{align*}
        UIC(k,T) = S(0) \tilde{\Psi}_{UI}(k,T) - Ke^{-rT} \Psi_{UI}(k,T)
    \end{align*}
    \par where:
    \begin{align*}
        \Psi_{UI}(k,T) = P^*(S(T) \geq e^{-k},  \tau_b < T), \quad \widetilde{\Psi}_{UI}(k,T) = \widetilde{P}(S(T) \geq e^{-k},  \tau_b < T) 
    \end{align*}
    % \par Remark: Here we will relies on a two-dimensional Laplace transform for botth the option price and the probabilities. The formulae after doing two-dimensional 
    % transforms become much simpler than the one-dimensional formulae in Kou and Wang (2003), which involve many special functions.
    }
    
\end{frame}


\begin{frame}{Barrier Options Con.}

    {\footnotesize \footnotesize
    \par Theorem:  For \(\xi\) and \(\alpha\) such that \(0 < \xi < \eta_1 - 1\) and \(\alpha > \max(G(\xi + 1) - r, 0)\) (such a choice of \(\xi\) and \(\alpha\) is possible 
    for all small enough \(\xi\) as \(G(1) - r = -\delta < 0\)). The Laplace transform with respect to \(k\) and \(T\) of \(UIC(k, T)\) is given by
    \begin{align*}
    \tilde{f}_{UIC}(\xi, \alpha) &= \int_{0}^{\infty} \int_{-\infty}^{\infty} e^{-\xi k - \alpha T} UIC(k, T) dk dT \\
    &= \frac{H^{\xi+1}}{\xi (\xi + 1)} \frac{1}{r + \alpha - G(\xi + 1)} \left( A(r + \alpha) \frac{\eta_1}{\eta_1 - (\xi + 1)} + B(r + \alpha) \right)
    \end{align*}
    \par where

    \begin{align*}
    A(h) &:= E^{\mathbb{P}^*} \left[ e^{-h\tau_b} \mathbf{1}_{\{X(\tau_b) > b\}} \right] =
     \frac{(\eta_1 - \beta_{1,h}) (\beta_{2,h} - \eta_1)}{\eta_1 (\beta_{2,h} - \beta_{1,h})} \left[ e^{-b\beta_{1,h}} - e^{-b\beta_{2,h}} \right]\\
    B(h) &:= E^{\mathbb{P}^*} \left[ e^{-h\tau_b} \mathbf{1}_{\{X(\tau_b = b)\}} \right] =
    \frac{\eta_1 - \beta_{1,h}}{\beta_{2,h} - \beta_{1,h}} e^{-b\beta_{1,h}} + \frac{\beta_{2,h} - \eta_1}{\beta_{2,h} - \beta_{1,h}} e^{-b\beta_{2,h}}
    \end{align*}
    \par with \(b = \log(H/S(0))\).
    }
    


\end{frame}

\begin{frame}{Barrier Options Con.}

    {\footnotesize \footnotesize
    \par  If \(0 < \xi < \eta_1\) and \(\alpha > \max(G(\xi), 0)\) (again this choice of \(\xi\) and \(\alpha\) is possible for 
    all \(\xi\) small enough as \(G(0) = 0\)), then the Laplace transform with respect to \(k\) and \(T\) of \(\Psi_{UI}(k, T)\) is:
     \vspace{1em}
    \begin{align*}
        \tilde{f}_{\Psi_{UI}}(\xi, \alpha) &= \int_0^{\infty} \left( \int_{-\infty}^{\infty} e^{-\xi k - \alpha T} \Psi_{UI}(k, T) dk \right) dT\\
    &= \frac{H^{\xi}}{\xi} \frac{1}{\alpha - G(\xi)} \left( A(\alpha) \frac{\eta_1}{\eta_1 - \xi} + B(\alpha) \right)
    \end{align*}

    \vspace{1em}
    \par The Laplace transforms with respect to \(k\) and \(T\) of \(\tilde{\Psi}_{UI}(k, T)\) is given 
    similarly with \(\tilde{G}\) replacing \(G\) and the functions \(\tilde{A}\) and \(\tilde{B}\) defined similarly. 
    % To perform the inversion, They use the two-sided Euler method(EUL) as in Petrella (2004). Compare with GS method, EUL use standard double
    %  precision, get stable answers, and convergence is much faster.
    }
\end{frame}

\begin{frame}{Barrier Options Con.}

    {\footnotesize \scriptsize
    \par Proof: Follow the pricing formula of UIC and the Fubini theorem:
    \begin{align*}
        \tilde{f}_{UIC}(\xi, \alpha) &= \int_{0}^{\infty} \int_{-\infty}^{\infty} e^{-\xi k - (r + \alpha) T}
         \mathbb{E}^{\mathbb{P}^*} \left[ \left( S(T) - e^{-k} \right)^+ \mathbf{1}_{\{\tau_b < T\}} \right] dkdT\\
        & = \mathbb{E}^{\mathbb{P}^*} \left[ \int_{0}^{\infty} e^{-(r + \alpha) T} \mathbf{1}_{\{\tau_b < T\}}
          \left( \int_{-\log S(T)}^{\infty} e^{-\xi k} \left( S(T) - e^{-k} \right) dk \right) dT \right]\\
          &= \frac{1}{\xi (\xi + 1)} \mathbb{E}^{\mathbb{P}^*} \left[ \int_{0}^{\infty} e^{-(r + \alpha) T} \mathbf{1}_{\{\tau_b < T\}} S(T)^{\xi + 1} dT \right]\\
          {\footnotesize \tiny(\text{$T = \tau_b + t$ with $t>0$})} &=\frac{1}{\xi (\xi + 1)} \mathbb{E}^{\mathbb{P}^*} \left[ e^{-(r + \alpha) \tau_b} \int_{0}^{\infty} e^{-(r + \alpha) t} S(t + \tau_b)^{\xi + 1} 
          dt \right]\\
          &=\frac{1}{\xi (\xi + 1)} \mathbb{E}^{\mathbb{P}^*} \left[ e^{-(r + \alpha) \tau_b} 
            \mathbb{E}^ {\mathbb{P}^*} \left[ \int_{0}^{\infty} e^{-(r + \alpha) t} S(t + \tau_b)^{\xi + 1} dt \middle| \mathcal{F}_{\tau_b} \right] 
           \right]\\
           {\footnotesize \tiny(\text{$i$})}&=\frac{1}{\xi (\xi + 1)} \mathbb{E}^{\mathbb{P}^*} \left[ e^{-(r + \alpha) \tau_b} 
            S(\tau_b)^{\xi+1} \int_0^\infty e^{-(r+\alpha)t} \mathbb{E}^{\mathbb{P}^*} \left[ e^{(\xi+1)X(t)} \right]  dt 
           \right]\\
    \end{align*}
    \vspace{-2em}
    \par Where $i$ is based on $S(\tau_b + t)^{\xi+1} = \left(S(\tau_b)\right)^{\xi+1} \cdot e^{(\xi+1)(X(\tau_b + t) - X(\tau_b))}$ and 
    strong Markov property that $\mathbb{E}^{\mathbb{P}^*} \left[ e^{(\xi+1)(X(\tau_b + t) - X(\tau_b))} \middle| \mathcal{F}_{\tau_b} \right] 
    = \mathbb{E}^{\mathbb{P}^*} \left[ e^{(\xi+1)X(t)} \right]$
    
    }
\end{frame}
\begin{frame}{Barrier Options Con.}

    {\footnotesize \scriptsize
    \par Con.
    \begin{align*}
         {\footnotesize \tiny(\text{MGF})}  &=\frac{1}{\xi (\xi + 1)} \mathbb{E}^{\mathbb{P}^*} \left[ e^{-(r + \alpha) \tau_b} 
            S(\tau_b)^{\xi+1} \int_0^\infty e^{-(r+\alpha)t} e^{G(\xi+1)t} dt 
           \right]\\
        & = \frac{1}{\xi(\xi+1)} \frac{1}{r + \alpha - G(\xi+1)} \mathbb{E}^{\mathbb{P}^*} \left[ e^{-(r+\alpha) \tau_b} S(\tau_b)^{\xi+1} \right]\\
        &= \frac{1}{\xi(\xi+1)} \frac{1}{r + \alpha - G(\xi+1)} \left\{\mathbb{E}^{\mathbb{P}^*} \left[ e^{-(r+\alpha) \tau_b} H^{\xi+1} \mathbf{1}_{\{X(\tau_b)>b\}} \right] 
        \mathbb{E}^{\mathbb{P}^*} \left[ e^{(\xi+1)\chi^+} \right]\right. \\
        & +\left. \mathbb{E}^{\mathbb{P}^*} \left[ e^{-(r+\alpha) \tau_b} H^{\xi+1} \mathbf{1}_{\{X(\tau_b)=b\}} \right] \right\}\\
        & = \frac{H^{\xi+1}}{\xi(\xi+1)} \frac{1}{r + \alpha - G(\xi+1)} 
        \left\{ A(r + \alpha) \frac{\eta_1}{\eta_1 - (\xi+1)} + B(r + \alpha) \right\}
    \end{align*}
    \par Where \( \chi^+ \sim \text{Exp}(\eta_1) \), and $A(h) := \mathbb{E}^{\mathbb{P}^*} [e^{-h\tau_b} \mathbf{1}_{\{X(\tau_b) > b\}}],  B(h) := \mathbb{E}^{\mathbb{P}^*} [e^{-h\tau_b} \mathbf{1}_{\{X(\tau_b) = b\}}].$
    \par Kou-Wang compute \( A(h) \) and \( B(h) \) explicitly (via first-passage Laplace transforms) with \(\beta_{1,h}, \beta_{2,h}\) being the two positive roots of \( G(\beta) = h\):
        \[
        A(h) = \frac{(\eta_1 - \beta_{1,h})(\beta_{2,h} - \eta_1)}{\eta_1 (\beta_{2,h} - \beta_{1,h})} \left( e^{-b\beta_{1,h}} - e^{-b\beta_{2,h}} \right)
        \]
        \[
        B(h) = \frac{\eta_1 - \beta_{1,h}}{\beta_{2,h} - \beta_{1,h}} e^{-b\beta_{1,h}} + \frac{\beta_{2,h} - \eta_1}{\beta_{2,h} - \beta_{1,h}} e^{-b\beta_{2,h}}
        \]
    }
\end{frame}
\begin{frame}{Barrier Options Con.}

    {\footnotesize \scriptsize
    \par For the Laplace transform of the probability \(\Psi_{UI}\), apply the same trick we have: 
    \begin{align*}
        \hat{f}_{\Psi_{UI}}(\xi, \alpha) &= \int_0^\infty \left[ \int_{-\infty}^\infty e^{-\xi k - \alpha T} \cdot
         \mathbb{E}^{\mathbb{P}^*} \left\{ \mathbf{1}_{\{k > -\log(S(T)), \tau_b < T\}} \right\}  dk \right] dT\\
        &= \mathbb{E}^{\mathbb{P}^*} \left\{ \int_{\tau_b}^\infty \left[ \int_{-\log S(T)}^\infty e^{-\xi k - \alpha T}  dk \right] dT \right\}\\
        &= \frac{1}{\xi} \mathbb{E}^{\mathbb{P}^*} \left\{ \int_{\tau_b}^\infty S(T)^{\xi} e^{-\alpha T} dT \right\}\\
        &= \frac{1}{\xi} \mathbb{E}^{\mathbb{P}^*} \left\{ e^{-\alpha \tau_b} \int_0^\infty \{ S(t + \tau_b) \}^{\xi} e^{-\alpha t} dt \right\}\\
        &= \frac{1}{\xi} \mathbb{E}^{\mathbb{P}^*} \left\{ e^{-\alpha \tau_b} \mathbb{E}^{\mathbb{P}^*} \left[ \int_0^\infty S(\tau_b + t)^\xi e^{-\alpha t} 
         dt \middle| \mathcal{F}_{\tau_b} \right]\right\}\\
         & =\frac{1}{\xi} \frac{1}{\alpha - G(\xi)} \mathbb{E}^{\mathbb{P}^*} \left\{ e^{-\alpha \tau_b} [S(\tau_b)]^{\xi} \right\}\\
         & =  \frac{1}{\xi} \frac{1}{\alpha - G(\xi)} \left\{ \mathbb{E}^{\mathbb{P}^*} \left[ e^{-\alpha \tau_b} H^{\xi} \mathbf{1}_{[X(\tau_b) > b]} \right] \mathbb{E}^{\mathbb{P}^*} \left[ e^{\xi \chi^+} \right] 
         + \mathbb{E}^{\mathbb{P}^*} \left[ e^{-\alpha \tau_b} H^{\xi} \mathbf{1}_{[X(\tau_b) = b]} \right] \right\}\\
         &= \frac{H^{\xi}}{\xi} \frac{1}{\alpha - G(\xi)} \left\{ A(\alpha) \frac{\eta_1}{\eta_1 - \xi} + B(\alpha) \right\}
    \end{align*}
    }
\end{frame}

% \begin{frame}

%     {\footnotesize \footnotesize

%     }
    
% \end{frame}
% % {\mathbb{P}^*}
% \tilde{\mathbb{P}}
% {\footnotesize \footnotesize
% }
% \tiny
% \scriptsize
% \footnotesize
% \small
% \normalsize (default)
\end{document}











