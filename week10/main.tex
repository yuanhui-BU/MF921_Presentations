\documentclass{beamer}

% \usepackage[utf8]{inputenc}
% \usepackage[T1]{fontenc}
\usepackage{lmodern}   % modern Latin Modern fonts
\usepackage{textcomp}  % provides \textquoteright
\usepackage{lmodern} % Latin Modern fonts with T1 shapes


\usepackage{graphicx}
\usepackage{ragged2e} % for generating dummy text
\usepackage[backend=biber,style=authoryear]{biblatex}
% \addbibresource{references.bib}

\usetheme{Madrid}
\usecolortheme{default}
\usefonttheme{professionalfonts} % keeps proper math fonts

\usepackage{amsmath,amssymb,amsfonts} % math symbols (\mathcal, \mathbb, etc.)
\usepackage{mathrsfs}    
\usepackage{multicol}             % optional: \mathscr for fancy script

% \setbeamercovered{invisible} 
\setbeamercovered{transparent}


\title{MF921 Topics in Dynamic Asset Pricing}
\subtitle{Week 10}
\author{Yuanhui Zhao}
\date{Boston University}

\begin{document}
\frame{\titlepage}
% \begin{frame}
% \frametitle{Outline}
% \tableofcontents
% \end{frame}


\begin{frame}{Chapter 14}

    {
    \begin{center}
        Chapter 14 Viscosity Solutions and HJB Equations
    \end{center}
    }
    
\end{frame}


\begin{frame}{Definition of Viscosity Solutions}

    {\footnotesize \footnotesize
     We start with an open domain  
\[
\Omega \subset \mathbb{R}^d,
\]
and a function \( u(t, x) \) satisfying a nonlinear second-order PDE  
\[
F(t, x, u(t, x), D_t u(t, x), D_x u(t, x), D_x^2 u(t, x)) = 0, \quad (t, x) \in [0, T) \times \Omega.
\]
Where :  
    \begin{itemize}
        \item \( D_t u \): time derivative \(\partial u / \partial t\)  
        \item \( D_x u = \nabla_x u = \left( \frac{\partial u}{\partial x_1}, \ldots, \frac{\partial u}{\partial x_d} \right)^T \)  
        \item \( D_x^2 u \): the Hessian matrix, with entries \((D_x^2 u)_{ij} = \frac{\partial^2 u}{\partial x_i \partial x_j}\)
    \end{itemize}
     with the terminal condition
    \[
u(T, x) = g(x).
\]
This is typical for backward PDEs (as in HJB equations).  
For infinite-horizon problems, there's no finite terminal time \( T \), so this condition disappears.
    }
   
    
\end{frame}
\begin{frame}{Definition of Viscosity Solutions}

    {\footnotesize \footnotesize
    Before defining viscosity solutions, we require \( F \) to behave "nicely" under 
    perturbations, this ensures the notion of viscosity sub/supersolutions makes sense.
    \begin{itemize}
        \item (i) Ellipticity condition: for symmetric matrices \( M, \hat{M} \):

\[
M \leq \hat{M} \Rightarrow F(t, x, u, q, p, M) \geq F(t, x, u, q, p, \hat{M}), \quad (t, x) \in [0, T) \times \Omega.
\]
So ellipticity ensures \( F \) is nonincreasing in the second derivative argument.
        \item (ii) Parabolicity condition: for the time derivative variable \( q \):  

\[
q \leq \hat{q} \Rightarrow F(t, x, u, q, p, M) \geq F(t, x, u, \hat{q}, p, M), \quad (t, x) \in [0, T) \times \Omega.
\]

    \end{itemize}
    A main motivation for viscosity: many HJB equations (or other nonlinear PDEs) 
    have nonsmooth solutions — the value function \( v(t, x) \) is 
    typically not differentiable.
So we can't plug \( v \) into the PDE in the classical sense (because \( Dv \) and \( D^2v \) don't exist everywhere).

Viscosity theory solves this by testing the PDE against smooth functions that touch \( v \) locally.
    }
   
    
\end{frame}

\begin{frame}{Definition of Viscosity Solutions}

    {\footnotesize \footnotesize
     Definition. Assume both the ellipticity and parabolicity conditions are satisfied.
     \begin{itemize}
        \item A continuous function \(u:\Omega\to\mathbb{R}\) is a viscosity subsolution of the above PDE
if for any \(C^{1}\times C^{2}\)  function \(\phi\) that touches \(u\) from above 
and any local maximum point \((t,y)\in[0,T)\times\Omega\) of \(u-\phi\) we have
    \[
    F(t,y,u(t,y),D_{t}\phi(t,y),D_{x}\phi(t,y),D_{x}^{2}\phi(t,y))\leq 0,
    \]
    and
    \[
    u(T,x)\leq g(x).
    \]  
        \item A continuous function \(u:\Omega\to\mathbb{R}\) is a viscosity 
        supersolution of the above PDE if for any \(C^{1}\times C^{2}\) 
        function \(\phi:[0,T)\times\Omega\to\mathbb{R}\) and any local minimum point \((t,y)\in[0,T)\times\Omega\) of \(u-\phi\) we have
    \[
    F(t,y,u(t,y),D_{t}\phi(t,y),D_{x}\phi(t,y),D_{x}^{2}\phi(t,y))\geq 0,
    \]
    and
    \[
    u(T,x)\geq g(x).
    \]
    \(\phi\) is called a test function. 
    If \(u\) is both a viscosity subsolution and a viscosity supersolution, then \(u\) is called a 
    viscosity solution (necessarily with \(u(T,x)=g(x)\)).
     \end{itemize}
    }
   
    
\end{frame}

\begin{frame}{Definition of Viscosity Solutions}

    {\footnotesize \footnotesize
      Lemma 1
      \begin{itemize}
        \item[(i)] A classical solution is a viscosity solution.
        \item[(ii)] A \(C^{1}\times C^{2}\) viscosity solution is a classical solution.
      \end{itemize}
    \par \textbf{Proof}
    \par $(i)$ Suppose $u$ is a classical solution, i.e., $C^{1}\times C^{2}$ and satisfying the
PDE. For any test function $\phi$ and any local maximum point
$(t,y)\in[0,T)\times \Omega$ of $u-\phi$ we have
\[
D_x u(t,y)=D_x \phi(t,y),\qquad D_x^2 u(t,y)\le D_x^2 \phi(t,y),
\]
because $\Omega\subset\mathbb{R}^d$ is an open domain, and the first–order inequality holds,
\[
D_t u(t,y)\le D_t \phi(t,y),
\]
because the maximum point may be at the boundary $t=0$. Thus,
\begin{align*}
&F\bigl(t,y,u(t,y), D_t\phi(t,y), D_x\phi(t,y), D_x^2\phi(t,y)\bigr)\\
&\qquad= F\bigl(t,y,u(t,y), D_t\phi(t,y), D_x u(t,y), D_x^2\phi(t,y)\bigr)\\
&\qquad\le F\bigl(t,y,u(t,y), D_t\phi(t,y), D_x u(t,y), D_x^2 u(t,y)\bigr)\\
&\qquad\le F\bigl(t,y,u(t,y), D_t u(t,y), D_x u(t,y), D_x^2 u(t,y)\bigr)\\
&\qquad= 0,
\end{align*}
where the first and second inequalities follow from the ellipticity and parabolicity
conditions, respectively. Hence $u$ is a viscosity subsolution. Similarly, $u$ is a
viscosity supersolution. Therefore, $u$ is a viscosity solution.
    }
   
    
\end{frame}

\begin{frame}{Definition of Viscosity Solutions}

    {\footnotesize \footnotesize
    (ii) Suppose $u$ is a viscosity solution and is $C^{1}\times C^{2}$. Then we can take
$\phi=u$. We have any point $(t,y)\in[0,T)\times \Omega$ is both a local maximum point and local
minimum point of $u-\phi$. Thus,
\begin{align*}
&F\bigl(t,y,u(t,y), D_t u(t,y), D_x u(t,y), D_x^2 u(t,y)\bigr)\\
&\qquad= F\bigl(t,y,u(t,y), D_t\phi(t,y), D_x\phi(t,y), D_x^2\phi(t,y)\bigr)\\
&\qquad= 0,
\end{align*}
where the last equality comes from the definitions of subsolution and supersolution.
This shows that $u$ is a classical solution.
\vspace{1em}
\par \textbf{Remark:}
\par $1)$: For the infinite-horizon problem, the first term \( t \) and \( D_t \) is dropped from \( F \), i.e., we have  
    \[
    F(x, u(x), D_x u(x), D_x^2 u(x)) = 0, \quad x \in \Omega,
    \]
    the terminal condition disappears, and we 
    do not need the parabolicity condition. For a finite-horizon deterministic control problem, the term \( D_x^2 \) is dropped from \( F \), i.e., we have  
    }
   
    
\end{frame}
\begin{frame}{Definition of Viscosity Solutions}

    {\footnotesize \footnotesize
    \[
    F(t, x, u(t, x), D_t u(t, x), D_x u(t, x)) = 0, \quad (t, x) \in [0, T) \times \Omega,
    \]
    and we do not need the ellipticity condition. For an infinite-horizon deterministic control problem, we have  
    \[
    F(x, u(x), D_x u(x)) = 0, \quad x \in \Omega,
    \]
    for which both ellipticity and parabolicity conditions are not needed.
    \vspace{1em}
    \par $2)$: For any viscosity subsolution, we can always choose the new test function \(\hat{\phi}\) touches \(u\) at one point, which is the local maximum point of \(u - \hat{\phi}\), and \(\hat{\phi}\) is above the subsolution \(u\). Indeed, for any viscosity subsolution and a test function at any local maximum point \((t_0, y_0) \in [0, T) \times \Omega\) of \(u - \phi\), we have
    \[
    u(t, x) - \phi(t, x) \leq u(t_0, y_0) - \phi(t_0, y_0), \quad \forall (t, x) \in N_{(t_0, y_0)},
    \]
    where \(N_{(t_0, y_0)}\) is a sufficiently small neighborhood of \((t_0, y_0)\) within \([0, T) \times \Omega\). We can define a new test function
    \[
    \hat{\phi}(t, x) = \phi(t, x) + u(t_0, y_0) - \phi(t_0, y_0).
    \]
    Then
    \[
    u(t_0, y_0) = \hat{\phi}(t_0, y_0),
    \]
    \[
    u(t, x) - \hat{\phi}(t, x) = u(t, x) - \phi(t, x) - u(t_0, y_0) + \phi(t_0, y_0) \leq 0, \quad \forall (t, x) \in N_{(t_0, y_0)}.
    \]
    }
   
    
\end{frame}

\begin{frame}{Definition of Viscosity Solutions}

    {\footnotesize \footnotesize
      Thus, the new test function \(\hat{\phi}\) touches \(u\) at one 
      point \((t_0, y_0)\), which is the local maximum point of \(u - \hat{\phi}\), 
      and \(\hat{\phi}\) is above the subsolution \(u\). Similarly, for any supersolution \(u\), 
      there is a test function \(\hat{\phi}\) that touches \(u\) at a local 
      minimum point of \(u - \hat{\phi}\), and \(\hat{\phi}\) is below the supersolution \(u\).
      \vspace{1em}
    \par $3)$: The fact that \( u \) is a viscosity solution to the PDE \( F = 0 \) does
    not imply that \( u \) is a viscosity solution to the PDE \(-F = 0\).
    }
   
    
\end{frame}
% \begin{frame}

%     {\footnotesize \footnotesize

%     }
    
% \end{frame}
% % {\mathbb{P}^*}
% \tilde{\mathbb{P}}
% {\footnotesize \footnotesize
% }
% \tiny
% \scriptsize
% \footnotesize
% \small
% \normalsize (default)
\end{document}