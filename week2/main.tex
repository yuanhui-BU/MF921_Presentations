\documentclass{beamer}

% \usepackage[utf8]{inputenc}
% \usepackage[T1]{fontenc}
\usepackage{lmodern}   % modern Latin Modern fonts
\usepackage{textcomp}  % provides \textquoteright
\usepackage{lmodern} % Latin Modern fonts with T1 shapes


\usepackage{graphicx}
\usepackage{ragged2e} % for generating dummy text
\usepackage[backend=biber,style=authoryear]{biblatex}
% \addbibresource{references.bib}

\usetheme{Madrid}
\usecolortheme{default}
\usefonttheme{professionalfonts} % keeps proper math fonts

\usepackage{amsmath,amssymb,amsfonts} % math symbols (\mathcal, \mathbb, etc.)
\usepackage{mathrsfs}                 % optional: \mathscr for fancy script

% \setbeamercovered{invisible} 
\setbeamercovered{transparent}



% \title{MF921 Topics in Dynamic Asset Pricing}
% \author{Stochastic Analysis \& Stochastic Calculus in Quantitative Finance}
% %\institute{Boston University}
% \date{Yuanhui Zhao\\Week 2\\Boston University}
\title{MF921 Topics in Dynamic Asset Pricing}
\subtitle{Stochastic Analysis \& Stochastic Calculus in Quantitative Finance}
\author{Yuanhui Zhao}
\date{Boston University \\ \;\;Week 2}

\begin{document}
\frame{\titlepage}
% \begin{frame}
% \frametitle{Outline}
% \tableofcontents
% \end{frame}
\section{Part I}
\begin{frame}{Part I, Chapter 15}
    \begin{center}
    Option
    Pricing via the Change of
    Numeraire Argument
    \end{center}
\end{frame}
\section{Change of Numeraire}
\begin{frame}{Change of Numeraire: Motivation and Key Idea}
    \par In option pricing, we usually price under the risk-neutral measure using 
    the money market account $B(t) = e^{rt}$ as the numeraire. 
    But sometimes payoffs become simpler if we change the unit of measurement (the numeraire).
    Instead of measuring in “dollars,” measure in “shares of stock”.
    \vspace{1em}
    \par The key idea is :
    \begin{itemize}
        \item Pick any strictly positive traded asset $N(t)$ as the numeraire.
        \item Then define a new probability measure $\tilde{\mathbb{P}}$ such that 
        $\frac{S(t)}{N(t)} \quad \text{is a martingale under } \tilde{\mathbb{P}}.$ No-arbitrage is preserved.
    \end{itemize}
    \vspace{1em}
    \par We first look at the details how this work (Radon Nikodym derivative \& Girsanov Theorem) and 
    then apply the scheme to price different type of options. 
\end{frame}
\begin{frame}{Change of Numeraire}
    \par Given $(\Omega, \mathcal{F}, (\mathcal{F}_t)_{t \geq 0}, \mathbb{P}^*)$ with $d$-dim Brownian $W$:
    \vspace{1em}
    \begin{itemize}
        \item Money market account (baseline numeraire): $ dB_t =  B_t r_t dt$
        \item Traded asset $S(t)$: $dS_t = S_t(r_t dt + \sigma_t\, dW_t),
    \quad \tfrac{S_t}{B_t} \text{ is a martingale.}$
        \item Derivative pricing rule: for payoff $X_T$ at maturity $T$, $V_0 = \mathbb{E}^{\mathbb{P}^*}\left[\frac{X_T}{B_T}\right]$
    \end{itemize}
    \vspace{1em}
    \par Our goal is to pick another strictly positive traded asset $N(t)$ and
  define a new measure $\tilde{\mathbb{P}}$ such that $\frac{S(t)}{N(t)}$ is a martingale for every traded asset $S(t)$.
\end{frame}
\begin{frame}{Change of Numeraire Con.}

    {\footnotesize \footnotesize
    \par Oberve $\displaystyle \frac{S(t)}{B(t)}$ is a martingale under $\mathbb{P}^*$. 
    We want $\displaystyle \frac{S(t)}{N(t)}$ to be a martingale under $\tilde{\mathbb{P}}$.
    \par Define $\tilde{\mathbb{P}}$ via the Radon$-$Nikodym derivative with respect to $\mathbb{P}^*$:
    \begin{align*}
        \left.\frac{d\tilde{\mathbb{P}}}{d\mathbb{P}^*}\right|_{\mathcal{F}_T} = Z_T : = \;\;\frac{N(T)/B(T)}{N(0)/B(0)}
    \end{align*}
    \par By construction, $\frac{N(T)}{B(T)}$ is a martingale under $\mathbb{P}^*$, $Z_T >0$ and $\mathbb{E}^{\mathbb{P}^*}[Z_T]=1$ and take any payoff $X_T$:
    \begin{align*}
         V(0) \;=\; N(0)\,\mathbb{E}^{\tilde{\mathbb{P}}}\!\left[\frac{X_T}{N(T)}\right] 
         =N(0) \mathbb{E}^{\mathbb{P}^*}\!\left[\frac{X_T}{N(T)}\,Z_T\right] = \mathbb{E}^{\mathbb{P}^*}\!\left[\frac{X_T}{B(T)}\right]
    \end{align*}
    \vspace{0.5em}
    \par So the choice of Radon$-$Nikodym derivative guarantees the prices are consistent under both measures and no arbitrage is preserved.
    }
    
\end{frame}
\begin{frame}{Change of Numeraire Con.}

    {\footnotesize \footnotesize
    \par What is the $dS(t)$ looks like under meausre $\tilde{\mathbb{P}}?$
    \par Note: Under $Q$, we have $\begin{cases}
        dS(t) = r(t) S(t) \,dt + \sigma(t) S(t) \,dW(t)\\dN(t) = r(t) N(t) \,dt + \gamma(t) N(t) \,dW(t)
    \end{cases}$
    \par Denote $\widehat N_t = \frac{N_t}{B_t}$, apply Itô we get
     $\frac{d\widehat N_t}{\widehat N_t}=\gamma_t\,dW_t$, $\widehat N_t
    = \widehat N_0 e^{\left(
      \int_0^t \gamma_s \cdot dW_s
      - \frac{1}{2} \int_0^t \|\gamma_s\|^2 \, ds
    \right)}.$
    \par Oberve that $Z_t  = \frac{\widehat N_t}{\widehat N_0} = e^{\left(
      \int_0^t \gamma_s \cdot dW_s
      - \frac{1}{2} \int_0^t \|\gamma_s\|^2 \, ds
    \right)}$
    \par Girsanov's theorem says: if we define a new measure $\tilde{\mathbb{P}}$ via this $Z_t$, then the process
            \[
        \tilde{W}(t) \;=\; W(t) - \int_{0}^{t}\gamma_sdt
            \]
        is a Brownian motion under $\tilde{\mathbb{P}}$. Substitute into $dS(t)$ to get the $\tilde{\mathbb{P}}$ dynamics:
        \begin{align*}
            dS(t) = S(t)\Big[ (r(t) + \sigma(t) \cdot \gamma(t))\,dt + \sigma(t)  \cdot d\tilde{W}(t) \Big]
        \end{align*}
        \begin{align*}
            S(t)
           = S_0 \exp\!\left(
            \int_0^t \!\Big(r(s) + \sigma(s) \!\cdot\!\gamma(s) - \tfrac12 \|\sigma(s)\|^2\Big) ds
            + \int_0^t \!\sigma(s) \!\cdot\! d\tilde{W}(s)\right)
        \end{align*}
    }
    
\end{frame}
\section{Black-Scholes Formula}
\begin{frame}{Black-Scholes Formula}


    {\footnotesize \footnotesize
    \par Given $r, \sigma$ are constant, we have $S(T) = S(0) \exp\left\{(r - \tfrac{1}{2} \sigma^2) T + \sigma W(T)\right\}$.
    \vspace{1em}
    \par The no-arbitrage price for the call option:
    \begin{align*}
        \psi_c(0)& = \mathbb{E}^{\mathbb{P}^*}(e^{-rT}(S(T) - K)^+)\\&= \mathbb{E}^{\mathbb{P}^*}(e^{-rT}(S(T) - K)I(S(T) \geq K)) \\
        &= \mathbb{E}^{\mathbb{P}^*}(e^{-rT}S(T)I(S(T) \geq K)) - Ke^{-rT}\mathbb{P}^*(S(T) \geq K)\\& = I - Ke^{-rT} \cdot II \
    \end{align*}
    \par For $II$:
    \begin{align*}
        II = \mathbb{P}^*(S(T) \geq K) &= 1 - \Phi \left( \frac{\log(K/S(0)) - (r - \frac{1}{2}\sigma^2)T}{\sigma\sqrt{T}} \right) \\
        &= \Phi \left( \frac{\log(S(0)/K) + (r - \frac{1}{2}\sigma^2)T}{\sigma\sqrt{T}} \right)
    \end{align*}
    \par Note: $\Phi$ is the CDF of the standard normal distribution.
    }
\end{frame}
\begin{frame}{Black-Scholes Formula Con.}

    {\footnotesize \footnotesize
    \par For $I$, we apply the change of numeraire and use stock itself as numeraire. Then based on the eraly definition
    we have $\left.\frac{d\tilde{\mathbb{P}}}{d\mathbb{P}^*}\right|_{\mathcal{F}_T} = Z_T :=  e^{-rT}\frac{S(T)}{S(0)}$
    and $\gamma_t = \sigma$. Therefore, under \(\tilde{\mathbb{P}}\) we have the following dynamics of \(S(t)\):
    \[
    \frac{dS_t}{S_t} = rdt + \sigma^2 dt + \sigma \tilde{dW_t},\;
    S(t) = S(0) \exp \left\{ (r + \sigma^2/2) t + \sigma \tilde{W_t} \right\}
    \]
    \par Then we can rewrite $I$:
    \begin{align*}
        I = S(0)\mathbb{E}^{\mathbb{P}^*} \left( e^{-rT} \frac{S(T)}{S(0)} I(S(T) \geq K) \right) &= S(0) \mathbb{E}^{\tilde{\mathbb{P}}} (I(S(T) \geq K)) \\
        &= S(0) \tilde{\mathbb{P}}(S(T) \geq K)\\
       & = S(0) \Phi \left( \frac{\log(S(0)/K) + (r + \frac{1}{2} \sigma^2) T}{\sigma \sqrt{T}} \right).
    \end{align*}
   \par Putting together, we have the price of the call option is given by:
    \[
    I - Ke^{-rT} \cdot II = S(0) \Phi(d_+) - Ke^{-rT} \Phi(d_-)
    \]
where $d_{\pm} = \frac{\log(S(0)/K) + (r \pm \frac{1}{2} \sigma^2) T}{\sigma \sqrt{T}}.$

    }
\end{frame}

\section{One Dimensional Barrier Options}
\begin{frame}{One Dimensional Barrier Options}

    {\footnotesize \footnotesize
    \par Barrier options are path-dependent derivatives whose payoff is activated (knock-in) or extinguished (knock-out) 
    if the underlying asset crosses a pre-specified barrier. They extend vanilla 
    calls/puts by adding a barrier condition.
    \par We first study continuously monitored barriers and derive 
    Merton's closed-form pricing formulas (1973) for single-barrier options.
    \par Given $(\Omega, \mathcal{F}, (\mathcal{F}_t)_{t \geq 0}, \mathbb{P}^*)$ with 1-dim Brownian $W$. 
    The Market setting following:
    \begin{align*}
        dB(t) =  B(t) r dt\;,\;dS(t) = r S(t)  dt + \sigma S(t)  dW(t)
    \end{align*}
    % \begin{center}
    %     $\begin{cases}
    %     dB_t =  B_t r dt\\ 
    % \end{cases}$
    % \end{center}
    \par A continuously monitored barrier option has 
    payoff = vanilla option payoff \(\times\) indicator of the barrier condition. For example: 
    \begin{itemize}
        \item Up-and-out call: $V_0 = \mathbb{E}^{\mathbb{P}^*} \left[ e^{-rT} (S(T) - K)^+ I{\left\{ \max\limits_{0 \leq t \leq T} S(t) \leq H \right\}} \right], \quad H > S(0)$
        \item Down-and-in put: $ V_0 = \mathbb{E}^{\mathbb{P}^*} \left[ e^{-rT} (K - S(T))^+ I{\left\{ \min\limits_{0 \leq t \leq T} S(t) \leq H \right\}} \right], \quad H < S(0)$
    \end{itemize}
     \par Study the case of the down-and-in call option (DAIC) with strike \( K \), barrier \( H < S(0) \):
    \begin{align*}
        \text{DAIC} = e^{-rT}  \mathbb{E}^{\mathbb{P}^* }
        \left[ (S(T) - K)^+ I{\left\{\min\limits_{0 \leq t \leq T} S(t) \leq H \right\}} \right] 
    \end{align*}
    }
    
\end{frame}
\begin{frame}{One Dimensional Barrier Options Con.}
    
    {\footnotesize \footnotesize
    \par For notation simplicity, denote a drifted Brownian motion:
          \[
          W_{\mu,\sigma}(t) = \mu t + \sigma W(t), \quad M_t = \max_{0 \leq s \leq t} W_{\mu,\sigma}(s).
          \]
    \par Some useful results from the reflection principle for a Brownian motion with a  drift:
    \par (i) When \( x \leq y \), $y > 0,  \sigma > 0$ :
    \begin{itemize}
         {\footnotesize \scriptsize
        \item $P(W_{\mu, \sigma}(t) \leq x,  M_t \geq y) = e^{2\mu y / \sigma^2} 
        \Phi \left( \frac{x - 2y - \mu t}{\sigma \sqrt{t}} \right)$
        \item $P(W_{\mu, \sigma}(t) \leq x,  M_t \leq y) = \Phi \left( \frac{x - \mu t}{\sigma \sqrt{t}} \right) - e^{2\mu y / \sigma^2} 
         \Phi \left( \frac{x - 2y - \mu t}{\sigma \sqrt{t}} \right)$
         }
    \end{itemize}
    \par (ii) When \( x \geq y>0 \), $\sigma > 0$:
    \begin{itemize}
         {\footnotesize \scriptsize
        \item $P(W_{\mu, \sigma}(t) \leq x,  M_t \leq y) = P(M_t \leq y) \nonumber =
            \Phi \left( \frac{y - \mu t}{\sigma \sqrt{t}} \right) - e^{2\mu y / \sigma^2} \Phi 
            \left( \frac{-y - \mu t}{\sigma \sqrt{t}} \right)$
        \item $P(W_{\mu, \sigma}(t) \leq x,  M_t \geq y) = P(W_{\mu, \sigma}(t) \leq x) - P(W_{\mu, \sigma}(t) \leq x,  M_t \leq y) \nonumber 
            = \Phi \left( \frac{x - \mu t}{\sigma \sqrt{t}} \right) - 
            \Phi \left( \frac{y - \mu t}{\sigma \sqrt{t}} \right) + e^{2\mu y / \sigma^2} \Phi \left( \frac{-y - \mu t}{\sigma \sqrt{t}} \right)$
         }
    \end{itemize}
    \par (iii) When \( x \geq y \), $y < 0$, $\sigma > 0$:
    \begin{itemize}{\footnotesize \scriptsize
        \item $P\left( W_{\mu, \sigma}(t) \geq x, \min\limits_{0 \leq s \leq t} W_{\mu, \sigma}(s) \leq y \right) 
        = e^{2\mu y / \sigma^2} \Phi \left( \frac{-x + 2y + \mu t}{\sigma \sqrt{t}} \right)$
        }
    \end{itemize}
    }
\end{frame}
\begin{frame}{One Dimensional Barrier Options Con.}

    {\footnotesize \scriptsize
    \par Back to the valuation of DAIC:
    {\footnotesize \scriptsize
    \begin{align*}
    &\mathbb{E}^{\mathbb{P}^*}\left[e^{-rT}(S(T) - K)^+ I\left(\min_{0 \leq t \leq T} S(t) \leq H\right)\right]\\
    &= \mathbb{E}^{\mathbb{P}^*}\left[e^{-rT}(S(T) - K)I\left(S(T) \geq K, \min_{0 \leq t \leq T} S(t) \leq H\right)\right] \\
    &= \mathbb{E}^{\mathbb{P}^*}\left[e^{-rT}S(T)I\left(S(T) \geq K, \min_{0 \leq t \leq T} S(t) \leq H\right)\right] \\
    &\quad - Ke^{-rT}P^*\left(S(T) \geq K, \min_{0 \leq t \leq T} S(t) \leq H\right) \\
    &= I - Ke^{-rT} \cdot II 
    \vspace{-0.5em}
    \end{align*}
    \par For II:
    \vspace{-0.5em}
    \begin{align*}
    II &= P^*\left(S(T) \geq K, \min_{0 \leq t \leq T} S(t) \leq H\right) \\
    &= P\left\{W_{r-\frac{\sigma^2}{2},\sigma}(T) \geq \log(K/S(0)), \min_{0 \leq t \leq T} W_{r-\frac{\sigma^2}{2},\sigma}(t) \leq \log(H/S(0))\right\} \\
    &= \exp\left\{\frac{2(r - \sigma^2/2)}{\sigma^2} \log(H/S(0))\right\} 
     \cdot \Phi \left(\frac{2 \log(H/S(0)) - \log(K/S(0)) + (r - \sigma^2/2)T}{\sigma \sqrt{T}}\right)\\
     &=(H/S(0))^{\frac{2r}{\sigma^2}-1} \Phi \left(\frac{\log(\{H^2/S(0)\}/K) 
     + (r - \sigma^2/2)T}{\sigma \sqrt{T}}\right)\\
    \end{align*}

    }
    }
\end{frame}
\begin{frame}{One Dimensional Barrier Options Con.}
    
    {\footnotesize \footnotesize
    \par For I, by changing of numeraire we can get:
    \vspace{-0.5em}
    {\footnotesize \scriptsize
    \begin{align*}
    I &= S(0) \mathbb{E}^{\mathbb{P}^*} \left( e^{-rT} \frac{S(T)}{S(0)} \cdot I\left\{S(T) \geq K, \min_{0 \leq t \leq T} S(t) \leq H\right\} \right) \\
    &= S(0) \tilde{P}\left(S(T) \geq K, \min_{0 \leq t \leq T} S(t) \leq H\right) \\
    &= S(0) P\left\{W_{r+\frac{\sigma^2}{2}, \sigma}(T) \geq \log(K/S(0)), \min_{0 \leq t \leq T} W_{r+\frac{\sigma^2}{2}, \sigma}(t) \leq \log(H/S(0))\right\} \\
    &= S(0) \cdot (H/S(0))^{\frac{2r}{\sigma^2}+1} \Phi \left( \frac{\log(\{H^2/S(0)\}/K)+(r+\sigma^2/2)T}{\sigma\sqrt{T}} \right) \\
    &= (H/S(0))^{\frac{2r}{\sigma^2}-1}(H^2/S(0)) \Phi \left( \frac{\log(\{H^2/S(0)\}/K)+(r+\sigma^2/2)T}{\sigma\sqrt{T}} \right)
    \end{align*}
    \par Putting the two terms together, we get $I - Ke^{-rT} \cdot II = (H/S(0))^{\frac{2r}{\sigma^2}-1} \text{BSC}(H^2/S(0))$. 
    Where \(\text{BSC}(x)\) is the Black-Scholes formula for a call option with the initial stock price being \(x\):
    \begin{align*}
        \text{BSC}(x) = x\Phi(d_+) - Ke^{-rT}\Phi(d_-) \text{ with } d_{\pm} = \frac{\log(x/K)+(r\pm\sigma^2/2)T}{\sigma\sqrt{T}}
    \end{align*}
    \vspace{-1em}
    }
    }
\end{frame}
\section{Exchange Options}
\begin{frame}{Exchange Options}

    {\footnotesize \footnotesize
    \par Given $(\Omega, \mathcal{F}, (\mathcal{F}_t)_{t \geq 0}, \mathbb{P}^*)$ 
    with 2-dim independent Brownian, \( W_1(t) \) and \( W_2(t) \). 
    We have two traded assets \( S_1(t) \) and \( S_2(t) \) with the following dynamics:
    \begin{align*}
    \frac{dS_1(t)}{S_1(t)} &= rdt + \sigma_1 dW_1(t) \\
    \frac{dS_2(t)}{S_2(t)} &= rdt + \sigma_2 \left\{ \rho dW_1(t) + \sqrt{1 - \rho^2} dW_2(t) \right\}
    \end{align*}
    \par The exchange option gives the holder the right, but not the obligation, 
    to exchange asset \(S_2\) for asset \(S_1\) at maturity \(T\). The price of this option as following:
    \begin{align*}
        u(0) &= \mathbb{E}^{\mathbb{P}^*}\left[e^{-rT}(S_1(T) - S_2(T))^+\right]\\
        &=S_2(0) \mathbb{E}^{\mathbb{P}^*}\left[ \frac{e^{-rT} S_2(T)}{S_2(0)} \left( \frac{S_1(T)}{S_2(T)} - 1 \right)^+ \right] \\
        &= S_2(0) \mathbb{E}^{\tilde{\mathbb{P}}} \left[ \left( \frac{S_1(T)}{S_2(T)} - 1 \right)^+ \right] \\
        &= S_2(0) \mathbb{E}^{\tilde{\mathbb{P}}} \left[ (F(T) - 1)^+ \right]
    \end{align*}

    }
\end{frame}
\begin{frame}{Exchange Options Con.}

    {\footnotesize \footnotesize
    \par Apply Itô, we have the Radon$-$Nikodym derivative for numeraire:
    \begin{align*}
        \left.\frac{d\tilde{\mathbb{P}}}{d\mathbb{P}^*}\right|_{\mathcal{F}_T}& = Z_T^2 := \frac{e^{-rT} S_2(T)}{S_2(0)} = \exp \left[ \sigma_2 
        \left\{ \rho W_1(T) + \sqrt{1 - \rho^2} W_2(T) \right\} - \frac{T}{2} \sigma_2^2 \right]\\
        \left.\frac{d\hat{\mathbb{P}}}{d\mathbb{P}^*}\right|_{\mathcal{F}_T}& = Z_T^1 := \frac{e^{-rT} S_1(T)}{S_1(0)} 
        =\frac{e^{-rT}S_1(T)}{S_1(0)} = \exp\left(\sigma_1 W_1(T) - \frac{1}{2}\sigma_1^2 T\right)\\
    \end{align*}

    \vspace{-2em}
    \par By Girsanov theorem, under new measure $\tilde{\mathbb{P}}$:
    \begin{align*}
        \tilde{W}_1(t) &= W_1(t) - \rho\sigma_2 t, \;\; \tilde{W}_2(t) = W_2(t) - \sigma_2\sqrt{1-\rho^2} t
    \end{align*}
    \par Apply Itô, we can get $d\ln S_1$, $d\ln S_2$:
    \begin{align*}
        d \ln F(t) &= d \ln S_1(t) - d \ln S_2(t) \\
        &= \left[  - \tfrac{1}{2}\sigma_1^2  - \tfrac{1}{2}\sigma_2^2 + \rho \sigma_1 \sigma_2 \right] dt 
        + (\sigma_1 - \rho \sigma_2) d\tilde{W}_1 - \sigma_2 \sqrt{1-\rho^2} d\tilde{W}ߢ_2.
    \end{align*}
    \par Apply Itô to $g(x) = e^x$ with $x = \ln F(t)$:
    \begin{align*}
        \frac{dF_t}{F_t} = d(\ln F_t) + \frac{1}{2} d<\ln F>_t 
        = (\sigma_1 - \rho\sigma_2) d\tilde{W}_{1t} - \sigma_2 \sqrt{1 - \rho^2} d\tilde{W}_{2t}
    \end{align*}
    }
\end{frame}
\begin{frame}{Exchange Options Con.}
    
    {\footnotesize \footnotesize
    \par Denote $\sigma  = \sqrt{\sigma_1^2 - 2\rho\sigma_1\sigma_2 + \sigma_2^2}$, $\tilde{W}(t) := \frac{1}{\sigma} \left\{ (\sigma_1 - \rho\sigma_2)\tilde{W}_1(t) - \sigma_2\sqrt{1 - \rho^2}\tilde{W}_2(t) \right\}$
    Observe that $\tilde{W}$ is a standard Brownian motion under $\tilde{\mathbb{P}}$. 
    We have $\frac{dF(t)}{F(t)} = \sigma d\tilde{W}(t)$, observe that $F_T = F_0 \exp \left( - \frac{1}{2} \sigma^2 T 
    + \sigma \sqrt{T} Z \right),  Z \sim N(0, 1)$ under $\tilde{\mathbb{P}}$. Similarly, we have 
    $F_T = F_0 \exp \left(  \frac{1}{2} \sigma^2 T 
    + \sigma \sqrt{T} Z \right),  Z \sim N(0, 1)$ under $\hat{\mathbb{P}}$.
    \vspace{1em}
    % and \(\ln F_T \sim N(\ln F_0 - \frac{1}{2} \sigma^2 T, \sigma^2 T)\).
     \par Then we can rewrite $u(0)$:
     
     \begin{align*}
        u(0) &= S_2(0) \mathbb{E}^{\tilde{\mathbb{P}}} \left[ (F(T) - 1)^+ \right]\\
        &= S_2(0) \mathbb{E}^{\tilde{\mathbb{P}}} \left[ (F(T) - 1)I(F(T)>1) \right]\\
        &=S_2(0) \left[ \mathbb{E}^{\tilde{\mathbb{P}}}[F_T I{\{F_T > 1\}}] 
        - \tilde{\mathbb{P}}(F_T > 1)\right] \\
        &=S_2(0) \left[ \mathbb{E}^{\mathbb{P}^*}[\frac{e^{-rT} S_2(T)}{S_2(0)} \frac{S_1(T)}{S_2(T)} I{\{F_T > 1\}}] 
        - \tilde{\mathbb{P}}(F_T > 1)\right] \\
        &=S_2(0) \left[ \frac{1}{S_2(0)}\mathbb{E}^{\mathbb{P}^*}[\frac{e^{-rT} S_1(T)}{S_1(0)} S_1(0) I{\{F_T > 1\}}] 
        - \tilde{\mathbb{P}}(F_T > 1)\right] \\
     \end{align*}
    
    }

\end{frame}
\begin{frame}{Exchange Options Con.}
    
    {\footnotesize \footnotesize   
    \begin{align*}
        = &  S_2(0) \left[ \frac{S_1(0)}{S_2(0)}\mathbb{E}^{\hat{\mathbb{P}}}[ I{\{F_T > 1\}}] 
        - \tilde{\mathbb{P}}(F_T > 1)\right] \\
        = & S_1(0)\hat{\mathbb{P}}[ I{\{F_T > 1\}}] 
        - S_2(0)\tilde{\mathbb{P}}(F_T > 1) \\
        = & S_1(0) \Phi (d_+) - S_2(0) \Phi (d_-) \\
    \end{align*}
    \par Where:
    \begin{align*}
        d_\pm = \frac{\log(F(0)) \pm \frac{1}{2} \sigma^2 T}{\sigma \sqrt{T}} 
        = \frac{\log(S_1(0)/S_2(0)) \pm \frac{1}{2} \sigma^2 T}{\sigma \sqrt{T}}.
    \end{align*}
    \begin{itemize}
        \item (i) If the second asset is cash, or \( S_2(t) = Ke^{-r(T-t)} \), 
        then the formula degenerates to the Black-Scholes formula.
        \item (ii) The hedging strategy is given by long \( \Phi(d_+) \) 
        shares of the first asset and short \( \Phi(d_-) \) shares of the second asset.
    \end{itemize}
    }
    
\end{frame}
\section{Two-Dimensional Barrier Options}
\begin{frame}{Two-Dimensional Barrier Options}
    
    {\footnotesize \footnotesize
    \par Suppose we have two Wiener processes, \( X(t) \) and \( Y(t) \), governed by the following dynamics
    \begin{align*}
    dX(t) &= \mu_1 dt + \sigma_1 dW_1(t), \quad X(0) = 0, \quad \sigma_1 > 0, \\
    dY(t) &= \mu_2 dt + \sigma_2 \left\{ \rho dW_1(t) + \sqrt{1 - \rho^2} dW_2(t) \right\}, \quad Y(0) = 0, \quad \sigma_2 > 0,
    \end{align*}
    \par where the \( W_1 \) and \( W_2 \) are two independent standard Brownian motions.
    \par For \( b > 0 \), consider the first passage time of the process \( Y(t) \):
    \[
    \tau_b^Y = \inf\{t \geq 0: Y(t) = b > 0\}.
    \]
    \par We shall prove that the joint distribution between \( X(T) \) and the first passage time of \( Y(t) \) is given by:
    \vspace{-1em}
    \begin{align*}
    &P(X(T) < a, \tau_b^Y > T) = P\left(X(T) < a, \max_{0 \leq t \leq T} Y(t) < b\right) \\
    = \Phi_2 & \left( \frac{a - \mu_1 T}{\sigma_1 \sqrt{T}}, \frac{b - \mu_2 T}{\sigma_2 \sqrt{T}} ; \rho \right) - 
    e^{2 \mu_2 b / \sigma_2^2} \Phi_2 \left( \frac{a - \mu_1 T - 2 \rho b \sigma_1 / \sigma_2}{\sigma_1 \sqrt{T}}, \frac{-b - \mu_2 T}{\sigma_2 \sqrt{T}} ; \rho \right)
    \end{align*}
    where $b > 0$ and \( \Phi_2(x, y; \rho) \) denotes the bivariate normal distribution given by
    \[
    \Phi_2(x, y; \rho) = P(Z_1 \leq x, Z_2 \leq y),
    \]
    with \( Z_1 \) and \( Z_2 \) being two standard normal random variables with correlation \( \rho \).
    \vspace{1em}

    }

\end{frame}

\begin{frame}{Two-Dimensional Barrier Options Con.}
    
    {\footnotesize \footnotesize
    \par Remark:
    \begin{itemize}
        \item Above equation holds for both \( a \geq b \) and \( a \leq b \), as long as \( b > 0 \).
        That's more general than the 1D reflection principle formulas which needed to be split into 
        separate cases depending on $a \leq b$ or $a \geq b$. 
        \item when \( \rho = 1 \), \( \mu_1 = \mu_2 = \mu \), \( \sigma_1 = \sigma_2 = \sigma \), 
        the two dimensional case reduces to the one-dimensional case, as it becomes:
        {\footnotesize \scriptsize
        \begin{align*}
        &P\left(X(T) < a, \max_{0 \leq t \leq T} X(t) < b\right) \\
        &= \Phi_2 \left( \frac{a - \mu T}{\sigma \sqrt{T}}, \frac{b - \mu T}{\sigma \sqrt{T}}; 1 \right) - e^{2\mu b / \sigma^2} \Phi_2 \left( \frac{a - \mu T - 2b}{\sigma \sqrt{T}}, \frac{-b - \mu T}{\sigma \sqrt{T}}; 1 \right) \\
        &= P \left\{ Z \leq \frac{a - \mu T}{\sigma \sqrt{T}}, Z \leq \frac{b - \mu T}{\sigma \sqrt{T}} \right\}
         - e^{2\mu b / \sigma^2} P \left\{ Z \leq \frac{a - \mu T - 2b}{\sigma \sqrt{T}}, Z \leq \frac{-b - \mu T}{\sigma \sqrt{T}} \right\}\\
        &=P \left\{ Z \leq \min\left\{\frac{a - \mu T}{\sigma \sqrt{T}}, \frac{b - \mu T}{\sigma \sqrt{T}}\right\} \right\}
         - e^{2\mu b / \sigma^2} P \left\{ Z \leq \min\left\{\frac{a - \mu T - 2b}{\sigma \sqrt{T}},\frac{-b - \mu T}{\sigma \sqrt{T}}\right\}  \right\}\\
        \end{align*}
        }
        \par Which incorporates two cases in one dimensional case.
    \end{itemize}
    }

\end{frame}

\begin{frame}{Two-Dimensional Barrier Options Con.}

    {\footnotesize \footnotesize
    Next we proof the formula of the joint distribution between \( X(T) \) 
    and the first passage time of \( Y(t) \):
    \par [Proof]
    \par Consider the case of $\sigma_1 = \sigma_2 = 1$. Define a new process $V(t)$ to decouple $X$ and $Y$:
    \begin{align*}
        V(t) := X(t) - \rho Y(t)
    \end{align*}
    \par First check independence between $V$ and $Y$:
    \begin{align*}
        dV(t)  dY(t)  & = (dX(t) - \rho  dY(t))  dY(t)\\
        &= \left( (1 - \rho^2) dW_1 - \rho \sqrt{1 - \rho^2}  
        dW_2 \right) \cdot \left( \rho dW_1 + \sqrt{1 - \rho^2}  dW_2 \right)\\
        &=(1 - \rho^2) \rho (dW_1)^2 - \rho (1 - \rho^2) (dW_2)^2\\
        &= (1 - \rho^2) \rho dt - (1 - \rho^2) \rho dt = 0
    \end{align*}
    \par  Since \( V(T) = X(T) - \rho Y(T) \), it is Gaussian. Its mean is:
    \begin{align*}
        \mathbb{E}[V(T)] = \mu_1 T - \rho \mu_2 T
    \end{align*}
    }
\end{frame}
\begin{frame}{Two-Dimensional Barrier Options Con.}

    {\footnotesize \footnotesize
    \par Its variance is:
    \begin{align*}
        \text{Var}(V(T)) 
        &= \text{Var}(X(T)) + \rho^2 \text{Var}(Y(T)) - 2\rho \text{Cov}(X(T), Y(T))\\
        &= T + \rho^2 T - 2\rho^2 T = (1 - \rho^2)T\\
    \end{align*}
    \par Thus:
        \[
        V(T) \sim N((\mu_1 - \rho \mu_2)T, (1 - \rho^2)T).
        \]
    \par Incidentally, the same logic applying to two standard normal random variables with correlation \(\rho\) also leads to a representation for the bivariate normal distribution:
    \[
    \Phi_2(\alpha, \beta; \rho) = \int_{z_2=-\infty}^{\beta} \int_{z_1=-\infty}^{\alpha} \frac{1}{\sqrt{1 - \rho^2}} \varphi\left( \frac{z_1 - \rho z_2}{\sqrt{1 - \rho^2}} \right) \varphi(z_2) \, dz_1 \, dz_2.
    \]
    \par Where $\varphi(\cdot)$ is the standard normal density function, $\varphi(z) = \frac{1}{\sqrt{2\pi}}\exp\left(-\frac{z^2}{2}\right).$

    }
    
\end{frame}
\begin{frame}{Two-Dimensional Barrier Options Con.}

    {\footnotesize \footnotesize
    \par Now, in terms of \(V(T)\), we can rewrite \(P(X(T) < a,  \tau_b^Y > T)\) as:
        \begin{align*}
        &P(X(T) < a,  \tau_b^Y > T) \\
        &= \int_{x=-\infty}^{a} \int_{y=-\infty}^{b} P(X(T) \in dx,  Y(T) \in dy,  \tau_b^Y > T) \\
        &\text{Note:the transformation is linear with determinant 1 and the independence of \(V\) and \(Y\)}\\
        &= \int_{x=-\infty}^{a} \int_{y=-\infty}^{b} P(V(T) \in dx - \rho dy) P(Y(T) \in dy,  \tau_b^Y > T)
        \end{align*}
    \par There are two terms inside the integrand. For the first term since \( V(T) \) has a normal distribution with mean \( \mu_1 T - \rho \mu_2 T \) 
    and variance \( (1 - \rho^2)T \), we have:
        \[
        P(V(T) \in dx - \rho dy) = \frac{1}{\sqrt{(1 - \rho^2)T}} \varphi \left( \frac{x - \rho y - \mu_1 T + \rho \mu_2 T}{\sqrt{(1 - \rho^2)T}} \right),
        \]
    }
\end{frame}
    
\begin{frame}{Two-Dimensional Barrier Options Con.}

    {\footnotesize \footnotesize
    \par For the second term, recall the eraly result in one-dim, When \( x \leq y \), $y > 0,  \sigma > 0$:
    \begin{align*}
        P(W_{\mu, \sigma}(t) \leq x,  M_t \leq y) = \Phi \left( \frac{x - \mu t}{\sigma \sqrt{t}} \right) - e^{2\mu y / \sigma^2} 
         \Phi \left( \frac{x - 2y - \mu t}{\sigma \sqrt{t}} \right)
    \end{align*}
    \par We have for all \( y < b,  b > 0 \):
            \[
        P(Y(T) \leq y,  \tau_b^Y > T) = \Phi \left( \frac{y - \mu_2 T}{\sqrt{T}} \right) - e^{2\mu_2 b} \Phi \left( \frac{y - 2b - \mu_2 T}{\sqrt{T}} \right).
        \]
    \par Differentiating the above equation yields:
        \[
        P(Y(T) \in dy,  \tau_b^Y > T) = \frac{1}{\sqrt{T}} \varphi \left( \frac{y - \mu_2 T}{\sqrt{T}} \right) - \frac{1}{\sqrt{T}} e^{2\mu_2 b} \varphi \left( \frac{y - 2b - \mu_2 T}{\sqrt{T}} \right).
        \]
    \par  Plugging the above two terms into:
    \begin{align*}
        \int_{x=-\infty}^{a} \int_{y=-\infty}^{b} P(V(T) \in dx - \rho dy) P(Y(T) \in dy,  \tau_b^Y > T)
    \end{align*}
    }

\end{frame}
\begin{frame}{Two-Dimensional Barrier Options Con.}

    {\footnotesize \footnotesize
    \begin{align*}
        P(X(T)<a, \tau_b^Y >T)=I - II
    \end{align*}
    \par where:
     {\footnotesize \tiny
    \begin{align*}
        I &= \int_{x=-\infty}^a \int_{y=-\infty}^b \frac{1}{\sqrt{(1-\rho^2)T}} \varphi \left( \frac{x-\rho y-\mu_1 T+\rho\mu_2 T}{\sqrt{(1-\rho^2)T}} \right) \frac{1}{\sqrt{T}} \varphi \left( \frac{y-\mu_2 T}{\sqrt{T}} \right) dy dx, \\
        II &= e^{2\mu_2 b} \int_{x=-\infty}^a \int_{y=-\infty}^b \frac{1}{\sqrt{(1-\rho^2)T}} \varphi \left( \frac{x-\rho y-\mu_1 T+\rho\mu_2 T}{\sqrt{(1-\rho^2)T}} \right) \frac{1}{\sqrt{T}} \varphi \left( \frac{y-2b-\mu_2 T}{\sqrt{T}} \right) dy dx,
        \end{align*}
    
     }
     \par and \(\varphi(x)=e^{-x^2/2}/\sqrt{2\pi}\) is the standard normal density function. 
     \vspace{1em}
     \par With $\tilde{x}=\frac{x-\mu_1 T}{\sqrt{T}}, \quad \tilde{y}=\frac{y-\mu_2 T}{\sqrt{T}}$, Then:
     \begin{align*}
        dx = \sqrt{T}  d\tilde{x}, \quad dy = \sqrt{T}  d\tilde{y}
     \end{align*}
     \begin{align*}
        x \leq a \iff \tilde{x} \leq \frac{a - \mu_1 T}{\sqrt{T}}, \quad y \leq b \iff \tilde{y} \leq \frac{b - \mu_2 T}{\sqrt{T}}
     \end{align*}

    }
    
\end{frame}
\begin{frame}{Two-Dimensional Barrier Options Con.}

    {\footnotesize \footnotesize
    \par  we have
    \begin{align*}
    I &= \int_{x=-\infty}^a \int_{y=-\infty}^b \frac{1}{\sqrt{(1-\rho^2)T}} \varphi \left( \frac{\tilde{x}-\rho\tilde{y}}{\sqrt{(1-\rho^2)}} \right)
     \frac{1}{\sqrt{T}} \varphi (\tilde{y}) dy dx \\
    &= \int_{-\infty}^{\frac{a-\mu_1 T}{\sqrt{T}}} \int_{-\infty}^{\frac{b-\mu_2 T}{\sqrt{T}}} \frac{1}{\sqrt{(1-\rho^2)}} 
    \varphi \left( \frac{\tilde{x}-\rho\tilde{y}}{\sqrt{(1-\rho^2)}} \right) \varphi (\tilde{y}) d\tilde{y} d\tilde{x}\\
    \end{align*}
    \par By the conditional-Gaussian factorization, this integrand is exactly the joint pdf of a standard bivariate normal \((Z_1, Z_2)\) with correlation \(\rho\):
\[
f_{Z_1, Z_2} (\tilde{x}, \tilde{y}) = \frac{1}{\sqrt{1 - \rho^2}} \varphi \left( \frac{\tilde{x} - \rho \tilde{y}}{\sqrt{1 - \rho^2}} \right) \varphi (\tilde{y})
\]
Hence the double integral is, by definition,
\[
I = \Phi_2 \left( \frac{a - \mu_1 T}{\sqrt{T}}, \frac{b - \mu_2 T}{\sqrt{T}}; \rho \right)
\]
            }
        
\end{frame}

\begin{frame}{Two-Dimensional Barrier Options Con.}

    {\footnotesize \footnotesize
    \par Similarly, with
        \[
    \hat{x} = \frac{x - \mu_1 T - 2 \rho b}{\sqrt{T}}, \quad \hat{y} = \frac{y - 2b - \mu_2 T}{\sqrt{T}}
    \]
    simplifying the term \( II \) yields
    \begin{align*}
    II &= \int_{x=-\infty}^{a} \int_{y=-\infty}^{b} \frac{1}{\sqrt{(1 - \rho^2)T}} \varphi \left( \frac{\hat{x} - \rho \hat{y}}{\sqrt{(1 - \rho^2)}} \right) \frac{1}{\sqrt{T}} e^{2 \mu_2 b} \varphi (\hat{y}) dy dx \\
    &= \int_{-\infty}^{\frac{a - \mu_1 T - 2 \rho b}{\sqrt{T}}} \int_{-\infty}^{\frac{-b - \mu_2 T}{\sqrt{T}}} \frac{1}{\sqrt{(1 - \rho^2)}} \varphi \left( \frac{\hat{x} - \rho \hat{y}}{\sqrt{(1 - \rho^2)}} \right) e^{2 \mu_2 b} \varphi (\hat{y}) dy dx \\
    &= e^{2 \mu_2 b} \Phi_2 \left( \frac{a - \mu_1 T - 2 \rho b}{\sqrt{T}}, \frac{-b - \mu_2 T}{\sqrt{T}} ; \rho \right),
    \end{align*}
    from which the result follows. The general case can be reduced to this particular case by letting:
    \begin{align*}
        \tilde{X}(t) &= X(t)/\sigma_1, \quad \tilde{Y}(t) = Y(t)/\sigma_2,\\
        \tilde{b} = b/\sigma_2, \quad \tilde{a} &= a/\sigma_1, \quad \tilde{\mu}_1 = \mu_1/\sigma_1, \quad \tilde{\mu}_2 = \mu_2/\sigma_2.
    \end{align*}
    }
    
\end{frame}

\begin{frame}{Two-Dimensional Barrier Options Con.}

    {\footnotesize \footnotesize
    \par Given the joint distribution between \( X(T) \) and the first passage time of \( Y(t) \) by :
    \vspace{-1em}
    {\footnotesize \tiny
    \begin{align*}
    P(X(T) < a, \tau_b^Y > T) &= P\left(X(T) < a, \max_{0 \leq t \leq T} Y(t) < b\right) \\
    = \Phi_2 & \left( \frac{a - \mu_1 T}{\sigma_1 \sqrt{T}}, \frac{b - \mu_2 T}{\sigma_2 \sqrt{T}} ; \rho \right) - 
    e^{\frac{2 \mu_2 b}{\sigma_2^2} } \Phi_2 \left( \frac{a - \mu_1 T - 2 \rho b \sigma_1 / \sigma_2}{\sigma_1 \sqrt{T}}, \frac{-b - \mu_2 T}{\sigma_2 \sqrt{T}} ; \rho \right)
    \end{align*}
    
    \par Remark: 
    \par (i) Using the facts that $P(X(T) > a, \tau_b^Y > T) = P(-X(T) < -a, \tau_b^Y > T)$, 
    that the correlation between \(-X(t)\) and \(Y(t)\) is \(-\rho\),we can show that for \(b > 0\)(the following equation will use for next example 
    to price of an up-and-out option):
    \begin{align*}
    &P(X(T) > a, \tau_b^Y > T) = P\left(-X(T) < - a, \max_{0 \leq t \leq T} Y(t) < b\right) \\
    &= \Phi_2 \left( -\frac{a - \mu_1 T}{\sigma_1 \sqrt{T}}, \frac{b - \mu_2 T}{\sigma_2 \sqrt{T}}; -\rho \right)
    - e^{\frac{2 \mu_2 b}{\sigma_2^2} } \Phi_2 \left( -\frac{a - \mu_1 T - 
    2\rho b\sigma_1/\sigma_2}{\sigma_1 \sqrt{T}}, \frac{-b - \mu_2 T}{\sigma_2 \sqrt{T}}; -\rho \right)
    \end{align*}
    \par (ii) Using the fact that $ P(X(T) < a, \tau_{-b}^Y > T) = P(X(T) < a, \tau_b^{-Y} > T),$
     that the correlation between \(X(t)\) and \(-Y(t)\) is \(-\rho\), we can show that for \(b > 0\):
    \begin{align*}
    P(X(T) < a, \tau_{-b}^Y > T) &= P\left(X(T) < a, \min_{0 \leq t \leq T} Y(t) > -b\right) 
    = P\left(X(T) < a, \max_{0 \leq t \leq T} (-Y(t)) < b\right) \\
    &= \Phi_2 \left( \frac{a - \mu_1 T}{\sigma_1 \sqrt{T}}, \frac{b + \mu_2 T}{\sigma_2 \sqrt{T}}; -\rho \right) 
    - e^{\frac{2 \mu_2 b}{\sigma_2^2} }\Phi_2 \left( \frac{a - \mu_1 T + 2\rho b\sigma_1/\sigma_2}{\sigma_1 \sqrt{T}}, \frac{-b + \mu_2 T}{\sigma_2 \sqrt{T}}; -\rho \right).
    \end{align*}
    }
    }
    
\end{frame}
\begin{frame}{Two-Dimensional Barrier Options Con.}

    {\footnotesize \footnotesize
    \par Let's calculate the price of an up-and-out call option, we have the following set up:
    {\footnotesize \scriptsize
    \begin{align*}
        U_0 = e^{-rT} \mathbb{E}^{\mathbb{P}^*}\left[ (S_1(T) - K)^+ 
       I{\left\{ \max_{0 \leq t \leq T} S_2(t) \leq H \right\}} \right], \quad S_i(t) = S_i(0)e^{X_i(t)}, X_1 
        = X,  X_2 = Y
    \end{align*}
    }
    \par Under the risk-neutral measure \(\mathbb{P}^*\),
    {\footnotesize \scriptsize
    \begin{align*}
    dX(t) &= \mu_1 dt + \sigma_1 dW_1(t), \quad \mu_1 = r - \frac{1}{2} \sigma_1^2, \\
    dY(t) &= \mu_2 dt + \sigma_2 \left\{\rho dW_1(t) + \sqrt{1 - \rho^2} dW_2(t)\right\}, \quad \mu_2 = r - \frac{1}{2} \sigma_2^2,
    \end{align*}
    }
    with \(W_1, W_2\) independent.
    \par Write the barrier level in log space
    \[
    b := \log \frac{H}{S_2(0)}, \quad \text{and } a := \log \frac{K}{S_1(0)}
    \]
    \par Then we have:\\
    {\footnotesize \scriptsize
    \begin{center}
        $\left\{ \max S_2(t) \leq H \right\} = \left\{ \max S_2(0)e^{Y(t)} \leq H \right\}
         = \left\{ \max\limits_{0 \leq t \leq T} Y(t) \leq b \right\}$
    \end{center}
     \begin{center}
        $\{S_1(T) > K\} = \{S_1(0)e^{X(t)} > K\} = \{X(T) > a\}$
    \end{center}
    }
    
    }
    
\end{frame}

\begin{frame}{Two-Dimensional Barrier Options Con.}

    {\footnotesize \scriptsize
    \begin{align*}
        U_0 & = e^{-rT} \mathbb{E}^{\mathbb{P}^*}\left[ (S_1(T) - K)^+ 
       I{\left\{ \max_{0 \leq t \leq T} S_2(t) \leq H \right\}} \right]\\
       &= e^{-rT} \mathbb{E}^{\mathbb{P}^*}\left[ (S_1(T) - K) 
       I{\left\{S_1(T) > K \max_{0 \leq t \leq T} S_2(t) \leq H \right\}} \right]\\
       &= e^{-rT} \mathbb{E}^{\mathbb{P}^*}\left[ S_1(T) I{\left\{ X(T) > a, \max\limits_{0 \leq t \leq T} Y(t) \leq b \right\}} \right] 
       - Ke^{-rT} {\mathbb{P}^*}(X(T) > a, \max\limits_{0 \leq t \leq T} Y(t) \leq b)\\
       &= I - Ke^{-rT} \cdot II 
    \end{align*}
    
    \par Apply Itô, we have the Radon$-$Nikodym derivative for numeraire:
    \begin{align*}
        \left.\frac{d\tilde{\mathbb{P}}}{d\mathbb{P}^*}\right|_{\mathcal{F}_T}& = Z_T := \frac{e^{-rT} S_1(T)}{S_1(0)}
         = \exp \left[ \sigma_1W_1(T) - \frac{1}{2}\sigma_1^2 T \right]\\
    \end{align*}
    \par \vspace{-2em}
    \par By Girsanov theorem, under new measure $\tilde{\mathbb{P}}$:
    \begin{align*}
        \tilde{W}_1(t) &= W_1(t) - \sigma_1 t, \;\; \tilde{W}_2(t) = W_2(t)
    \end{align*}
    \par And the drifts of $X$ and $Y$ under $\tilde{\mathbb{P}}$ become:
    \begin{align*}
        \mu_1^{(1)} = \mu_1 + \sigma_1^2 = r + \frac{1}{2}\sigma_1^2, 
        \quad \mu_2^{(1)} = \mu_2 + \rho\sigma_1\sigma_2 = r - \frac{1}{2}\sigma_2^2 + \rho\sigma_1\sigma_2
    \end{align*}
    }
\end{frame}

\begin{frame}{Two-Dimensional Barrier Options Con.}

    {\footnotesize \footnotesize
    \par For I, using change of numeraire and the formula in remark (i) :
    {\footnotesize \tiny
    \begin{align*}
        I = & \mathbb{E}^{\mathbb{P}^*}\left[ e^{-rT}S_1(T) I{\left\{ X(T) > a, \max\limits_{0 \leq t \leq T} Y(t) \leq b \right\}} \right] \\
        = & \mathbb{E}^{\mathbb{P}^*}\left[ \frac{e^{-rT}S_1(T)}{S_1(0)}S_1(0) I{\left\{ X(T) > a, \max\limits_{0 \leq t \leq T} Y(t) \leq b \right\}} \right] \\
        = &  S_1(0) \tilde{\mathbb{P}}{\left\{ X(T) > a, \max\limits_{0 \leq t \leq T} Y(t) \leq b \right\}} \\
        = S_1(0) & \left[\Phi_2 \left( -\frac{a - \mu_1^{(1)} T}{\sigma_1 \sqrt{T}}, \frac{b - \mu_2^{(2)} T}{\sigma_2 \sqrt{T}}; -\rho \right)
        - e^{\frac{2 \mu_2^{(2)} b}{\sigma_2^2} } \Phi_2 \left( -\frac{a - \mu_1^{(1)} T - 
        2\rho b\sigma_1/\sigma_2}{\sigma_1 \sqrt{T}}, \frac{-b - \mu_2^{(2)} T}{\sigma_2 \sqrt{T}}; -\rho \right)\right]\\
    \end{align*}
    }
    \par For II, apply the formula in remakr(i) directly:
    {\footnotesize \tiny
    \begin{align*}
        II = & {\mathbb{P}^*}(X(T) > a, \max\limits_{0 \leq t \leq T} Y(t) \leq b)\\
        = & \Phi_2 \left( -\frac{a - \mu_1 T}{\sigma_1 \sqrt{T}}, \frac{b - \mu_2 T}{\sigma_2 \sqrt{T}}; -\rho \right)
        - e^{\frac{2 \mu_2 b}{\sigma_2^2} } \Phi_2 \left( -\frac{a - \mu_1 T - 
        2\rho b\sigma_1/\sigma_2}{\sigma_1 \sqrt{T}}, \frac{-b - \mu_2 T}{\sigma_2 \sqrt{T}}; -\rho \right)
    \end{align*}
    }
    }
    
\end{frame}

\section{Part II}
\begin{frame}{Part II: Chapter 17}
    \begin{center}
        Introduction to Stochastic
    Calculus for Jump
    Processes
    \end{center}
\end{frame}

\section{Poisson Process}
\begin{frame}{Counting Processes}
    \par A counting process \( N(t) \): tracks the number of events up to time \( t \). 
    We have the following Key properties of any counting process:
    \begin{itemize}
        \item \( N(t) \geq 0 \)
        \item Takes integer values
        \item Non-decreasing \( (N(s) \leq N(t) \) if \( s < t \))
        \item Increment \( N(t) - N(s) \) counts events in \( (s, t] \)
    \end{itemize}
    \par To make analysis tractable, the following assumptions are typically imposed:
     \begin{itemize}
        \item Independent increments: the number of events occurring in disjoint time
            intervals is statistically independent
        \item Stationary increments: distribution of increments depends only on interval length, not location.
    \end{itemize}
    \par These two properties are also underpin the definition of Brownian motion.
\end{frame}

\begin{frame}{Two Equivalent Definitions of Poisson Processes}

    {\footnotesize \footnotesize
    \par First definition(I) of a Poisson process:
    \par Counting process $\{N(t), t \geq 0\}$ with rate $\lambda > 0$ such that :
    \begin{enumerate}
        \item $N(0) = 0$  
        \item The process exhibits independent and stationary increments  
        \item For each $t\geq 0$, the random variable $N(t)$ follows a Poisson distribution:
        \[
        \mathbb{P}[N(t) = n] = e^{-\lambda t} \frac{(\lambda t)^n}{n!} \quad \text{for } n = 0, 1, 2, \ldots, \;\; \mathbb{E}[N(t)] = \lambda t
        \]
    \end{enumerate}
    \par Second definition(II) of a Poisson process:
    \par Counting process $\{N(t), t \geq 0\}$ with rate $\lambda > 0$ such that :
    \begin{enumerate}
        \item $N(0) = 0$  
        \item The process exhibits independent and stationary increments
        \item $\mathbb{P}[N(h) = 1] = \lambda h + o(h)$
        \item $\mathbb{P}[N(h) \geq 2] = o(h)$
    \end{enumerate}
    }
\end{frame}

\begin{frame}{Two Equivalent Definitions of Poisson Processes Con.}

     {\footnotesize \scriptsize
    \par We show the equivalence of the two definitions:
    \par Proof:
    \par (i) $I \Rightarrow II$
    \par For small $h$, use Taylor expansion of the exponential:
    \begin{align*}
         \mathbb{P}(N(h) = 1) = e^{-\lambda h}(\lambda h) = (1 - \lambda h + \frac{1}{2}(\lambda h)^2 + o(h^2))(\lambda h) = \lambda h + o(h)
    \end{align*}
    \par Similarly:
    \begin{align*}
        \mathbb{P}(N(h) \geq 2) = 
        1 - \mathbb{P}(0) - \mathbb{P}(1) = 1 - \left(1 - \lambda h 
        + \frac{1}{2}(\lambda h)^2 + o(h^2)\right) - \left(\lambda h + o(h)\right) = o(h)
    \end{align*}
    \par (ii) $II \Rightarrow I$(Some Intuition)
    \par Partition $(0, t]$ into $m$ small subintervals of length $h = t/m$ and define increments $X_i = N(ih) - N((i - 1)h)$:
    \begin{align*}
        N(t) = \sum_{i=1}^m X_i
    \end{align*}
    \par Let $m \rightarrow \infty$, by the small-interval conditions:
    \begin{align*}
        \mathbb{P}\{X_i = 1\} = \lambda h + o(h), \quad \mathbb{P}\{X_i = 0\} = 1 - \lambda h + o(h), \quad \mathbb{P}\{X_i \geq 2\} = o(h)
    \end{align*}
     }
\end{frame}
\begin{frame}{Two Equivalent Definitions of Poisson Processes Con.}


    {\footnotesize \footnotesize
    \par So each $X_i$ behaves like a Bernoulli$(\lambda h)$. due to the independent increments property, the \( X_i \) are mutually independent. 
    % Since the sum of independent and identically distributed Bernoulli random variables follows a binomial distribution, 
    It follows that \( N(t) \) is approximately binomial with parameters \( m \) and \( p = \lambda h \).
    \vspace{1em}
    \par As \( m \to \infty \) and \( p \to 0 \), the classical Poisson approximation to the binomial distribution implies
     that \( N(t) \) converges in distribution to a Poisson random variable with rate

        \[
        mp = m\lambda h = m\lambda \frac{t}{m} = \lambda t,
        \]
    \vspace{1em}
    \par  which precisely corresponds to the distribution given in equation of the requirement three in the first definition. 

    \vspace{1em}
    \par \textbf{Remark}: Powerful tool for modeling infrequent extreme events. In financial contexts, 
    poisson processes can capture market shocks and discontinuities missed by continuous-path models. Important for pricing derivatives sensitive to jump risk.

    }
\end{frame}


\begin{frame}{Interarrival and Waiting Times}


     {\footnotesize \footnotesize
    \par Interarrival time \( T_n \) is the time between the \((n - 1)\)st and \(n\)th event. Waiting time \( S_n \) is the time of the \(n\)th event:
    \begin{align*}
        S_n = \sum_{i=1}^n T_i
    \end{align*}
    \par Key properties:
    \vspace{1em}
    \begin{itemize}
        \item \( S_n \leq t \iff N(t) \geq n \). (\(n\)th event occurs by time \(t\) $\iff$ at least \(n\) arrivals by \(t\)).\vspace{1em}
        \item Alternative representation of counting process:
    
        \[
        N(t) = \max\{n : S_n \leq t\} = \min\{n : S_{n+1} > t\}.
        \]
    \end{itemize}
     }
    
\end{frame}



\begin{frame}{The Third Definition of Poisson Processes}


    {\footnotesize \footnotesize
    \par Suppose the counting process \( N(t) \) satisfies the second definition of a Poisson process. 
    Demonstrate that the interarrival times \( T_n,  n \geq 1 \), are independent exponential 
    random variables with rate \(\lambda\). 
    Consequently, the expected value of the first interarrival time is \( \mathbb{E}[T_1] = 1/\lambda \).
    \vspace{1em}
    \par [Proof]
    \par Npte that since \( T_1 > t \) means that there is no event before time \( t \). Therefore, we have  
        \[
        P(T_1 > t) = P(N(t) = 0) = e^{-\lambda t}.
        \]  
    \par Furthermore,  
    \[
    P(T_2 > t|T_1 = s) = P(\text{no events in } (s, s + t]|T_1 = s) = P(\text{no events in } (s, s + t]) = e^{-\lambda t},
    \]  
    \par Thus, conditioning on \( T_1 \) we have  
    \[
    P(T_2 > t) = \int_{0}^{\infty} P(T_2 > t|T_1 = s)f_{T_1}(s)ds = \int_{0}^{\infty} e^{-\lambda t}f_{T_1}(s)ds
     =  e^{-\lambda t}
    \]  
     \par Therefore, \( T_2 \) has an exponential distribution with same rate $\lambda$ and  \( T_1 \) and \( T_2 \) are independent.  
    Repeating the same argument, we can show $T_3, T_4.....$
    }
    
\end{frame}

\begin{frame}{The Third Definition of Poisson Processes Con.}

    {\footnotesize \footnotesize
   \par  Since the sum of independent and identically distributed exponential random variables follows a gamma distribution:
   \begin{align*}
    S_n = \sum_{i=1}^n T_i \sim \text{Gamma}(n, \lambda),\;\;\; f_{S_n}(t) = \frac{\lambda e^{-\lambda t} (\lambda t)^{n-1}}{\Gamma(n)}
   \end{align*}
    \par  Third equivalent definition(III), poisson process can be defined via arrival times:
    \begin{align*}
        N(t) = \max\{n : S_n \leq t\} = \min\{n : S_{n+1} > t\} \tag{*}
    \end{align*}
    \par With \( T_i \overset{iid}{\sim} \text{Exp}(\lambda) \).
    \par Note: If \( N(t) \) satisfies Definition II, then \( T_i \) are exponential \(\Rightarrow\) $(*)$. 
    Conversely, if we build \( N(t) \) from i.i.d. exponential interarrivals via $(*)$, 
    then \( N(t) \) has Poisson(\(\lambda t\)) law \(\Rightarrow\) Definition I.
    \par Moreover, a Poisson process can be expressed as \( N(t) = M(t) - 1 \), where $M(t)$ corresponds to a special case of a …rst passage
    time process, defned as:
    \begin{align*}
        M(t) = \min\{n > 0 : S_n > t\}, \quad S_0 = 0, \quad t > 0
    \end{align*}

    }
    
\end{frame}



% \begin{frame}
%     {\footnotesize \footnotesize

%     }
    
% \end{frame}
% % {\mathbb{P}^*}
% \tilde{\mathbb{P}}
% {\footnotesize \footnotesize
% }
% \tiny
% \scriptsize
% \footnotesize
% \small
% \normalsize (default)
% \large
% \Large
% \LARGE
% \huge
% \Huge
\end{document}